\uuid{Su3X}
\titre{Approximation d'une intégrale}
\theme{simulation, méthode de Monte Carlo}
\auteur{Maxime Nguyen}
\organisation{AMSCC}

\begin{SaveVerbatim}{Su3Xpython}
	n=1000
	S=0
	for i in range(n):
	u = rand()
	S= S + sqrt(-2*log(1-u))
	print("Valeur approchée de I = ")
	print(S/n)
\end{SaveVerbatim}


\contenu{
\texte{ 	On suppose dans cet exercice que l'on dispose d'une fonction \texttt{normale()} pour simuler la variable $\mathcal{N}(0,1)$ et de \texttt{rand()} pour simuler la loi  $\mathcal{U}([0;1])$. 
	
	Soient $(X_i)_{i \geq 1}$ une suite de variables aléatoires indépendantes et identiquement distribuées (iid) selon la loi $\mathcal{N}(0,1)$. 
	
	On cherche une valeur approchée de l'intégrale suivante : 
	$$I = \int_{0}^{+\infty} x^2 e^{-\frac{x^2}{2}} dx$$ }
	
	\begin{enumerate}
		\item \question{ Pour tout $i \geq 1$, exprimer $\mathbb{E}(X_i^2)$ sous forme d'intégrale.  }
		\reponse{On applique le théorème de transfert : 
			\begin{align*}
				\mathbb{E}(X_i^2) &= \int_{-\infty}^{+\infty} x^2 e^{-\frac{x^2}{2}} \frac{1}{\sqrt{2\pi}} dx \\
				&= 2 \int_{0}^{+\infty} x^2 e^{-\frac{x^2}{2}} \frac{1}{\sqrt{2\pi}} dx \\
				&= \sqrt{\frac{2}{\pi}}I
			\end{align*}
		}
		\item \question{ Montrer que $\displaystyle \frac{1}{n}\sum_{i=1}^n X_i^2 \xrightarrow[n \to \infty]{{\rm{p.s.}}} \sqrt{\frac{2}{\pi}}I$. }
		\reponse{On applique la loi forte des grands nombres à la suite de variables aléatoires iid $(X_i^2)_{i \geq 1}$ : $\displaystyle \frac{1}{n}\sum_{i=1}^n X_i^2 \xrightarrow[n \to \infty]{{\rm{p.s.}}} \mathbb{E}(X_1^2) =  \sqrt{\frac{2}{\pi}}I$.}
		%	\item En utilisant la question précédente, détailler une méthode permettant d'approcher  $I$.
		%En déduire la description d'un programme pour approcher la valeur de $I$ (on ne demande pas de syntaxe particulière).
		\item 	\question{ Soit $\tilde{f} \colon x \mapsto xe^{-\frac{x^2}{2}} \textbf{1}_{[0;+\infty[}(x)$. 
		Vérifier que $\tilde{f}$ définit une densité de probabilité et exprimer la fonction de répartition associée à cette loi.  }
		\reponse{On constate que $\tilde{f}(x) \geq 0$ pour tout $x \geq 0$. Par ailleurs, 
			\begin{align*}
				\int \tilde{f}(x) dx &= \int_{0}^{+\infty}  xe^{-\frac{x^2}{2}} dx \\
				&=  \left[ -e^{-\frac{x^2}{2}}\right]_0^{+\infty} \\
				&= 1
			\end{align*}	
			Donc $\tilde{f}$ est bien une fonction densité de probabilité. 
			
			Soit $\tilde{F}$ la fonction de répartition associée : si $t \leq 0$, il est clair que $\tilde{F}(t) = 0$. Soit $t>0$ : 
			\begin{align*}
				\tilde{F}(t) &= \int_{0}^{t}  xe^{-\frac{x^2}{2}} dx \\
				&= 1 - e^{-\frac{t^2}{2}}
			\end{align*}
		}
		\item \question{ Soit $U$ une variable aléatoire suivant une loi $\mathcal{U}([0;1])$. Démontrer que la variable aléatoire $\sqrt{-2\ln(1-U)}$ admet $\tilde{f}$ comme densité. En déduire une commande permettant de simuler la loi d'une variable aléatoire admettant $\tilde{f}$ pour densité.   }
		\reponse{Plusieurs méthodes sont possibles. On peut utiliser le théorème d'identification pour identifier la loi de la variable aléatoire $\sqrt{-2\ln(1-U)}$. On peut aussi remarquer que si $u \in [0;1]$ alors $u=\tilde{F}(x)$ pour $x = \sqrt{-2\ln(1-x)}$. D'après le théorème de simulation par inversion de la fonction de répartition, cela prouve que la fonction de répartition de la variable $\sqrt{-2\ln(1-U)}$ est $\tilde{F}$, ce qui revient à dire que sa loi admet $\tilde{f}$ pour densité. }
		\item \question{ En remarquant que $\displaystyle I = \int_{0}^{+\infty} x \tilde{f}(x) dx$, 	déterminer une nouvelle suite convergeant presque sûrement vers $I$. En déduire une méthode (que l'on détaillera) permettant d'approcher  $I$. }
		\reponse{ Soit $J_n = \frac{1}{n} \sum_{i=1}^n Y_i$ où $(Y_i)$ est une suite de variables aléatoires i.i.d. selon une loi admettant $\tilde{f}$ pour densité. Alors on remarque que $I = \mathbb{E}(Y_1)$ et d'après la loi forte des grands nombres, 
			$$ \displaystyle \frac{1}{n}\sum_{i=1}^n Y_i \xrightarrow[n \to \infty]{{\rm{p.s.}}} I$$
			
			Sachant simuler la loi de $Y_1$, on déduit l'algorithme suivant permettant de déterminer une valeur approchée de $I$ :
			
			{\centering \fbox{\BUseVerbatim{Su3Xpython}}\par}
			
		}
	\end{enumerate}
}