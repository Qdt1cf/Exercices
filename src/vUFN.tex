\uuid{vUFN}
\chapitre{Probabilité discrète}
\sousChapitre{Variable aléatoire discrète}
\titre{Loi d'un couple}
\theme{variables aléatoires discrètes, loi conjointe}
\auteur{}
\datecreate{2024-09-09}
\organisation{AMSCC}	

\contenu{

Soit $(\Omega, \mathcal{A}, \prob)$ un espace probabilisé. On considère deux variables aléatoires discrètes $X$ et $Y$ telles que $X(\Omega) = \{1,2,3\}$, $Y(\Omega) = \{1,2,3,4\}$ et $\prob(X = x, Y = y)$ et : 
$$\forall i \in \{1,2,3\}, \quad \prob(X=i, Y=i) = \prob(X=i, Y=i+1) = \frac{1}{6}.$$

\begin{enumerate}
	\item \question{ Déterminer la loi du couple de variables aléatoires $(X,Y)$ sous forme d'un tableau.  } 
	\reponse{
		\begin{center}
			\begin{tabular}{|c|c|c|c|c|c|}
				\hline
				$X/Y$ & 1 & 2 & 3 & 4 & $\prob(X = x)$ \\
				\hline
				1 & $\frac{1}{6}$ & $\frac{1}{6}$ & 0 & 0 & $\frac{1}{3}$ \\
				\hline
				2 & 0 & $\frac{1}{6}$ & $\frac{1}{6}$ & 0 & $\frac{1}{3}$ \\
				\hline
				3 & 0 & 0 & $\frac{1}{6}$ & $\frac{1}{6}$ & $\frac{1}{3}$ \\
				\hline
				$\prob(Y = y)$ & $\frac{1}{6}$ & $\frac{1}{3}$ & $\frac{1}{3}$ & $\frac{1}{6}$ & 1 \\
				\hline
			\end{tabular}
		\end{center}
	}
	\item \question{ Déterminer les lois marginales du couple $(X,Y)$ puis calculer $\mathbb{E}(X)$ et $\mathbb{E}(Y)$. }
	\reponse{
		\begin{itemize}
			\item $\prob(X = 1) = \frac{1}{3}$, $\prob(X = 2) = \frac{1}{3}$, $\prob(X = 3) = \frac{1}{3}$.
			\item $\prob(Y = 1) = \frac{1}{6}$, $\prob(Y = 2) = \frac{1}{3}$, $\prob(Y = 3) = \frac{1}{3}$, $\prob(Y = 4) = \frac{1}{6}$.
			\item $\mathbb{E}(X) = \sum\limits_{i=1}^{3} i \times \prob(X = i) = 1 \times \frac{1}{3} + 2 \times \frac{1}{3} + 3 \times \frac{1}{3} = 2$.
			\item $\mathbb{E}(Y) = \sum\limits_{i=1}^{4} i \times \prob(Y = i) = 1 \times \frac{1}{6} + 2 \times \frac{1}{3} + 3 \times \frac{1}{3} + 4 \times \frac{1}{6} = 2.5$.
		\end{itemize}
	}
	\item \question{ Les variables aléatoires $X$ et $Y$ sont-elles indépendantes ? }
	\reponse{
		Non, car $\prob(X = 1, Y = 1) = \frac{1}{6} \neq \prob(X = 1) \times \prob(Y = 1) = \frac{1}{3} \times \frac{1}{6} = \frac{1}{18}$.
	}
	\item \question{ On pose $Z = X - Y$. Déterminer la loi du couple $(X,Z)$. Les variables aléatoires $X$ et $Z$ sont-elles indépendantes ? }
	\reponse{
L'ensemble des valeurs prises par $Z$ est $\{-3, -2, -1, 0, 1, 2\}$. On a $\prob(X = 1, Z = -2) = \prob(X = 1, Y = 3) = 0$. De même, on obtient le tableau de la loi du couple $(X,Z)$ : 

\begin{center}
	\begin{tabular}{|c|c|c|c|c|c|c|c|}
		\hline
		$X/Z$ & -3 & -2 & -1 & 0 & 1 & 2 & $\prob(X = x)$ \\
		\hline
		1 & 0 & 0 & $\frac{1}{6}$ & $\frac{1}{6}$ & $0$ & $0$ & $\frac{1}{3}$ \\
		\hline
		2 & 0 & 0 & $\frac{1}{6}$ & $\frac{1}{6}$ & $0$ & $0$ & $\frac{1}{3}$ \\
		\hline
		13 & 0 & 0 & $\frac{1}{6}$ & $\frac{1}{6}$ & $0$ & $0$ & $\frac{1}{3}$ \\
		\hline
		$\prob(Z = z)$ & $0$ & $0$ & $\frac{1}{2}$ & $\frac{1}{2}$ & $0$ & $0$ & 1 \\
		\hline
	\end{tabular}
\end{center}
On en déduit que les variables aléatoires $X$ et $Z$ sont indépendantes (toutes les colonnes sont proportionnelles entre elles).
	}
%	\item \question{ Déterminer $\mathbb{E}(Z)$. }
%	\reponse{
%		$\mathbb{E}(Z) = \sum\limits_{i=-3}^{2} i \times \prob(Z = i) = -1 \times \frac{1}{2} + 0 \times \frac{1}{2} = -\frac{1}{2}$.
%	}
\end{enumerate}
}