\uuid{aUn2}
\titre{Contrôle qualité}
\theme{statistiques, tests d'hypothèses}
\auteur{}
\organisation{AMSCC}
\contenu{

\texte{ 	Désirant juger le travail d'un ouvrier ajusteur, un chef d'atelier prélève un échantillon de 50 pièces métalliques dans sa production. On note $X$ l'épaisseur de ses pièces. L'objectif est d'avoir $\EX = 5~mm$. Les résultats des mesures sur cet échantillon sont portés dans le tableau suivant :
	\begin{center}	
	\begin{tabular}{|c|c|c|c|c|}
		\hline $n_i$ & 5 & 15 & 20 & 10 \\ 
		\hline $x_i$ en mm & 4.8 & 4.9 & 5.0 & 5.1 \\ 
		\hline 
	\end{tabular} 
	\end{center} }

\begin{enumerate}
	\item \question{ 	Cette vérification permet-elle de conclure que le résultat est conforme aux exigences, au seuil de confiance de $99\%$ ? }
	\reponse{
		\begin{align*}
			\bar{x} & = \frac{1}{50} \sum_{i=1}^4 n_i x_i \\
			& = \frac{1}{50} (5 \times 4.8 + 15 \times 4.9 + 20 \times 5.0 + 10 \times 5.1) \\
			& = 4.98 \\
		\end{align*}
		donc une estimation sans biais de l'épaisseur moyenne des pièces est $\bar{x} = 4.98~mm$. 

		De plus, la variance observée dans cet échantillon est : 
		\begin{align*}
			\sigma^2 & = \frac{1}{50} \sum_{i=1}^4 n_i (x_i - \bar{x})^2 \\
			& = \frac{1}{50} (5 \times (4.8 - 4.98)^2 + 15 \times (4.9 - 4.98)^2 + 20 \times (5.0 - 4.98)^2 + 10 \times (5.1 - 4.98)^2) \\
			& = 0.081 \\
		\end{align*}
		donc une estimation sans biais de la variance de l'épaisseur des pièces est $s^2 = \frac{50}{49} \sigma^2 = 0.083$.

		On réalise le test d'hypothèse suivant :
		\begin{align*}
			H_0 &: \EX = 5 \\
			H_1 &: \EX \neq 5
		\end{align*}
		avec un risque de première espèce de $1\%$. 

		La variable de décision est $Z = \frac{\bar{X} - 5}{\frac{S}{\sqrt{n}}} \sim \mathcal{N}(0,1)$.	
		
		On fait un test bilatéral, donc on rejette $H_0$ si $|Z| > z_{\frac{\alpha}{2}} = 2.58$ par lecture de la table de la loi normale. Or la valeur observée est $Z_{obs} = \frac{4.98 - 5}{\frac{\sqrt{0.083}}{\sqrt{50}}} = -2{,}33$. Donc on ne rejette pas $H_0$. On peut donc conclure que le résultat est conforme aux exigences, au seuil de confiance de $99\%$.
		}
		\item \question{ Quel risque de première espèce devrait-on prendre pour que la prise de décision soit différente ? }
		\reponse{
Pour que la prise de décision soit différente, il faudrait que la valeur critique soit $2.33$, ce qui correspond, par lecture de table, à un risque de première espèce de $1.98\%$. 
		}
\end{enumerate}
}
