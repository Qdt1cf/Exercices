\titre{}
\theme{}
\auteur{Q. Liard}
\organisation{AMSCC}
\contenu{

\texte{

On considère l'anneau $(\mathbb{Z}/21\mathbb{Z}, +, \times)$.
\begin{enumerate}
    \item Dans cette première partie, on considère le groupe additif $(\mathbb{Z}/21\mathbb{Z}, +)$.
    \begin{enumerate}
        \item Déterminer $\langle 3 \rangle$, le sous-groupe engendré par 3, puis déterminer $\omega(3)$, l'ordre de 3.
        \item A-t-on $\langle 3\rangle = \langle 7 \rangle$ ? Déterminer le groupe engendré par $3$ et $7:\,\,\langle 3,7 \rangle$. Est-ce un groupe cyclique ?
        \item Donner un générateur de $(\mathbb{Z}/21\mathbb{Z}, +)$.
      %  \item Déterminer $\langle 3 \rangle$, le sous-groupe engendré par 3, puis déterminer $\omega(3)$, l'ordre de 3.
    \end{enumerate}
    \item Dans la suite, on étudie l'ensemble $((\mathbb{Z}/21\mathbb{Z})^*, \times)$.
    \begin{enumerate}
        \item Quel est le cardinal de $(\mathbb{Z}/21\mathbb{Z})^*$ ? Cet ensemble est-il un groupe ?
        \\
        Dans la suite, on note $U := U((\mathbb{Z}/21\mathbb{Z})^*, \times)$, le groupe multiplicatif composé des inverses d'éléments de $(\mathbb{Z}/21\mathbb{Z}, \times)$ pour la loi $\times$.
        \item Montrer que $20 \in U$. En déduire $\langle 20 \rangle$, le sous-groupe de $U$ engendré par $20$.
         \item Justifier l'existence d'un élément d'ordre $6$ et d'un élément d'ordre $3$. Donner un exemple d'élément d'ordre $6$ et en déduire un autre d'ordre $3$.
    \end{enumerate}
\end{enumerate}












}
}