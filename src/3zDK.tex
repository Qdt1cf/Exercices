\uuid{3zDK}
\titre{Application de la loi de Benford}
\theme{Statistiques et tests d’hypothèses}
\auteur{Maxime NGUYEN}
\datecreate{10/12/2024}
\organisation{AMSCC}

\contenu{

\texte{
La loi de Benford, initialement appelée loi des nombres anormaux par Benford, fait référence à une fréquence de distribution statistique observée empiriquement sur de nombreuses sources de données dans la vraie vie, ainsi qu’en mathématiques. Dans une série de données numériques, on pourrait s’attendre à voir les chiffres de 1 à 9 apparaître à peu près aussi fréquemment comme premier chiffre significatif, soit avec une fréquence de $1/9 = 11,1\%$ pour chacun. Or, contrairement à cette intuition (biais d’équiprobabilité), la série suit très souvent approximativement la loi de Benford :
\def\data{%
  {1}{30.10}
  {2}{17.61}
  {3}{12.49}
  {4}{9.69}
  {5}{7.92}
  {6}{6.69}
  {7}{5.80}
  {8}{5.12}
  {9}{4.58}%
}

\begin{tikzpicture}[font=\small]

% Dimensions axes :
% On a 9 barres, on peut disposer les barres à x=1,2,3,...9
% Les axes iront donc de x=0.5 à x=9.5 environ.
% L'axe Y ira de 0% à 35%, soit 7 cm de haut.

% Dessin de l'axe Y
\draw[->,thick] (0.5,0) -- (0.5,7.5) node[above] {Fréquence (\%)};

% Graduations sur l'axe Y : de 0 à 35% par pas de 5%
\foreach \v in {0,5,10,15,20,25,30,35} {
   \draw (0.5,\v*0.2) -- (0.45,\v*0.2) node[left]{\v};
}

% Dessin de l'axe X
\draw[->,thick] (0.5,0) -- (9.5,0) node[right]{1ère décimale};

% Ajout du titre principal
\node[font=\bfseries,anchor=south] at (5,7.7) {Loi de BENFORD};
% Sous-titre
\node[font=\small,anchor=south] at (5,7.4) {Fréquences relatives d'apparition de la 1ère décimale};

% Graduation et labels sur l'axe X
\foreach \x in {1,2,3,4,5,6,7,8,9} {
  \draw (\x,0) -- (\x,-0.1) node[below]{\x};
}

% Dessin des barres
% Pour chaque paire (X, Y%), on dessine un rectangle
\foreach \X/\Y in \data {
  % Hauteur du rectangle = \Y% * 0.2 cm
  \pgfmathsetlengthmacro{\h}{\Y*\yscale}
  % Le rectangle va de (X - barwidth/2, 0) à (X + barwidth/2, h)
  \filldraw[fill=cyan!70!black,draw=black] (\X-\barwidth/2,0) rectangle (\X+\barwidth/2,\h);
  % Etiquette de pourcentage au-dessus de la barre
  \node[font=\footnotesize,anchor=south] at (\X,\h) {\Y\%};
}

\end{tikzpicture}
Cette loi est utilisée notamment dans la détection des fraudes. 

En 2019, le mathématicien Mickäel Launay a relevé 1226 prix dans un supermarché, et a obtenu comme fréquences successives pour les premiers chiffres de 1 à 9 : 32\%, 26\%, 15\%, 9\%, 5\%, 4\%, 3\%, 2\%, 4\%.
}


\begin{enumerate}
    \item \question{Avec un risque de première espèce de 5\%, peut-on affirmer que l’observation du mathématicien est incompatible avec la loi de Benford ?}
    \indication{Effectuer un test du $\chi^2$ pour comparer les fréquences observées avec celles attendues selon la loi de Benford.}
    \reponse{À compléter.}
\end{enumerate}

}
