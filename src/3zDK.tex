\uuid{3zDK}
\titre{Application de la loi de Benford}
\theme{Statistiques et tests d’hypothèses}
\auteur{Maxime NGUYEN}
\datecreate{10/12/2024}
\organisation{AMSCC}

\contenu{

\texte{
La loi de Benford, initialement appelée loi des nombres anormaux par Benford, fait référence à une fréquence de distribution statistique observée empiriquement sur de nombreuses sources de données dans la vraie vie, ainsi qu’en mathématiques. Dans une série de données numériques, on pourrait s’attendre à voir les chiffres de 1 à 9 apparaître à peu près aussi fréquemment comme premier chiffre significatif, soit avec une fréquence de $1/9 = 11,1\%$ pour chacun. Or, contrairement à cette intuition (biais d’équiprobabilité), la série suit très souvent approximativement la loi de Benford :
\begin{tikzpicture}

    \begin{axis}[
        ybar,
        width=14cm,
        height=8cm,
        bar width=0.5cm,
        enlarge x limits=0.1,
        ymin=0,
        ymax=35,
        ylabel={Fréquence (\%)},
        xlabel={1ère décimale},
        ytick={0,5,10,15,20,25,30,35},
        xtick={1,2,3,4,5,6,7,8,9},
        xticklabel style={font=\small},
        yticklabel style={font=\small},
        title={Loi de BENFORD},
        title style={yshift=0.5cm,font=\bfseries},
        axis lines=left,
        axis on top=true,
        yminorticks=false
    ]
    
    \addplot[fill=cyan!70!black,draw=black] coordinates {
        (1,30.10)
        (2,17.61)
        (3,12.49)
        (4,9.69)
        (5,7.92)
        (6,6.69)
        (7,5.80)
        (8,5.12)
        (9,4.58)
    };
    
    % Ajout du sous-titre
    \node[font=\small,align=center,anchor=south] at (rel axis cs:0.5,1.04) {Fréquences relatives d'apparition de la 1ère décimale};
    
    % Ajouter les étiquettes de pourcentage sur chaque barre
    \foreach \X/\Y in {1/30.10,2/17.61,3/12.49,4/9.69,5/7.92,6/6.69,7/5.80,8/5.12,9/4.58}{
        \node[font=\footnotesize,anchor=south] at (axis cs:\X,\Y) {\Y\%};
    }
    
    \end{axis}
    
    \end{tikzpicture}
Cette loi est utilisée notamment dans la détection des fraudes. 

En 2019, le mathématicien Mickäel Launay a relevé 1226 prix dans un supermarché, et a obtenu comme fréquences successives pour les premiers chiffres de 1 à 9 : 32\%, 26\%, 15\%, 9\%, 5\%, 4\%, 3\%, 2\%, 4\%.
}


\begin{enumerate}
    \item \question{Avec un risque de première espèce de 5\%, peut-on affirmer que l’observation du mathématicien est incompatible avec la loi de Benford ?}
    \indication{Effectuer un test du $\chi^2$ pour comparer les fréquences observées avec celles attendues selon la loi de Benford.}
    \reponse{À compléter.}
\end{enumerate}

}
