\uuid{51mL}
\chapitre{Probabilité continue}
\niveau{L2}
\module{Probabilité et statistique}
\sousChapitre{Densité de probabilité}
\titre{Calcul de probabilité}
\theme{variables aléatoires à densité}
\auteur{Maxime Nguyen}
\datecreate{2023-09-13}

\organisation{AMSCC}

\contenu{
    \texte{ 	Soit une fonction $f \colon x \mapsto \begin{cases}
        2x & \text{ si } x \in [0;1] \\
        0 & \text{ sinon }
    \end{cases}$. }

    \begin{enumerate}
        \item \question{ Montrer que $f$ est une densité de probabilité. }
        \reponse{
            \begin{enumerate}
                \item $f$ est positive sur $\R$.
                \item $\int_{-\infty}^{+\infty} f(x) \mathrm{d}x = \int_0^1 2x \mathrm{d}x = \left[ x^2 \right]_0^1 = 1$.
            \end{enumerate}
            Donc $f$ est une densité de probabilité.
        }
        \item \question{ \'Ecrire $f$ à l'aide d'une fonction indicatrice $x \mapsto \textbf{1}_A(x)$ où $A$ est un ensemble à préciser. }
        \reponse{
            $f(x) = 2x \textbf{1}_{[0;1]}(x)$.
        }
        \item \question{  Soit $X$ une variable aléatoire admettant $f$ pour densité de probabilité. Calculer $\prob(X \leq \frac{1}{2})$ et $\prob(-1 \leq X \leq 0.2)$. }
        \reponse{ 
            \begin{align*}
                \prob(X \leq \frac{1}{2}) &= \int_{-\infty}^{\frac{1}{2}} f(x) \mathrm{d}x \\
                &= \int_0^{\frac{1}{2}} 2x \mathrm{d}x \\
                &= \left[ x^2 \right]_0^{\frac{1}{2}} \\
                &= \frac{1}{4}
            \end{align*}
            \begin{align*}
                \prob(-1 \leq X \leq 0.2) &= \int_{-1}^{0.2} f(x) \mathrm{d}x \\
                &= \int_0^{0.2} 2x \mathrm{d}x \\
                &= \left[ x^2 \right]_0^{0.2} \\
                &= 0.04
            \end{align*}
        }
    \end{enumerate}
}