<<<<<<< Updated upstream
\titre{Cryptographie}
=======
\titre{Statistique}
>>>>>>> Stashed changes
\theme{AM}
\auteur{Q. Liard}
\organisation{AMSCC}
\contenu{
\texte{
<<<<<<< Updated upstream
On considère le crypto-système sans clef suivant. Un grand nombre premier $p$ est public et les unités de message sont des entiers $m$, $1 \leq m < p$.
\begin{enumerate}
\item
Si Alice veut envoyer un message $m$ à Bob, elle procède comme suit (on suppose que les transmissions s’effectuent sans erreurs) :
\begin{enumerate}
    \item[(i)] Alice choisit un entier $a$ tel que $1 \leq a < p$ et $pgcd(a, p - 1) = 1$. Elle calcule l’inverse $a'$ de $a$ dans $(\mathbb{Z}/(p-1)\mathbb{Z})$ et envoie
  $  c \equiv m^a \mod p$    à Bob.
    
    \item[(ii)] Bob choisit un entier $b$ tel que $1 \leq b < p$, et $pgcd(b, p - 1) = 1$. Il calcule l’inverse $b'$ de $b$ dans $(\mathbb{Z}/(p-1)\mathbb{Z})$ et renvoie
  $  d \equiv c^b \mod p$  à Alice.
        \item[(iii)] Alice envoie $e \equiv d^{a'} \mod p$  à Bob.
    
    \item[(iv)] Bob calcule $ m \equiv e^{b'} \mod p$ et retrouve $m$. Pourquoi ?
\end{enumerate}
\item 
Soit $p = 31$.
\begin{enumerate}

    \item Quels sont les ordres multiplicatifs possibles des éléments de $\big((\mathbb{Z}/31\mathbb{Z})^{\ast}, \times \big)$ ? Donner les ordres multiplicatifs de 2 et 4.
        \item Soit $\mathcal{A} = \{x \in \{1, 2, \ldots, 30\} \, | \, \text{pgcd}(x, 30) = 1\}$. Calculer $card(\mathcal{A})$, puis énumérer tous ses éléments ainsi que leur inverse modulo 30.
   

    \item Trouver $b \in \mathcal{A}$, $b \neq 1$ tel que $ 4^b \equiv 4 \mod 31.$

     \item On utilise le crypto-système du 1), avec $p = 31$. Un pirate intercepte les échanges entre Alice et Bob et connaît $c = 4$, $d = 4$ et $e = 8$. Montrer qu’il peut facilement retrouver $m$, dont on donnera la valeur.


 

\end{enumerate}
\end{enumerate}
=======



L'état-major d’une armée souhaite optimiser l’entretien de ses \textbf{véhicules blindés} en fonction de leur \textbf{durée d’opération en environnement hostile}. Pour cela, on cherche à modéliser la relation entre :

\begin{itemize}
    \item \( X \) : \textbf{Nombre de jours en mission dans une zone de conflit} .
    \item \( Y \) : \textbf{Coût moyen des réparations après mission} (en milliers d’euros) .
\end{itemize}

Les valeurs suivantes ont été observées sur un échantillon de véhicules après différentes missions :

\begin{center}
    \begin{tabular}{c c}
        \toprule
        \textbf{Jours en mission (\( X \))} & \textbf{Coût de réparation (\( Y \)) (k€)} \\
        \midrule
        5  & 12.1 \\
        10 & 20.5 \\
        15 & 34.2 \\
        20 & 55.8 \\
        8  & 16.3 \\
        12 & 27.4 \\
        18 & 48.9 \\
        25 & 88.3 \\
        30 & 120.7 \\
        7  & 14.8 \\
        14 & 30.1 \\
        22 & 70.5 \\
        28 & 110.2 \\
        \bottomrule
    \end{tabular}
\end{center}

\begin{enumerate}
    \item \textbf{Déterminer la nature des variables étudiées} (\( X \) et \( Y \)) et justifier leur classification.
    \item \textbf{Calculer les caractéristiques statistiques suivantes} pour \( X \) et \( Y \) :
    \begin{itemize}
        \item La moyenne \( \bar{X} \) et \( \bar{Y} \).
        \item L’écart-type \( \sigma_X \) et \( \sigma_Y \).
        \item La médiane de \( X \) et \( Y \).
    \end{itemize}
\end{enumerate}

\begin{enumerate}
    \setcounter{enumi}{2}
    \item \textbf{Tracer le nuage de points} \( (X, Y) \) et observer la tendance de la relation entre les deux variables.
\end{enumerate}

\begin{enumerate}
    \setcounter{enumi}{3}
    \item \textbf{Effectuer un changement de variable logarithmique} :
    \[
    Z = \log(Y)
    \]
    Montrer que cette transformation linéarise la relation entre \( X \) et \( Z \).
    
    \item \textbf{Déterminer l’équation de la droite de régression} des moindres carrés entre \( X \) et \( Z \) :
    \[
    Z = aX + b
    \]
    puis en déduire l’équation de la relation entre \( X \) et \( Y \) sous forme exponentielle :
    \[
    Y = e^{aX + b}
    \]

    \item \textbf{Calculer le coefficient de corrélation linéaire} \( r \) entre \( X \) et \( Z \), et interpréter son sens physique.
    
    \item \textbf{Prédire le coût moyen des réparations} pour un véhicule ayant effectué \textbf{23 jours} en mission.

   
    \end{itemize}
\end{enumerate}






>>>>>>> Stashed changes
}
}