\uuid{zAlF}
\chapitre{Polynôme, fraction rationnelle}
\sousChapitre{Fraction rationnelle}
\titre{Elements simples}
\theme{fractions rationnelles}
\auteur{}
\datecreate{2023-01-24}
\organisation{AMSCC}
\contenu{

\question{ Pour chacune des fractions rationnelles, indiquer s'il s'agit d'un élément simple dans $\mathbb{R}(X)$.

\begin{center}
	\begin{tabular}{|c|c|c||c|c|c|}
	\hline Fraction rationnelle & Oui & Non & Fraction rationnelle & Oui & Non \\
	\hline$F_1(X)=\frac{1}{(X+1)}$ & & & $F_4(X)=\frac{X-1}{\left(X^2+1\right)}$ & & \\
	\hline$F_2(X)=\frac{1}{\left(X^2+1\right)}$ & & & $F_5(X)=\frac{X-1}{(X+1)^2}$ & &  \\
	\hline$F_3(X)=\frac{1}{(X+1)^3}$ & & & $F_6(X)=\frac{X^2+X+1}{\left(X^3+1\right)}$ & &  \\
	\hline
\end{tabular}
\end{center} }

\reponse{ $F_1(X)=\frac{1}{(X+1)}$ est un élément simple car le dénominateur $(X+1)$ est un polynôme de degré 1 donc irréductible sur $\mathbb{R}$ et le numérateur est une constante, donc de degré $0(<1)$.
$F_2(X)=\frac{1}{\left(X^2+1\right)}$ est un élément simple car le dénominateur $\left(X^2+1\right)$ est un polynôme de degré 2 irréductible sur $\mathbb{R}$ et le numérateur est une constante, donc de degré $0(<2)$.

$F_3(X)=\frac{1}{(X+1)^3}$ est un élément simple car le dénominateur $(X+1)^3$ est constitué d'un polynôme de degré 1 irréductible sur $\mathbb{R}$, élevé à la puissance 3 , et le numérateur est une constante, donc de degré $0(<1)$.
$F_4(X)=\frac{X-1}{\left(X^2+1\right)}$ est un élément simple car le dénominateur $\left(X^2+1\right)$ est un polynôme de degré 2 irréductible sur $\mathbb{R}$, et le numérateur est un polynôme de degré $0(<2)$.
$F_5(X)=\frac{X-1}{(X+1)^2}$ n'est pas un élément simple car le dénominateur $(X+1)^2$ est constitué d'un polynôme de degré 1 irréductible sur $\mathbb{R}$, élevé à la puissance 2, et le numérateur est un polynôme lui aussi de degré 1 .
$F_6(X)=\frac{X^2+X+1}{\left(X^3+1\right)}$ n'est pas un élément simple car le dénominateur $\left(X^3+1\right)$ n'est pas un polynôme irréductible sur $\mathbb{R}$, car c'est un polynôme de degré 3 . }}
