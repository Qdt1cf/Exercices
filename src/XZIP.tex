\chapitre{Probabilité continue}
\sousChapitre{Densité de probabilité}
\uuid{XZIP}
\titre{File d'attente}
\theme{variables aléatoires à densité, probabilités conditionnelles}
\auteur{}
\datecreate{2023-09-13}
\organisation{AMSCC}
%
\contenu{

\indication{Dans les calculs, on pourra utiliser une intégration par parties:
	\[\int_a^b u(x)v'(x) \dx= {[uv]}_a^b-\int_a^b u'(x)v(x) \dx \]
	ou bien se souvenir que 
	\[\int_0^{+\infty} x^n e^{-x} \dx= n!=1\times 2\times \ldots \times n\text{.}\]}

\texte{	
Après enquête, on estime que le temps de passage à une caisse, exprimé en unités de temps, est une \va $T$ dont une densité de probabilité est donnée par la fonction $f_T$ définie par: 
$$ f_T(t)=\begin{cases} 0 & \text{ si } t<0  \\
	t e^{-t}& \text{ si } t\geq 0.
\end{cases}$$
}
\begin{enumerate}
	\item \begin{enumerate}
		\item \question{ Soit $X$ une variable aléatoire exponentielle de paramètre $\lambda=1$. Donner une densité $f_X$ de $X$, son espérance et sa variance. }
		\reponse{
			$X\sim \mathcal{E}(1)$, \hspace{2em} $f_X(t)=\begin{cases} e^{-t} & \text{ si } t\geq 0 \\
				0 & \text{ sinon}
			\end{cases}$, \hspace{2em}
			$\E(X)=\frac{1}{\lambda}$ \hspace{2em} et \hspace{2em} $\V(X)=\frac{1}{\lambda^2}$.
		}
		
		\item \question{ Vérifier que $f_T$ est bien une densité de probabilité. Déterminer $\E(T)$ et $\V(T)$. }
		
		\reponse{$f_T$ est une fonction positive sur $\R$ et
			\[ \int_\R f_T(x)\dx=\int_0^{+\infty}xe^{-x}\dx=\left[-xe^{-x} \right]_0^{+\infty} +\int_0^{+\infty} e^{-x}\dx = \left[-e^{-x}\right]_0^{+\infty}=1.\]
			Donc $f_T$ est bien une densité de probabilité. \\
			%\hspace{0.3em}
			
			L'espérance de $T$ peut se calculer soit par intégrations par parties, soit en utilisant le rappel:
			\[ \E(T)=\int_0^{+\infty} x^2e^{-x}\dx=\left[ -x^2e^{-x}\right]_0^{+\infty} +\int_0^{+\infty} 2xe^{-x}\dx=2\int_0^{+\infty}xe^{-x} \dx=2.\] \\
			
			De même pour la variance de $T$: $\V(T)=\E(T^2)-\E(T)^2$. Or
			\[ \E(T^2)=\int_0^{+\infty} x^3e^{-x}\dx=6\]
			donc $\V(T)=6-2^2=2$.
		}
		
	\end{enumerate}
	\item \begin{enumerate}
		\item \question{ Démontrer que la fonction de répartition de $T$, notée $F_T$, est définie par:
		\[
		F_T(x)=
		\begin{cases}
			0  \text{ si } x<0, \\
			1-(1+x)e^{-x} \text{ si } x \geq 0.
		\end{cases}
		\] }
		\reponse{Par définition, 
			$F_T(x)=\int_{-\infty}^x f_T(t)\dx t$ donc si $x< 0$, $F_T(x)=0$ et si $x\geq 0$,
			\begin{align*}
				F_T(x) & =\int_0^x te^{-t}\dx t
				= \left[-te^{-t}\right]_0^x + \int_0^x e^{-t}\dx t
				= -xe^{-x}+0+\left[-e^{-t}\right]_0^x 
				=-xe^{-x}-e^{x}+e^0 \\
				&= 1-(1+x)e^{-x},
			\end{align*}
			ce qui correspond à la formule donnée dans l'énoncé.
		}
		
		
		
		\item \question{ Montrer que la probabilité que le temps de passage en caisse soit inférieur à deux unités de temps sachant qu'il est supérieur à une unité, est égale à $\frac{2e-3}{2e}$. }
		\reponse{
			\begin{align*}
				\prob(T\leq 2|T\geq 1)
				&=\frac{\prob(\{T\leq 2\}\cap \{T\geq 1\})}{\prob(T\geq 1)}
				=\frac{\prob(1\leq T\leq 2)}{1-\prob(T\leq 1)}
				= \frac{F_T(2)-F_T(1)}{1-F_T(1)} \\
				&=\frac{1-3e^{-2}-1+2e^{-1}}{1-1+2e^{-1}}
				=\frac{2e^{-1}-3e^{-2}}{2e^{-1}}
				=\frac{2e-3}{2e}
			\end{align*}
		}
		
		
	\end{enumerate}
	\item \texte{ Un jour donné, trois clients $A$, $B$ et $C$ se présentent simultanément devant deux caisses libres. Par courtoisie, $C$ décide de laisser passer $A$ et $B$ et de prendre la place du premier d'entre eux qui aura terminé. On suppose que les variable aléatoires $T_A$ et $T_B$ correspondant aux temps de passage en caisse de $A$ et de $B$ sont indépendantes. }
	\begin{enumerate}
		\item \question{ Soit $M$ la variable aléatoire égale au temps d'attente du client $C$. Exprimer l'événement $\{M>x\}$ en fonction des événements $\{T_A>x\}$ et $\{T_B>x\}$. }
		\reponse{
			$\{M>x\}=\{T_A>x\}\cap\{T_B>x\}$
		}
		
		\item \question{ En déduire une expression de la fonction de répartition $F_M$ de $M$, en fonction de $F_{T_A}$ et $F_{T_B}$. }
		\reponse{Pour tout $t\in \R$,
			\begin{align*}
				F_M(t)
				&=\prob(M\leq t) \\
				&=1-\prob(M>t) \\
				&=1-\prob(\{T_A>t\}\cap\{T_B>t\}) \\
				&=1-\prob(T_A>t)\prob(T_B>t) \quad \text{ par indépendance de $T_A$ et de $T_B$} \\
				&=1-(1-\prob(T_A\leq t))(1-\prob(T_B\leq t)) \\
				&= 1- (1-F_{T_A}(t))(1-F_{T_B}(t))
			\end{align*}
		}
		
		
		\item \question{ Déterminer explicitement une densité de $M$. }
		\reponse{
			Comme $T_A$ et $T_B$ suivent la même loi que $T$, on $F_{T_A}=F_{T_B}=F_T$. On obtient donc :
			\begin{equation*}
				F_M (x)
				\,=\, 1-{(1-F_T(x))}^2 \,=\, 
				\begin{cases}
					1-(1+x)^2e^{-2x} & \text{ si } x\geq 0 \\
					0 \text{ si } x<0.
				\end{cases} 
			\end{equation*}
			Afin d'obtenir la densité de la variable aléatoire $M$, on dérive sa fonction de répartition en tous les points de continuité de sa densité :
			\begin{align*}
				f_M(t)&= \begin{cases}
					-2(1+x)e^{-2x}+2(1+x)^2e^{-2x} & \text{ si } x\geq 0 \\
					0 \text{ si } x<0
				\end{cases} \\
				&= \begin{cases}
					2x\, (1+x)e^{-2x} & \text{ si } x\geq 0 \\
					0 \text{ si } x<0
				\end{cases}
			\end{align*}
		}
		
	\end{enumerate}
\end{enumerate}
}