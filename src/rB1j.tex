\uuid{rB1j}
\chapitre{Déterminant, système linéaire}
\niveau{L2}
\module{Algèbre}
\sousChapitre{Système linéaire, rang}
\titre{Résolution de système linéaire}
\theme{systèmes linéaires}
\auteur{}
\datecreate{2023-01-16}
\organisation{AMSCC}
\contenu{

\question{ Résoudre le système :
$$
\left\{\begin{aligned}
2 x+4 y+3 z-t & =5 \\
x+2 y+4 z-3 t & =10 \\
-x+4 y+3 z-2 t & =1 \\
3 x+y-z+2 t & =0
\end{aligned}\right.
$$ }

\reponse{ 
	On résout par opérations sur les lignes afin de faire apparaître un système triangulaire (principe du pivot de Gauss)et achever la résolution par substitution :
$$
\begin{aligned}
& \left\{\begin{aligned}
2 x+4 y+3 z-t & =5 \\
x+2 y+4 z-3 t & =10 \\
-x+4 y+3 z-2 t & =1 \\
3 x+y-z+2 t & =0
\end{aligned}\right. \\
& \begin{array}{l}
\ell_1 \leftarrow \ell_1 \\
\ell_2 \leftarrow \ell_3+\ell_4 \\
\ell_3 \leftarrow-\frac{1}{5} \ell_2 \\
\ell_4 \leftarrow \ell_4
\end{array}\left\{\begin{aligned}
x+2 y+4 z-3 t & =10 \\
y-6 z+6 t & =-19 \\
z-t & =3 \\
-5 y-13 z+11 t & =-30
\end{aligned}\right. \\
& \begin{array}{l}
\ell_1 \leftarrow \ell_1 \\
\ell_2 \leftarrow \ell_2 \\
\ell_3 \leftarrow \ell_3 \\
\ell_4 \leftarrow \ell_4+5 \ell_2
\end{array} \quad\left\{\begin{aligned}
x+2 y+4 z-3 t & =10 \\
y-6 z+6 t & =-19 \\
z-t & =3 \\
-43 z+41 t & =-125
\end{aligned}\right. \\
& \begin{array}{l}
\ell_1 \leftarrow \ell_1 \\
\ell_2 \leftarrow \ell_2 \\
\ell_3 \leftarrow \ell_3 \\
\ell_4 \leftarrow \ell_4+43 \ell_3
\end{array}\left\{\begin{aligned}
x+2 y+4 z-3 t & =10 \\
y-6 z+6 t & =-19 \\
z-t & =3 \\
-2 t & =4
\end{aligned}\right. \\
&
\end{aligned}
$$

$$\mathcal{S}  = \{(2,-1,1,-2)\}$$ }
}
