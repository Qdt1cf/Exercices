\titre{Vrai ou Faux}
\theme{fonctions de plusieurs variables}
\auteur{}
\organisation{AMSCC}


\texte{ Soit $f$ une fonction définie sur un ensemble $\mathcal{D}$ de $\mathbb{R}^2$, à valeurs dans $\mathcal{R}$. Soit $a\in \mathcal{D}$.  }
\question{ 	Dire si les affirmations suivantes sont vraies ou fausses.  On ne demande pas de justifier mais une mauvaise réponse entraîne une pénalité de points. }
	\begin{enumerate}
		\item \question{ Si $f$ admet des dérivées partielles en $a$, alors $f$ est différentiable en a. }
		\reponse{ FAUX }
		\item \question{ Si $f$ admet des dérivée partielles en $a$, alors $f$ est continue en $a$. }
		\reponse{ VRAI }
		\item\question{  Si $f$ est différentiable en tout point de $\mathcal{D}$, alors $f$ est de classe $C^1$ sur $\mathcal{D}$. }
		\reponse{ FAUX }
		\item \question{ Si $Df(a)=0$, alors $f$ admet un extremum local en $a$. }
		\reponse{ FAUX }
	\end{enumerate}
