\titre{Autour de la méthode d'Euler implicite}
\theme{analyse numérique}
\auteur{}
\organisation{AMSCC}
\contenu{

%50p166 Analyse Numérique Canon 

\texte{ Soit $f \colon [0;T] \times \mathbb{R} \to \mathbb{R}$, $a \in \mathbb{R}$ et l'équation différentielle :
$$(E)	\begin{cases} 
y'(t) = f(t,y(t)) \\
y(0) = a
\end{cases} $$

pour laquelle on admet l'existence et l'unicité d'une solution $y$ de classe $\mathcal{C}^2$. On suppose de plus qu'il existe $M>0$ tel que $\forall t \in [0;T]$, $\forall y \in \mathbb{R}$ : 
$$ \frac{\partial^2 f}{\partial y^2}(t,y)>0 \qquad ; \qquad \left| \frac{\partial f}{\partial y}(t,y)   \right| \leq M$$

On rappelle que la méthode d'Euler implicite est donnée par le schéma 
$$y_{n+1} = y_n + hf(t_{n+1},y_{n+1})$$
et on suppose que le pas vérifie $h \leq \frac{1}{2M}$.

Pour tout $n \geq 0$, on pose $\varphi_n(x) = y_n+hf(t_{n+1},x)-x$ et $H_n(x) = y_n+hf(t_{n+1},x)$. }

\begin{enumerate}
	\item \question{ Vérifier que $H_n$ est une application contractante et en déduire que le schéma est bien défini, c'est-à-dire qu'il permet bien de définir explicitement $y_{n+1}$ en fonction de $y_n$. }
	\reponse{$|H_n'(x)| = h \left| \frac{\partial f}{\partial y}(t_{n+1},x)  \right| \leq hM \leq \frac{1}{2} < 1$. Si $n$ est fixé et $y_n$ est défini, alors $H_n$ admet donc un unique point fixe que l'on note $y_{n+1}$. }
	\item \question{ On propose la méthode suivante :
	$$(S) : \begin{cases}
	\widehat{y}_{n+1}=y_n - \frac{\varphi_n(y_n)}{\varphi_n'(y_n)} \\
	y_{n+1} = y_n + hf(t_{n+1},\widehat{y}_{n+1})
	\end{cases}$$
	On admet que cette méthode est stable. Expliquer pourquoi cette méthode ainsi décrite permet de définir explicitement $y_{n+1}$ en fonction de $y_n$. Décrire en particulier la méthode utilisée pour définir $\widehat{y}_{n+1}$. Puis montrer que cette méthode $(S)$ est consistante, donc convergente. }
	\reponse{Ce schéma permet de résoudre l'écriture implicite en utilisant la méthode de Newton. On écrit le schéma sous la forme standard 
		$$y_{n+1}=y_n+hF(t_n,y_n,h)$$ avec 
		$$F(t,y,h)= f\left(t+h,y+h\frac{f(t+h,y)}{1-h\frac{\partial f}{\partial y}(t+h,y)}  \right)$$
		et on vérifie que $F(t,y,0)=f(t,y)$ : la méthode est consistante au moins d'ordre 1. On pourrait vérifier que la méthode n'est pas d'ordre 2 en calculant $\frac{\partial F}{\partial h}$ et en constatant que $\frac{\partial F}{\partial h}(t,y,0) = f^{[1]}(t,y) \neq \frac{1}{2}f^{[1]}(t,y)$. La méthode étant supposée stable, elle est donc convergente.}
	\item \texte{ On suppose maintenant que l'équation différentielle est autonome : $f(t,y) = f(y)$ et que $\forall y \in \mathbb{R}$, $|f(y)| \leq M$, $|f'(y)| \leq M$ et $|f''(y)| \leq M$.  }
	\begin{enumerate}
		\item \question{ Simplifier le schéma $(S)$ en l'écrivant en fonction de $f$ et $f'$. }
		\reponse{Le schéma devient 
			$$(S') : \begin{cases}
			\widehat{y}_{n+1}=y_n - \frac{f(y_n)}{1-hf'(y_n)} \\
			y_{n+1} = y_n + hf(t_{n+1},\widehat{y}_{n+1})
			\end{cases}$$}
		\item \question{ Montrer que $\forall y,z \in \mathbb{R}$, 
		$$|f(y)(f'(z)-f(z)f'(y)| \leq 2M^2|y-z|$$ }
		\reponse{\begin{align*}
				|f(y)(f'(z)-f(z)f'(y)| &= |f(y)(f'(z)-f'(y))+f'(y)(f(y)-f(z)| \\
				&\leq |f(y)||f'(z)-f'(y)|+|f'(y)||f'(y)-f(z)|\\
				&\leq M\left(|f'(z)-f'(y)|+|f'(y)-f(z)|\right)\\
			\end{align*}
			Or $|f'(y)| \leq M$ et $|f''(y)| \leq M$ donc $f$ et $f'$ sont $M$-lipschitziennes, d'où le résultat.}
		\item \question{ Montrer que $\forall y \in \mathbb{R}$ :
		$$\left|\frac{1}{1-hf'(y)} \right| \leq 2$$ }
		\reponse{On a $hf'(y) \leq Mh \leq \frac{1}{2}$ d'où $1-hf'(y) \geq \frac{1}{2}$ d'où le résultat demandé.}
		\item \question{ On pose $g_R(y,h)=\frac{f(y)}{1-hf'(y)}$. Montrer que $\forall y,z \in \mathbb{R}$ :
		$$|g_R(y,h)-g_R(z,h)| \leq 4 |f(y)f'(z)-f(y)-f(z)f'(y)+f(z)|$$
		puis 
		$$|g_R(y,h)-g_R(z,h)| \leq 4M(1+2Mh)|y-z|$$ }
		\reponse{\begin{align*}
				|g_R(y,h)-g_R(z,h)| &= \left| \frac{f(y)}{1-hf'(y)}-\frac{f(z)}{1-hf'(z)}  \right| \\
				&= \frac{|f(y)-f(z)+h(f(z)f'(y)-f(y)f'(z) |}{|1-hf'(y)||1-hf'(z)|}\\
				&\leq 4|f(y)-f(z)|+4h|f(z)f'(y)-f(y)f'(z)| \\
				&\leq 4M|y-z| + 8M^2h|y-z|\\
				&\leq 4M(1+2Mh)|y-z|
		\end{align*} }
		\item \question{ En déduire que la méthode est stable }
		\reponse{On étudie la stabilité du nouveau schéma $(S')$ : il s'écrit $y_{n+1}=y_n+hF(y_n,h)$ avec $F(y,h)=f(y+hg_R(y,h))$. Or $f$ est $M$-lipschitzienne donc 
			\begin{align*} F(y,h)-F(z,h) &\leq M(|y-z|+h|g_R(y,h)-g_R(z,h)|)\\
				&\leq M(1+2Mh(1+2Mh))|y-z|\\
				&\leq 5M|y-z|
			\end{align*}
			Donc $F$ est lipschitzienne uniformément en $h$ : cela prouve la stabilité du schéma. }
		\item \question{ En déduire que la méthode est convergente. }
		\reponse { La consistance ayant déjà établie dans le cas général, cela prouve que la méthode est convergente.}
	\end{enumerate}
\end{enumerate}}
