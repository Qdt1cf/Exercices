\uuid{M00y}
\chapitre{Probabilité continue}
\niveau{L2}
\module{Probabilité et statistique}
\sousChapitre{Densité de probabilité}
\titre{Propriété des fonctions de répartition}
\theme{variables aléatoires}
\auteur{}
\datecreate{2022-10-07}
\organisation{AMSCC}
\difficulte{3}
\contenu{

\texte{ Soit la fonction $F$ définie pour tout $x$ réel par 
$$F(x) = \begin{cases}
ae^x & \text{ si } x <0 \\
-\frac{1}{2}e^{-x} + b & \text{ si } x \geq 0
\end{cases}$$ }

\begin{enumerate}
	\item \question{ Déterminer des conditions sur les réels $a$ et $b$ de sorte qu'il existe une variable aléatoire réelle $X$ telle que $F$ soit la fonction de répartition de $X$.  }
	\item \question{ \`A quelles conditions cette fonction $F$ définit-elle la fonction de répartition d'une variable aléatoire à densité ? }
\end{enumerate}
 
 \reponse{	Il est nécessaire que $b=1$ pour que $\lim\limits_{+\infty} F = 1$.
 	
 	Pour que $F$ soit croissante, il est nécessaire que $a \geq 0$ et $F(0^-) \leq F(0)$ soit $a \leq b-\frac{1}{2} = \frac{1}{2}$. 
 	
 	Dans tous les cas, $F$ est bien continue à droite.
 	
 	Réciproquement, $F$  vérifie toutes les conditions suffisantes pour que $F$ soit une fonction de répartition.
 	
 	Si on ajoute la condition que $a=\frac{1}{2}$ alors $F$ est continue sur $\R$ et c'est la fonction de répartition d'une variable absolument continue. 
}}
