\titre{Produit matriciel}
\theme{calcul matriciel}
\auteur{}
\organisation{AMSCC}
\contenu{

\texte{ Soient les matrices $A=\left(\begin{array}{ll}0 & 0 \\ 0 & 1\end{array}\right)$ et $B=\left(\begin{array}{ll}0 & 0 \\ 1 & 0\end{array}\right)$. }

\begin{enumerate}
	\item \question{ A-t-on $A^2-B^2=(A+B)(A-B)$ ? Justifier. }
	\item \question{ Déterminer une condition nécessaire et suffisante pour obtenir l'égalité. }
\end{enumerate}
 
\reponse{ $$
\begin{aligned}
	& A^2=\left(\begin{array}{ll}
		0 & 0 \\
		0 & 1
	\end{array}\right) \cdot\left(\begin{array}{ll}
		0 & 0 \\
		0 & 1
	\end{array}\right)=\left(\begin{array}{ll}
		0 & 0 \\
		0 & 1
	\end{array}\right) \\
	& B^2=\left(\begin{array}{ll}
		0 & 0 \\
		1 & 0
	\end{array}\right) \cdot\left(\begin{array}{ll}
		0 & 0 \\
		1 & 0
	\end{array}\right)=\left(\begin{array}{ll}
		0 & 0 \\
		0 & 0
	\end{array}\right) \\
	& A^2-B^2=\left(\begin{array}{ll}
		0 & 0 \\
		0 & 1
	\end{array}\right) \\
	& A+B=\left(\begin{array}{ll}
		0 & 0 \\
		1 & 1
	\end{array}\right) \\
	& A-B=\left(\begin{array}{ll}
		0 & 0 \\
		-1 & 1
	\end{array}\right) \\
	& (A+B) \cdot(A-B)=\left(\begin{array}{ll}
		0 & 0 \\
		1 & 1
	\end{array}\right) \cdot\left(\begin{array}{cc}
		0 & 0 \\
		-1 & 1
	\end{array}\right)=\left(\begin{array}{cc}
		0 & 0 \\
		-1 & 1
	\end{array}\right)
\end{aligned}
$$
Ainsi :
$$
A^2-B^2 \neq(A+B)(A-B)
$$
Justification :
$$
(A+B)(A-B)=A^2 \underbrace{-AB+BA}_{\neq 0}-B^2
$$
car $BA \neq A B$.
Pour avoir l'égalité, il faut et il suffit que $A$ et $B$ commutent : $B A=A B$. }}
