\uuid{FaOL}
\chapitre{Matrice}
\sousChapitre{Propriétés élémentaires, généralités}
\titre{Produit matriciel}
\theme{calcul matriciel}
\auteur{}
\datecreate{2022-12-15}
\organisation{AMSCC}

\contenu{
	
	\texte{ Soient les matrices $A=\begin{pmatrix}0 & 0 \\ 0 & 1\end{pmatrix}$ et $B=\begin{pmatrix}0 & 0 \\ 1 & 0\end{pmatrix}$. }
	
	\begin{enumerate}
		\item \question{ A-t-on $A^2-B^2=(A+B)(A-B)$ ? Justifier. }
		\item \question{ Déterminer une condition nécessaire et suffisante pour obtenir l'égalité. }
	\end{enumerate}
	
	\reponse{ $$
		\begin{aligned}
		& A^2=\begin{pmatrix}
		0 & 0 \\
		0 & 1
		\end{pmatrix} \cdot \begin{pmatrix}
		0 & 0 \\
		0 & 1
		\end{pmatrix}=\begin{pmatrix}
		0 & 0 \\
		0 & 1
		\end{pmatrix} \\
		& B^2=\begin{pmatrix}
		0 & 0 \\
		1 & 0
		\end{pmatrix} \cdot \begin{pmatrix}
		0 & 0 \\
		1 & 0
		\end{pmatrix}=\begin{pmatrix}
		0 & 0 \\
		0 & 0
		\end{pmatrix} \\
		& A^2-B^2=\begin{pmatrix}
		0 & 0 \\
		0 & 1
		\end{pmatrix} \\
		& A+B=\begin{pmatrix}
		0 & 0 \\
		1 & 1
		\end{pmatrix} \\
		& A-B=\begin{pmatrix}
		0 & 0 \\
		-1 & 1
		\end{pmatrix} \\
		& (A+B) \cdot(A-B)=\begin{pmatrix}
		0 & 0 \\
		1 & 1
		\end{pmatrix} \cdot \begin{pmatrix}
		0 & 0 \\
		-1 & 1
		\end{pmatrix}=\begin{pmatrix}
		0 & 0 \\
		-1 & 1
		\end{pmatrix}
		\end{aligned}
		$$
		Ainsi :
		$$
		A^2-B^2 \neq(A+B)(A-B)
		$$
		Justification :
		$$
		(A+B)(A-B)=A^2 \underbrace{-AB+BA}_{\neq 0}-B^2
		$$
		car $BA \neq A B$.
		Pour avoir l'égalité, il faut et il suffit que $A$ et $B$ commutent : $B A=A B$. }}

