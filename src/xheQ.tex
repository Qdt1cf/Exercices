\uuid{xheQ}
\chapitre{Statistique}
\sousChapitre{Tests d'hypothèses, intervalle de confiance}
\titre{ Lois pour les statistiques}
\theme{loi normale, loi du chi2, loi de Student}
\auteur{Maxime Nguyen}
\datecreate{2023-11-20}
\organisation{AMSCC}
\contenu{

	On définit trois variables aléatoires indépendantes $(X_1,X_2,X_3)$ suivant chacune une loi normale d'espérance $\mu=10$ et de variance $\sigma^2=4$.
	
	
	
	On pose 
	$$\overline{X} = \frac{1}{3}\sum_{i=1}^{3} X_i \qquad {T} = \frac27 X_1+\frac37 X_2+\frac27 X_3$$
	$$U = \frac{1}{4}\sum_{i=1}^{3}  {(X_i-10)^2} \qquad V = \frac{1}{4}\sum_{i=1}^{3}  {(X_i-\overline{X})^2}$$
	
	\begin{enumerate}
		\item\question{  Déterminer, en justifiant par un calcul, la loi de la variable aléatoire $T$. }
		\reponse{ Par linéarité, on calcule $\mathbb{E}(T) = \frac{2+3+2}{7} \times 10  = 10$. Par indépendance et propriété de la variance, $\sigma^2(\overline{T}) = \frac{4+9+4}{49} \times  4= \frac{68}{49}$. Par somme de lois normales, $\overline{X}$ suit une loi normale $\mathcal{N}(10,\sigma^2 = \frac{68}{49})$. }
%		\item \question{ Vérifier que $\overline{X}$ et $T$ sont deux estimateurs sans biais de $\mu$. Lequel de ces deux estimateurs de $\mu$ est le plus efficace ? }
%		\reponse{ Pour étudier le biais d'un estimateur, on doit calculer leur espérance. Par linéarité, $\mathbb{E}(\overline{X}) = \mu$ donc le biais de $\overline{X}$ est $B(\overline{X}) = \mathbb{E}(\overline{X}-\mu) = 0$. De même, $B(T)=0$.
%		 }
		\item \question{ Vérifier que $\overline{X}$ et $T$ sont deux estimateurs sans biais de $\mu$. Lequel de ces deux estimateurs de $\mu$ est le plus efficace ? }
\reponse{ Pour étudier le biais d'un estimateur, on doit calculer leur espérance. Par linéarité, $\mathbb{E}(\overline{X}) = \mu$ donc le biais de $\overline{X}$ est $B(\overline{X}) = \mathbb{E}(\overline{X}-\mu) = 0$. De même, $B(T)=0$.
	
		Pour comparer l'efficacité de ces deux estimateurs, on peut comparer leurs variances respectives : par indépendance et propriété de la variance, $\sigma^2(\overline{X}) = \frac{3 \times 4}{9} = \frac{4}{3} < \frac{68}{49} = \sigma^2(T)$. Par conséquent, $\overline{X}$ est plus efficace que $T$ pour estimer $\mu$. }
		\item \question{ Déterminer, en justifiant, la loi de la variable aléatoire $U$ et la loi de la variable $V$. }
		\reponse{ $U = \frac{1}{4}\sum_{i=1}^{3}  {(X_i-10)^2} = \sum_{i=1}^{3}  {  \left( \frac{ X_i-10}{2}\right)^2}$. Or les variables aléatoires $X_i$ sont indépendantes et $ \frac{ X_i-10}{2}$ suit une loi normale centrée réduite donc par définition $U$ suit une loi $\chi^2(3)$.
	
	Par théorème du cours (Théorème de Fisher), $V$ suit une loi $\chi^2(2)$. }
		\item \question{ A l'aide des tables de valeurs, déterminer un réel $t$ tel que $\PP(V>t) = 0.95$. }
		\reponse{ Par lecture de la table d'une loi $\chi^2(2)$, on a $\PP(V \leq 0.1026) = 0.05$ donc on peut prendre $t=0{,}1026$. }
%		\item \question{ On pose $$Y = \frac{\overline{X}-10}{\sqrt{\frac{2V}{3}}}$$
%		Déterminer la loi de $Y$. }
%	\reponse{  on a $$Y = \frac{\overline{X}-10}{\sqrt{\frac{2V}{3}}} = \frac{\overline{X}-10}{\sqrt{\frac{4}{3} \frac{V}{2}}} = 
%		\frac{ \frac{ \overline{X}-10}{\sqrt{\frac43}}}{\sqrt{\frac{V}{2}}}$$ et on reconnaît une loi de Student $St(2)$. }
\item \question{  On pose $$Y = \frac{\overline{X}-10}{\sqrt{\frac{2V}{3}}}$$ 
Parmi les formules suivantes, laquelle permet de déterminer le réel $t$ tel que $\prob(Y > t) = 0.025$ ?
\begin{itemize}
	\item \texttt{=LOI.NORMALE.STANDARD.INVERSE(0.95)}
	\item \texttt{=LOI.KHIDEUX.INVERSE(0,975;2)}
	\item \texttt{=1-LOI.KHIDEUX.INVERSE(0,025;3)}
	\item \texttt{=LOI.STUDENT.INVERSE.N(0,975;2)}
	\item \texttt{=1-LOI.STUDENT.INVERSE.N(0,025;3)}
\end{itemize}
}
\reponse{  on a $$Y = \frac{\overline{X}-10}{\sqrt{\frac{2V}{3}}} = \frac{\overline{X}-10}{\sqrt{\frac{4}{3} \frac{V}{2}}} = 
			\frac{ \frac{ \overline{X}-10}{\sqrt{\frac43}}}{\sqrt{\frac{V}{2}}}$$ et on reconnaît une loi de Student $St(2)$. On cherche $t$ tel que $\prob(Y > t) = 0.025 \iff \prob(Y \leq t) = 0.975$ d'où la formule : {=LOI.STUDENT.INVERSE.N(0,975;2)}  }
	\end{enumerate}

}