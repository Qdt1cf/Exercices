\uuid{EOnD}
\titre{Test de conformité d'un simulateur }
\theme{statistiques}
\auteur{}
\organisation{AMSCC}

\contenu{
	Le 26 novembre 2023, le sous-lieutenant Mounir Offski a programmé un générateur pseudo-aléatoire de nombres distribués selon une loi géométrique. On observe 100 entiers naturels produits par ce générateur aléatoire (voir le fichier de données). 
	
	On rappelle que si $X$ suit une loi géométrique de paramètre $p$, alors pour tout entier $k \geq 1$, $\prob(X=k)=p(1-p)^{k-1}$ et $\mathbb{E}(X) = \frac{1}{p}$.
	
	\begin{enumerate}
		\item Expliquer le contenu des cellules \texttt{D2:D6}.
		\item Donner une estimation du paramètre $p$ que le sous-lieutenant a utilisé pour simuler la loi géométrique, en précisant l'estimateur utilisé.
		\item Avec un risque de première espèce de $1\%$, au vu de l'échantillon fourni, peut-on affirmer que ce simulateur de loi géométrique fonctionne correctement ? Avec un risque de première espèce de $5\%$ ? de $10\%$ ?
		%\item Mettre côte à côte la représentation graphique de la distribution de l'échantillon et celle d'une loi géométrique dont on précisera comment on a choisi le paramètre (choix de l'estimateur). 
	\end{enumerate}
}