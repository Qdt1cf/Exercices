\uuid{DEZs}
\chapitre{Statistique}
\niveau{L2}
\module{Probabilité et statistique}
\sousChapitre{Tests d'hypothèses, intervalle de confiance}
\titre{ Estimateur et intervalle de confiance }
\theme{statistiques, estimateurs, intervalle de confiance}
\auteur{}
\datecreate{2022-09-24}
\organisation{AMSCC}
\difficulte{}
\contenu{

\texte{ Soient $X_1,...,X_n$ des variables aléatoires indépendantes et de même loi ayant pour densité :
	$$f_\theta \colon x \mapsto \left\{
	\begin{array}{ll}
		\frac{1}{2}(1+\theta x) & \mbox{si } -1 \leq x \leq 1 \\
		0 & \mbox{sinon.}
	\end{array}
	\right.$$
	où $\theta \in [-1;1]$ est un paramètre.  }

\begin{enumerate}
	\item \question{ Montrer que pour tout $\theta \in [-1;1]$, $f_\theta $ est une densité de probabilité. }
	\reponse{Le fait que $\theta \in [-1;1]$ garantit que $f_{\theta}(x) \geq 0$ pour tout $x \in [-1;1]$. De plus, 
		\begin{align*}
			\int f_\theta(x) dx &= \int_{-1}^{1} \frac{1}{2}(1+\theta x) dx \\
			&= \int_{-1}^{1} \frac{1}{2}(1+0) dx \quad \text{ par imparité de x} \\
			&= 1
		\end{align*}	
		Donc $f_{\theta}$ définit une bien une densité de probabilité.
	}
	\item \question{ Calculer $\mathbb{E}(X_1)$ et $\mathbb{V}(X_1)$. En déduire l'espérance et la variance de la variable aléatoire $\overline{X} = \frac{1}{n}\sum_{i=1}^n X_i$.  }
	\reponse{On calcule les moments d'ordre 1 et 2 de la variable à densité $X_1$ :
		\begin{align*}
			\mathbb{E}(X_1) &= \int x f_\theta(x) dx \\
			&= \int_{-1}^{1} \frac{1}{2}(x+\theta x^2) dx \\
			&= \int_{-1}^{1} \frac{1}{2}(\theta x^2) dx \quad \text{ par imparité de $x$} \\
			&= 2\int_{0}^{1} \frac{1}{2}(\theta x^2) dx \quad \text{ par parité de $x^2$} \\
			&= \frac{\theta}{3}
		\end{align*}
		De même, 		
		\begin{align*}
			\mathbb{E}(X_1^2) &= \int x^2 f_\theta(x) dx \\
			&= \int_{-1}^{1} \frac{1}{2}(x^2+\theta x^3) dx \\
			&= \int_{-1}^{1} \frac{1}{2}(x^2) dx \quad \text{ par imparité de $x^3$} \\
			&= \int_{0}^{1}  x^2 dx \quad \text{ par parité de $x^2$} \\
			&= \frac{1}{3}
		\end{align*}
		
		Avec la formule de Koenig Huygens, on en déduit que 
		$$\mathbb{V}(X_1) = \mathbb{E}(X_1^2) - 	\mathbb{E}(X_1)^2 = \frac{1}{3}-\frac{\theta^2}{9} = \frac{3-\theta^2}{9}$$
	}
	\item \question{ On pose $T_n = \frac{3}{n} \sum_{i=1}^{n} X_i$. Montrer que $T$ est un estimateur sans biais et convergent de $\theta$. }
	\reponse{On calcule par linéarité de l'espérance : $\mathbb{E}(T_n) = \frac{3}{n} \times n \mathbb{E}(X_1) = \theta$. Donc le biais de $T_n$ est 
		$B(T_n) = \mathbb{E}(T_n-\theta) =\theta - \theta = 0$.
		
		De plus, par propriétés de la variance et indépendance, 
		$$\mathbb{V}(T_n) =  \frac{9}{n^2} \times n \times \mathbb{V}(X_1) = \frac{3-\theta^2}{n}$$
		Or $EQM(T_n) = \mathbb{V}(T_n) + B(T_n)^2$ donc $EQM(T_n) = \frac{3-\theta^2}{n} \xrightarrow[n \to +\infty]{} 0$ : cela prouve que l'estimateur $T_n$ est convergent.
	}		
	
	\item \question{ \`A l'aide du Théorème Central Limite, donner une approximation de la loi de $T_n$. }
	\reponse{On pose $\mu = \mathbb{E}(X_1)$ et $\sigma = \sqrt{\mathbb{V}(X_1)}$. Les variables $(X_i)$ sont iid, admettent une espérance et une variance donc d'après le théorème central limite, la variable 
		$$\frac{\sum_{i=1}^n X_i - n\mu}{\sigma\sqrt{n} }$$ converge en loi vers une loi normale $\mathcal{N}(0,1)$.
		En réécrivant, cela revient à dire que 	
		$$\frac{\frac{3}{n}\sum_{i=1}^n X_i - 3\mu}{3\frac{\sigma}{\sqrt{n}} } = \frac{T_n-\theta}{\sqrt{\frac{3-\theta^2}{n}}}$$ converge en loi vers une loi normale $\mathcal{N}(0,1)$.
		
		Si $n$ est grand ($n \geq 30$), cela revient à dire que $T_n$ suit approximativement une loi normale $\mathcal{N}(\theta, \sigma^2 = \frac{3-\theta^2}{n})$.  
	}
	\item \question{ Démontrer qu'il existe une constante $M_n$ ne dépendant pas de $\theta$ telle que si $\lambda >0$, 
		$$\PP(|T_n-\theta| < \lambda) \geq 1-\frac{M_n}{\lambda^2}$$ }
	\reponse{	D'après l'inégalité de Bienaymé Tchebichev, 
		$$\PP(|T_n-\mathbb{E}(T_n)| \geq  \lambda) \leq \frac{\mathbb{V}(T_n)}{\lambda^2}$$
		d'où $$\PP(|T_n-\theta| \geq \lambda) \leq \frac{3-\theta^2}{n\lambda^2} \leq \frac{3}{n\lambda^2} = \frac{M_n}{\lambda^2}$$
		en posant $M_n = \frac{3}{n}$.  Par passage au complémentaire, on obtient finalement :
		$$\PP(|T_n-\theta| < \lambda) \geq 1-\frac{3}{n\lambda^2}$$
	}
	\item \question{ Déterminer un intervalle de  confiance permettant d'estimer $\theta$ avec une confiance d'au moins $95\%$. }
	\reponse{On cherche un intervalle $I$ tel que $\PP(\theta \in I) \geq 0{,}95$. Or d'après ce qui précède,
		
		\begin{align*}
			\PP(|T_n-\theta| < \lambda) \geq 1-\frac{3}{n\lambda^2} &\iff \PP(-\lambda < T_n-\theta < \lambda ) \geq 1-\frac{3}{n\lambda^2} \\
			&\iff	\PP( \theta \in ]T_n-\lambda ; T_n + \lambda [) \geq 1-\frac{3}{n\lambda^2} 	
		\end{align*} 
		Or 	$1-\frac{3}{n\lambda^2} = 0{,}95 \iff \frac{3}{n\lambda^2} = 0{,}05 \iff \lambda^2 = \frac{3}{0{,}05 n}$
		donc 
		$$\PP\left( \theta \in \left]T_n - \sqrt{\frac{3}{0{,}05 n}} ; T_n + \sqrt{\frac{3}{0{,}05 n}}  \right[\right) \geq 0{,}95$$
		d'où l'intervalle de confiance $I = \left]T_n - \sqrt{\frac{3}{0{,}05 n}} ; T_n + \sqrt{\frac{3}{0{,}05 n}}  \right[$.
	}
	%	\item 
	%	Déterminer une constante $M$ ne dépendant pas de $\theta$ telle que $\mathbb{V}(T) \leq M$. En appliquant l'inégalité de Bienaymé-Tchebichev à la variable $T$, en déduire un intervalle de  confiance permettant d'estimer $\theta$ avec une confiance d'au moins $95\%$.
	%\item En déduire la construction d'un intervalle de confiance permettant d'estimer $\theta$ avec une confiance de $95\%$.
	%	\item On considère un échantillon $(X_1,...,X_n)$ de taille $n=100$ suivant la loi à densité $f_\theta$ avec $\theta=1$. Donner une valeur approchée de la probabilité $\PP(T>1{,}2)$.
\end{enumerate}
}
