\titre{Calcul par approximation}
\theme{probabilités}
\auteur{Maxime Nguyen}
\organisation{AMSCC}

\contenu{

\texte{ Un fournisseur d'accès internet met en place un point local d'accès qui dessert $\nombre{5000}$ abonnés. À un instant donné, chaque abonné a une probabilité égale à $0.20$ d'être connecté. Les comportements des abonnés sont supposés indépendants les uns des autres. }
\begin{enumerate}
	\item \question{ On note $X$ la variable aléatoire égale au nombre d'abonnés connectés à un instant $t$. Quelle est la loi de $X$ ? Quelle est son espérance ? Son écart-type ? }
	\reponse{ 
		$X\sim \mathcal{B}(\nombre{5000},0.2)$, $\E(X)=\nombre{1000}$ et $\sigma^2(X)=800$.
	}
	
	
	\item \question{ On pose $Y=\frac{X-\nombre{1000}}{\sqrt{800}}$. Justifier précisément que l'on peut approcher la loi de $Y$ par la loi normale centrée réduite. }
	\reponse{ 
		$X$ peut être approchée par une loi $\mathcal{N}(\nombre{1000},\sigma=\sqrt{800})$ donc en centrant et en réduisant, on obtient que $Y\sim \mathcal{N}(0,1)$.
	}
	
	\item \question{ Le fournisseur d'accès souhaite savoir combien de connexions simultanées le point d'accès doit pouvoir gérer pour que sa probabilité d'être saturé à un instant donné soit inférieure à $2.5$\%. En utilisant l'approximation précédente, proposer une valeur approchée de ce nombre de connexions. }
	\reponse{ 
		Soit $n$ le nombre de connexions simultanées au point d'accès. On cherche $n$ tel que $\prob(X\geq n)\leq 0.025$, c'est-à-dire
		\[ \prob\left(Y\geq \frac{n-\nombre{1000}}{\sqrt{800}}\right) \leq 0.025 ,\]
		autrement dit
		\[  \prob\left(Y\leq \frac{n-\nombre{1000}}{\sqrt{800}}\right) \geq 0.975.\]
		Par lecture de table de loi, on obtient $\displaystyle\frac{n-\nombre{1000}}{\sqrt{800}}\simeq 1.96$, soit $n\simeq 1055.44$.
		On en conclut qu'il faut qu'au minimum le point d'accès puisse gérer $1056$ connexions simultanées pour que la probabilité d'être saturé soit inférieure à $2.5$\%.
	}
	
\end{enumerate}
}