\titre{Nombres de Carmichael}
\theme{AM}
\auteur{Q. Liard}
\organisation{AMSCC}
\contenu{
\texte{
Nous allons vérifier que $561 = 3 \times 11 \times 17
$
est un entier non premier vérifiant que pour tout entier \( a \) premier avec 561 :
$
a^{561} \equiv a \pmod{561}.
$

\begin{enumerate}
    \item Calculer \( 561 \) modulo 2, modulo 10 et modulo 16.

    \item En déduire que, pour tout entier \( a \) premier avec 561 :
 $$
    \begin{cases} 
    a^{561} \equiv a \pmod{3}, \\
    a^{561} \equiv a \pmod{11}, \\
    a^{561} \equiv a \pmod{17}.
    \end{cases}
$$

    \item Montrer que si \( pgcd(a, 561) = 1 \), alors :
 $$
    a^{560} \equiv 1 \pmod{561}.
 $$

    \item Énoncer le petit théorème de Fermat. Ce théorème caractérise-t-il les nombres premiers ?
\end{enumerate}
}

}