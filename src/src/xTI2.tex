\titre{Densité de probabilité, question de cours}
\theme{probabilités}
\auteur{}
\organisation{AMSCC}
\contenu{



\texte{ Soit la fonction $f=\frac{1}{3}\textbf{1}_{[-1;2]}$. }

	\begin{enumerate}
		\item \question{ Vérifier que $f$	 définit une densité de probabilité.  }
		\reponse{ On vérifie que $f$ est une fonction positive et son intégrale vaut $\int_{-\infty}^{+\infty}f(x)dx = \int_{-1}^{2} \frac{1}{3}dx = 1$.  }
		\item Soit $X$ une variable aléatoire ayant pour densité $f$. Calculer $\PP(1 \leq X \leq 3)$.
		\reponse{ $\PP(1 \leq X \leq 3) = \int_1^3 \frac{1}{3} \textbf{1}_{[-1;2]} dx = \frac{1}{3} \int_1^2 dx = \frac{1}{3}$. }
	\end{enumerate}


}
