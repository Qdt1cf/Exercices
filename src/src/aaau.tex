\titre{Théorème Chinois}
\theme{AM}
\auteur{Q. Liard}
\organisation{AMSCC}
\contenu{
\texte{
\begin{enumerate}
\item 
Soient \( m_1, m_2, m_3 \) des nombres deux à deux premiers entre eux. On considère le système :
\[
\begin{cases}
x \equiv x_1 \mod m_1 \\
x \equiv x_2 \mod m_2 \\
x \equiv x_3 \mod m_3
\end{cases}
\]

On pose \( m = m_1 \times m_2 \times m_3 \) et 
\[
m'_i = \frac{m}{m_i}, \quad \text{et} \quad z_i = (m'_i)^{-1} \mod m_i.
\]
Montrer que 
\[
x = x_1 m'_1 z_1 + x_2 m'_2 z_2 + x_3 m'_3 z_3 \mod m
\]
est solution du système.
\item  Résoudre les systèmes d’équations suivants :
\[
\begin{cases}
x \equiv 2 \mod 2 \\
x \equiv 3 \mod 3 \\
x \equiv 4 \mod 5
\end{cases}
\quad \quad
\begin{cases}
x \equiv 1 \mod 2 \\
x \equiv 2\mod 3\\
x \equiv 3\mod 4\\
x \equiv 4\mod 5\\
\end{cases}
\]
\end{enumerate}
}
}