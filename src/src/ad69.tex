\titre{Marche aléatoire}
\theme{probabilités}
\auteur{}
\organisation{AMSCC}
\contenu{

\texte{ On considère un carré ABCD et son centre de gravité O. On note $\mathcal{E} = \{A,B,C,D,O\}.$.  Une puce se déplace aléatoirement en sautant d’un point de $\mathcal{E}$ à un autre. La seule contrainte est que si un saut relie deux sommets du carré, ceux-ci doivent être adjacents. 

A chaque saut, tous les déplacements possibles sont équiprobables. La puce ne
reste pas deux fois de suite au même endroit.

Au départ (c'est-à-dire avant son premier saut) elle se trouve au point $A$. .

Pour tout $n \in \N$, et tout $K \in \mathcal{E}$, on note $K_n$ l’événement \og la puce se trouve au point $K$ après son $n$-ième saut. On notera $p_n=\PP(O_n)$ (de sorte que $p_0=0$). }

\begin{enumerate}
	\item \question{ Calculer $\PP(K_2)$ pour tout  $K \in \mathcal{E}$. }
	\item \question{ Démontrer que pour tout $n \geq 1$, $p_{n+1}=\frac{1}{3}(1-p_n)$ et en déduire une expression de $p_n$ en fonction de $n$. }
	\item \question{ En utilisant les symétries du problème, calculer $\PP_{B_2}(O_3)$. }
	\item \question{ Sachant que la puce est en $O$ après 3 sauts, quelle est la probabilité qu'elle soit passée par $B$ ? }
\end{enumerate}}
