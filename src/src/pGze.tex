\titre{Décomposition en éléments simples}
\theme{fractions rationnelles}
\auteur{}
\organisation{AMSCC}
\contenu{

\begin{enumerate}
\item \question{ Calculer la division euclidienne de $X^5+1$ par $X^4-2 X^3+X^2$. }
\reponse{ $$
X^5+1=(X+2) \cdot\left(X^4-2 X^3+X^2\right)+3 X^3-2 X^2+1
$$ }
\item \question{ Factoriser dans $\mathbb{R}[X]$ le polynôme $X^4-2 X^3+X^2$. }
\reponse{ $$
X^4-2 X^3+X^2=X^2 \cdot\left(X^2-2 X+1\right)=X^2 \cdot(X-1)^2
$$ }
\item \question{ On note $F(X)=\frac{X^5+1}{X^4-2 X^3+X^2}$. Donner la forme de la décomposition en éléments simples de $F(X)$ dans $\mathbb{R}(X)$. }
\reponse{ $$
F(X)=\frac{X^5+1}{X^4-2 X^3+X^2}=\frac{X^5+1}{X^2 \cdot(X-1)^2}=X+2+\frac{A}{X}+\frac{B}{X^2}+\frac{C}{(X-1)}+\frac{D}{(X-1)^2}
$$ }
\item \question{ Calculer $\lim\limits_{X \rightarrow 0}\left(X^2 \cdot F(X)\right)$ }
\reponse{ $$
\begin{aligned}
	\lim _{X \rightarrow 0}\left(X^2 \cdot F(X)\right) & =\lim _{X \rightarrow 0} \frac{X^2 \cdot\left(X^5+1\right)}{X^2 \cdot(X-1)^2}=1 \\
	& =\lim _{X \rightarrow 0} X^2 \cdot\left[X+2+\frac{A}{X}+\frac{B}{X^2}+\frac{C}{(X-1)}+\frac{D}{(X-1)^2}\right]=B
\end{aligned}
$$ }
\item \question{ Calculer $\lim\limits_{X \rightarrow 1}\left((X-1)^2 \cdot F(X)\right)$. }
\reponse{ $$
\begin{aligned}
	\lim _{X \rightarrow 1}\left((X-1)^2 \cdot F(X)\right) & =\lim _{X \rightarrow 1} \frac{(X A)^2 \cdot\left(X^5+1\right)}{X^2 \cdot(X-1)^2}=2 \\
	& =\lim _{X \rightarrow 1}(X-1)^2 \cdot\left[X+2+\frac{A}{X}+\frac{B}{X^2}+\frac{C}{(X-1)}+\frac{D}{(X-1)^2}\right]=D
\end{aligned}
$$ }

\item \question{ Calculer $F(-1)$ et $F(2)$ de deux façons différentes. }
\reponse{ $$
\begin{aligned}
	& F(-1)=\frac{(-1)^5+1}{(-1)^2 \cdot(-1-1)^2}=0 \\
	&=-1+2+\frac{A}{-1}+\frac{1}{(-1)^2}+\frac{C}{(-1-1)}+\frac{2}{(-1-1)^2}=2-A-\frac{C}{2}+\frac{1}{2} \\
	& \Rightarrow 2 A+C=5 \\
	& F(2)=\frac{(2)^5+1}{(2)^2 \cdot(2-1)^2}=\frac{33}{4} \\
	&=2+2+\frac{A}{2}+\frac{1}{(2)^2}+\frac{C}{(2-1)}+\frac{2}{(2-1)^2}=4+\frac{A}{2}+\frac{1}{4}+C+2 \\
	& \Rightarrow A+2 C=4 \\
	&\left\{\begin{array} { l } 
		{ 2 A + C = 5 } \\
		{ A + 2 C = 4 }
	\end{array} \Leftrightarrow \left\{\begin{array}{l}
		A=2 \\
		C=1
	\end{array}\right.\right.
\end{aligned}
$$ }
\item \question{ En déduire la décomposition en éléments simples de $F(X)$ dans $\mathbb{R}(X)$. }
\reponse{ $$
F(X)=\frac{X^5+1}{X^4-2 X^3+X^2}=\frac{X^5+1}{X^2 \cdot(X-1)^2}=X+2+\frac{2}{X}+\frac{1}{X^2}+\frac{1}{(X-1)}+\frac{2}{(X-1)^2} .
$$ }
\end{enumerate}}
