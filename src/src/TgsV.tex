\titre{Probabilités conditionnelles et indépendance}
\theme{probabilités}
\auteur{}
\organisation{AMSCC}
\contenu{


\texte{ 	Une entreprise fabrique des cartes graphiques. Deux ateliers de fabrication se répartissent la production d'une journée de la façon suivante : l'atelier A produit 900 cartes et l'atelier B produit 600 cartes. Les contrôles qualités ont montré qu'un jour donné, 2\% des cartes produite par l'atelier $A$ et 1\% des cartes produites par l'atelier $B$ sont défectueuses. 
	
	On prélève au hasard une carte produite dans la journée et on note les événements comme suit :
	
	\begin{itemize}
		\item $A$ : \og la carte provient de l'atelier A \fg{}
		\item $B$ : \og la carte provient de l'atelier B \fg{}
		\item $D$ : \og la carte est défectueuse \fg{}.
	\end{itemize} }
	
	\begin{enumerate}
		%	\item Construire un arbre pondéré décrivant la situation ;
		%	\item Calculer $\PP(D)$.
		\item \question{ Sachant qu'une carte prélevée est défectueuse, quelle est la probabilité qu'elle provienne de l'atelier A ? }
		\item \question{ Les événements $A$ et $D$ sont-ils indépendants ? }
	\end{enumerate}}
