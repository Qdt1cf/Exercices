\titre{Calcul de dérivées partielles}
\theme{calcul différentiel}
\auteur{}
\organisation{AMSCC}
\contenu{
\texte{


\paragraph{Exercise 3: }





\begin{center}
\begin{animateinline}[controls, loop]{1} % 1 image par seconde, contrôles visibles, boucle activée
    \multiframe{5}{i=1+1}{ % 5 étapes d'animation
        \begin{tikzpicture}[scale=1]
            \def\layersep{2cm}
            \tikzstyle{every pin edge}=[thick]
            \tikzstyle{neuron}=[circle,fill=black!25,minimum size=12pt,inner sep=0pt]
            \tikzstyle{entree}=[];
            \tikzstyle{input neuron}=[neuron, fill=green!50];
            \tikzstyle{output neuron}=[neuron, fill=red!50];
            \tikzstyle{hidden neuron}=[neuron, fill=blue!50];
            \tikzstyle{annot} = [text width=4em, text centered]
            
            % Entree
            \node[entree,blue] (E-1) at (-\layersep,-0.5) {$1$};
            \node[entree,blue] (E-2) at (-\layersep,-2.5) {$x$};
            \node[entree,blue] (E-3) at (-\layersep,-5) {$1$};
            
            \ifnum\i>1
                % Première couche
                \node[input neuron] (I-1) at (0,-1.5) {};
                \node[input neuron] (I-3) at (0,-3.75) {};
                
                \node[below right=0.8ex,scale=0.7] at (I-1) {$\Sigma$};
                \node[above right=0.8ex,scale=0.7] at (I-3) {$\Sigma$};
                
                % Arêtes de la première couche
                \path[thick] (E-1) edge node[pos=0.8,above,scale=0.7]{$w_{1}$} (I-1) ;
                \path[thick] (E-2) edge node[pos=0.8,above left,scale=0.7]{$w_{2}$} (I-1);
                \path[thick] (E-2) edge node[pos=0.8,above,scale=0.7]{$w_4$} (I-3);
                \path[thick] (E-3) edge node[pos=0.8,above,scale=0.7]{$w_{3}$} (I-3);
            \fi
            
            \ifnum\i>2
                % Deuxième couche (couches cachées)
                \node[hidden neuron] (O) at (\layersep,-3.75 cm) {};
                \node[hidden neuron] (P) at (\layersep,-1.5 cm) {};
                \node[below right=0.8ex,scale=0.7] at (O) {$h_{11}$};
                \node[above right=0.8ex,scale=0.7] at (P) {$h_{12}$};
                
                % Arêtes de la deuxième couche
                \path[thick] (I-1) edge node[pos=0.8,above,scale=0.7]{$f_{11}$} (P);
                \path[thick] (I-3) edge node[pos=0.8,below,scale=0.7]{$f_{12}$}(O);
            \fi
            
            \ifnum\i>3
                % Troisième couche (nouvelle couche cachée)













































                
                % Arêtes vers la nouvelle couche
                \path[thick] (P) edge node[pos=0.8,above,scale=0.7]{$w_{5}$} (N);
                \path[thick] (O) edge node[pos=0.8,below,scale=0.7]{$w_{6}$} (N);
            \fi
            
            \ifnum\i>4
                % Sortie
                \node[output neuron] (Q) at (3*\layersep,-2 cm) {};
                \path[thick] (N) edge node[pos=0.8,above,scale=0.7]{$w_{7}$} (Q);
                
                % Flèche de sortie
                \draw[->,thick] (Q)-- ++(2,0) node[right,blue]{$\hat{y}$};
            \fi
            
        \end{tikzpicture}
 
\end{animateinline}





\end{center}
}   


}