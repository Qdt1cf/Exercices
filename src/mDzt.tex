\uuid{mDzt}
\titre{Modélisation avec des estimateurs}
\theme{estimateurs, théorème central limite}
\auteur{}
\datecreate{2022-12-14}
\organisation{AMSCC}
\contenu{


\texte{ 	Lors d'une  élection, un  candidat a obtenu 65\% des voix. On considère deux échantillons de 
	votants.  Déterminer  la  probabilité  pour  que  deux  échantillons  de  200  votants  chacun 	indiquent  plus  de  10 points  de  différence  entre  les  fréquences  de  gens  qui  ont  voté  pour  ce 	candidat.  }	

\reponse{
	On note respectivement $F_1$ et $F_2$ la proportion de votants pour ce candidat dans chaque échantillon de taille 200. En notant $(X_1,...,X_n)$ le premier échantillon, on sait que $n=200$ et que chaque $X_i$ suit une loi de Bernoulli $\mathcal{B}(p)$ où $p = 0{,}65$. Ainsi, d'après le Théorème Central Limite, $F_1 = \frac{1}{n} \sum_{i=1}^n X_i$ suit approximativement une loi normale de moyenne $p$ et de variance $\frac{p(1-p)}{n}$. Il est clair que $F_2$ suit la même loi de probabilité que $F_1$. 
	
	Ainsi, on peut s'intéresser à la différence des résultats pour chaque échantillon $D=F_1-F_2$. Par somme de lois normales, cette variable $D$ suit une loi normale centrée (moyenne $p-p=0$) de variance $\frac{p(1-p)}{n}+\frac{p(1-p)}{n} = \frac{2p(1-p)}{n}$.
	
	Il reste donc à calculer $\prob(|D|>0.10) = 2 \times \prob(D>0.10) = \PP\left(\frac{D}{\sqrt{\frac{2p(1-p)}{n}}} > 2.1\right) = 2 \times 0.00179 = 0.036$
}}
