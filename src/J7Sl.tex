\titre{ \'Etude d'une loi absolument continue }
\theme { probabilités }
\auteur{ }
\organisation{AMSCC}

\contenu{
\texte{
    Soit $k \in \R$ et $f$ la fonction définie sur $\R$ pour tout $x \in \R$ par :
    $$ f(x) = \begin{cases}
        kx(1-x) & \text{ si } x \in [0,1] \\
        0 & \text{ sinon }
    \end{cases} $$
 }

 \begin{enumerate}
    \item \question{ Déterminer la valeur de $k$ pour que $f$ soit une densité de probabilité. }
    \item \question{ Soit $X$ une variable aléatoire réelle de densité $f$. Déterminer la fonction de répartition $F$ de $X$. }
    \item \question{ Déterminer la probabilité que $X$ prenne une valeur dans l'intervalle $[0{,}5\,;\,1]$. }
    \item \question{ Déterminer l'espérance et la variance de $X$. }
    \item \question{ Soit la variable aléatoire $Y = X^2$. En distinguant éventuellement selon les valeurs de $t \in \R$, déterminer $a(t)$ et $b(t)$ de telle sorte que $\{Y \leq t\} = \left\{a(t) \leq X \leq b(t)\right\}$. }
    \item \question{ En déduire une expression de la fonction de répartition de $Y$ puis une densité de probabilité de $Y$. }
 \end{enumerate}
}