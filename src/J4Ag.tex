\uuid{J4Ag}
\chapitre{Probabilité discrète}
\niveau{L2}
\module{Probabilité et statistique}
\sousChapitre{Lois de distributions}
\titre{Fonction caractéristique d'une loi de Poisson}
\theme{fonction caractéristique, loi de Poisson}
\auteur{}
\datecreate{2022-09-27}
\organisation{AMSCC}
\difficulte{}
\contenu{

	\begin{enumerate}
		\item \question{ Soit $X$ une variable aléatoire de loi de Poisson de paramètre $\lambda$. Calculer sa fonction caractéristique. }
		      \reponse{ Soit $t\in\mathbb{R}$.
		      	\begin{align*}
		      	\phi_X(t)=\mathbb{E}(e^{itX})
		      	=\sum_{k\geq 0} e^{itk}\p(X=k)
		      	=\sum_{k\geq 0} e^{itk}\frac{\lambda^k}{k!}e^{-\lambda}
		      	=e^{-\lambda}\sum_{k\geq 0} \frac{(\lambda e^{it})^k}{k!}
		      	=e^{\lambda(e^{it}-1)}
		      	\end{align*} }
		\item \question{ Soit $X$ et $Y$ deux variables aléatoires indépendantes de loi de Poisson de paramètres respectifs $\lambda$ et $\mu$. En utilisant la fonction caractéristique, montrer que la variable $X+Y$ suit une loi de Poisson de paramètre à déterminer. }
		      \reponse{ Comme les variables aléatoires $X$ et $Y$ sont indépendantes, on a pour tout $t\in\mathbb{R}$,
		      	\[ \phi_{X+Y}(t)=\phi_X(t) \phi_Y(t)=e^{(\lambda + \mu)(e^{it}-1)},\]
		      	ce qui correspond à la fonction caractéristique d'une loi de Poisson de paramètre $\lambda+\mu$. Donc $X+Y \sim \mathcal{P}(\lambda+\mu)$. }
	\end{enumerate}}
