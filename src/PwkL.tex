\titre{Estimation des paramètres d'une loi normale}
\theme{statistiques}
\auteur{Maxime NGUYEN}
\organisation{AMSCC}

\texte{ 	On considère un échantillon $X_1,...,X_n$ suivant une loi normale $\mathcal{N}(\mu,\sigma^2)$. On cherche un estimateur de $\mu$ et de $\sigma$ par la méthode du maximum de vraisemblance. On note $(x_1,...,x_n)$ une réalisation de cet échantillon. On rappelle que la densité d'une loi normale est $$f(x)=\frac{1}{\sigma \sqrt{2\pi}} e^{-\frac{(x-\mu)^2}{2\sigma^2}}$$ }
\begin{enumerate}
	\item \question{ Exprimer la fonction de vraisemblance $L(x_1,...,x_n,\mu,\sigma)$, puis son logarithme. }
	\item \question{ Dériver $\ln L(x_1,...,x_n,\mu,\sigma)$ par rapport à $\mu$. }
	\item \question{ En déduire un estimateur de $\mu$. }
	\item \question{ Déterminer un estimateur de $\sigma$ avec une démarche analogue. } 
\end{enumerate}