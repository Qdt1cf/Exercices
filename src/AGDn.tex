\titre{Calcul de déterminant}
\theme{calcul déterminant}
\auteur{}
\organisation{AMSCC}

\question{ Calculer :
$$
\Delta=\left|\begin{array}{cccc}
	a^2 & a b & a b & b^2 \\
	a b & a^2 & b^2 & a b \\
	a b & b^2 & a^2 & a b \\
	b^2 & a b & a b & a^2
\end{array}\right|,
$$
où $a$ et $b$ sont deux réels non nuls. }

\reponse{ 
$$
\begin{aligned}
	& \Delta=\left|\begin{array}{cccc}
		a^2 & a b & a b & b^2 \\
		a b & a^2 & b^2 & a b \\
		a b & b^2 & a^2 & a b \\
		b^2 & a b & a b & a^2
	\end{array}\right|=\ell_2-\ell_3\left|\begin{array}{cccc} 
		& \ell_4 \\
		a^2-b^2 & 0 & 0 & b^2-a^2 \\
		0 & a^2-b^2 & b^2-a^2 & 0 \\
		a b & b^2 & a^2 & a b \\
		b^2 & a b & a b & a^2
	\end{array}\right| \\
	& =\left(a^2-b^2\right)^2 \cdot\left|\begin{array}{cccc}
		1 & 0 & 0 & -1 \\
		0 & 1 & -1 & 0 \\
		a b & b^2 & a^2 & a b \\
		b^2 & a b & a b & a^2
	\end{array}\right|=(a-b)^2 \cdot(a+b)^2 \cdot\left|\begin{array}{cccc}
		1 & 0 & 0 & -1 \\
		0 & 1 & -1 & 0 \\
		a b & b^2 & a^2 & a b \\
		b^2 & a b & a b & a^2
	\end{array}\right| \\
	& c_1+c_2+c_3+c_4 \\
	& =(a-b)^2 \cdot(a+b)^2 \cdot\left|\begin{array}{cccc}
		0 & 0 & 0 & -1 \\
		0 & 1 & -1 & 0 \\
		(a+b)^2 & b^2 & a^2 & a b \\
		(a+b)^2 & a b & a b & a^2
	\end{array}\right| \\
	& =(a-b)^2 \cdot(a+b)^4 \cdot\left|\begin{array}{cccc}
		0 & 0 & 0 & -1 \\
		0 & 1 & -1 & 0 \\
		1 & b^2 & a^2 & a b \\
		1 & a b & a b & a^2
	\end{array}\right|=(a-b)^2 \cdot(a+b)^4 \cdot\left|\begin{array}{ccc}
		0 & 1 & -1 \\
		1 & b^2 & a^2 \\
		1 & a b & a b
	\end{array}\right| \\
	& \Delta^{\prime}=\left|\begin{array}{ccc}
		0 & 1 & -1 \\
		1 & b^2 & a^2 \\
		1 & a b & a b
	\end{array}\right|=\left|\begin{array}{ccc}
		0 & 1 & 0 \\
		1 & b^2 & a^2+b^2 \\
		1 & a b & 2 a b
	\end{array}\right|=-\left|\begin{array}{cc}
		1 & a^2+b^2 \\
		1 & 2 a b
	\end{array}\right|=a^2+b^2-2 a b \\
	& \Delta=(a-b)^4 \cdot(a+b)^4 \\
	&
\end{aligned}
$$ }