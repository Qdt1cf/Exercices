\uuid{vnXU}
\titre{ Loi log-normale }
\theme{variables aléatoires à densité}
\auteur{ Djessy Vianne}
\datecreate{2023-12-01} 
\organisation{ AMSCC }

\contenu{
    \texte{
    Soit $X$ une variable aléatoire réelle définie sur un espace probabilisé $(\Omega, \mathcal{T}, \prob)$. On suppose que $X(\Omega) = ]0,+\infty[$ et que $\ln(X)$ suit une loi normale d'espérance $\mu = 1$ et d'écart-type $\sigma = 2$. On note $\Phi$ la fonction de répartition de la loi normale centrée réduite $\mathcal{N}(0,1)$. 
    }
    \begin{enumerate}
        \item \question{ Donner une valeur approchée, à $10^{-2}$ près, de $\prob(-0{,}5 \leq \ln(X) \leq 2{,}5)$. }
        \reponse{
            On a $\prob(-0{,}5 \leq \ln(X) \leq 2{,}5) = \prob\left(\frac{-0{,}5 - 1}{2} \leq \frac{\ln(X) - 1}{2} \leq \frac{2{,}5 - 1}{2}\right) = \prob(-0{,}75 \leq Z \leq 0{,}75)$. 

            On a $\Phi(0{,}75) \approx 0{,}7734$ et $\prob(-0{,}75 \leq Z \leq 0{,}75) = 2 \Phi(0{,}75) - 1 \approx 0{,}55$.
        }
        \item \question{
             On note $F_X$ la fonction de répartition de $X$. Exprimer $F_X$ en fonction de $\Phi$.
        }
        \reponse{
            Soit $t \in \R$. Si $t \leq 0$, alors $F_X(t) = \prob(X \leq t) = 0$ car $X(\Omega) = ]0,+\infty[$. Si $t > 0$, alors : 
            \begin{align*}
                F_X(t) &= \prob(X \leq t) \\
                &= \prob(\ln(X) \leq \ln(t)) \\
                &= \prob\left(\frac{\ln(X) - 1}{2} \leq \frac{\ln(t) - 1}{2}\right) \\
                &= \Phi\left(\frac{\ln(t) - 1}{2}\right)
            \end{align*}
        }
        \item \question{
            En déduire la loi de $X$.
        }
        \reponse{
            La fonction $F_X$ est dérivable presque partout, sa dérivée est une densité de probabilité que l'on note $f$. On note au passage que pour tout $x \in \R$, $\Phi'(x) = \frac{1}{\sqrt{2\pi}} e^{-\frac{x^2}{2}}$. 

               Si $x \leq 0$, alors $f(x) = 0$. Si $x > 0$, alors par dérivation d'une fonction composée : $$f(x) = \frac{1}{2x} \times \frac{1}{\sqrt{2\pi}} e^{-\frac{(\ln(x) - 1)^2}{2}}.$$
        }

    \end{enumerate}

    }