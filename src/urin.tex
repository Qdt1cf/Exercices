\titre{ Calcul d'une somme de série entière}
\theme {séries}
\auteur{ }
\organisation{ AMSCC }

\begin{enumerate}
	\item \question{ Déterminer la somme de la série entière à valeurs réelles $\displaystyle S_1(x)= \sum_{k=0}^{+\infty} (k+1)^2x^k$. On précisera son domaine de convergence. }
	\reponse{ 
		\[\forall x \in ]-1;1[,\quad S_1(x)=1+\frac{1+x}{(1-x)^3}.\]
	}
	
	\item \question{ Après avoir décomposé la fraction rationnelle $\displaystyle \frac{1}{n(n+2)}$ en éléments simples,  déterminer la somme de la série entière à valeurs réelles $\displaystyle S_2(x)=\sum_{n= 1}^{+\infty} \frac{x^n}{n(n+2)}$. On précisera son rayon de convergence. }
	\reponse{ Ici, $R=1$ :
		\[\forall x \in ]-1;1[, \quad S_2(x)=\frac{1}{2}\left( -\ln(1-x)+\frac{1}{x^2}\left[\ln(1-x)+x+\frac{x^2}{2}\right]\right).\]
	}
\end{enumerate}


