\titre{Résolution de système linéaire à paramètre}
\theme{systèmes linéaires}
\auteur{}
\organisation{AMSCC}
\contenu{

\question{ Résoudre les deux systèmes linéaires suivants en distinguant les cas selon les valeurs de $\lambda \in \mathbb{R}$ et $m \in \mathbb{R}$. 

$\left(\mathcal{S}_1\right)\left\{\begin{aligned} x+y+\lambda . z & =\lambda \\ x+\lambda \cdot y-z & =1 \\ x+y-z & =1\end{aligned}\right.$

$\left(\mathcal{S}_2\right)\left\{\begin{aligned} x-2 y+m z & =-m-3 \\ y+z & =m+2 \\ 4 x+y+9 z & =5 m+6 \\ x+y+3 z & =2 m+3\end{aligned}\right.$ }


\reponse{ Résolution de :
$\left(\mathcal{S}_1\right)\left\{\begin{aligned} x+y+\lambda \cdot z & =\lambda \\ x+\lambda \cdot y-z & =1 \\ x+y-z & =1\end{aligned}\right.$
Calculons le déterminant associé à ce système :

- Pour $\lambda \in \mathbb{R} \backslash\{-1 ; 1\}$, par les formules de CRAMER :
$$
\begin{aligned}
& x=\frac{\left|\begin{array}{ccc}
\mathcal{C}_1 & \mathcal{C}_2 & \mathcal{C}_3 \\
\lambda & 1 & \lambda \\
1 & \lambda & -1 \\
1 & 1 & -1
\end{array}\right|}{\left|\begin{array}{ccc}
1 & 1 & \lambda \\
1 & \lambda & -1 \\
1 & 1 & -1
\end{array}\right|}=\frac{\left|\begin{array}{ccc}
\mathcal{C}_1 - \mathcal{C}_3 & \mathcal{C}_2 & \mathcal{C}_3 \\
2\lambda & 1 & \lambda \\
0 & \lambda & -1 \\
0 & 1 & -1
\end{array}\right|}{-(\lambda-1)(\lambda+1)}=\frac{2 \lambda(-\lambda+1)}{-(\lambda-1)(\lambda+1)}=\frac{2 \lambda}{\lambda+1} \\
& y=\frac{\left|\begin{array}{ccc}
1 & \lambda & \lambda \\
1 & 1 & -1 \\
1 & 1 & -1
\end{array}\right|}{\left|\begin{array}{lll}
1 & 1 & \lambda \\
1 & \lambda & -1 \\
1 & 1 & -1
\end{array}\right|}=\frac{0}{-(\lambda-1)(\lambda+1)} \operatorname{car} \ell_2=\ell_3 \\
&
\end{aligned}
$$

La solution de $\left(\mathcal{S}_1\right)$ est : $\left(\frac{2 \lambda}{\lambda+1}, 0, \frac{\lambda-1}{\lambda+1}\right)$.

- Pour $\lambda=1$ :
$\left(\mathcal{S}_1\right)\left\{\begin{array}{l}x+y+z=1 \\ x+y-z=1 \\ x+y-z=1\end{array} \Leftrightarrow\left\{\begin{array}{l}x+y+z=1 \\ x+y-z=1\end{array} \Leftrightarrow\left\{\begin{array}{r}x+y=1 \\ z=0\end{array}\right.\right.\right.$
L'ensemble des solutions est :
$$
\{(x, 1-x, 0) / x \in \mathbb{R}\}
$$

- Pour $\lambda=-1$ :
$\left(\mathcal{S}_1\right)\left\{\begin{array}{l}x+y-z=-1 \\ x-y-z=1 \\ x+y-z=1\end{array} \quad \ell_1-\ell_3 \Leftrightarrow 0=-2\right.$ Impossible !
Résolution de :
$$
\left(\mathcal{S}_2\right)\left\{\begin{aligned}
x-2 y+m z & =-m-3 \\
y+z & =m+2 \\
4 x+y+9 z & =5 m+6 \\
x+y+3 z & =2 m+3
\end{aligned}\right.
$$
Remarque : $\ell_3=4 \ell_4-3 \ell_2$.

- Pour $m \in \mathbb{R}^*$, par les formules de Cramer :

$$
\begin{aligned}
x & =\frac{\left|\begin{array}{ccc}
-m-3 & -2 & m \\
m+2 & 1 & 1 \\
2 m+3 & 1 & 3
\end{array}\right|}{\left|\begin{array}{ccc}
1 & -2 & m \\
0 & 1 & 1 \\
1 & 1 & 3
\end{array}\right|}=\frac{\left|\begin{array}{ccc}
-m-1 & -2 & m \\
m+1 & 1 & 1 \\
2 m+2 & 1 & 3
\end{array}\right|}{-m} \\
& =\frac{-m(m+1)}{-m} \\
& =m+1
\end{aligned}
$$

De même, 

$$y = 
\frac{\left|\begin{array}{ccc}
1 & -m-3 & m \\
0 & m+2 & 1 \\
1 & 2 m+3 & 3
\end{array}\right|}{\left|\begin{array}{ccc}
1 & -2 & m \\
0 & 1 & 1 \\
1 & 1 & 3
\end{array}\right|} =\frac{-m(m+2)}{-m} = m+2 
$$

et enfin : 

$$z = 
\frac{\left|\begin{array}{ccc}
1 & -2 & -m-3 \\
0 & 1 & m+2 \\
1 & 1 & 2 m+3
\end{array}\right|}{\left|\begin{array}{ccc}
1 & -2 & m \\
0 & 1 & 1 \\
1 & 1 & 3
\end{array}\right|}
=\frac{\left|\begin{array}{ccc}
1 & -2 & -m-3 \\
0 & 1 & m+2 \\
0 & 3 & 3 m+6
\end{array}\right|}{-m}
=\frac{\left|\begin{array}{ccc}
1 & -2 & -m-3 \\
0 & 1 & m+2 \\
0 & 0 & 0
\end{array}\right|}{-m}
=0
$$
La solution est :
$$
\{(m+1, m+2,0)\}
$$
- Pour $m=0$ :
$$
\left(\mathcal{S}_2\right) \Leftrightarrow\left\{\begin{array} { r l } 
{ x - 2 y } & { = - 3 } \\
{ y + z } & { = 2 } \\
{ x + y + 3 z } & { = 3 }
\end{array} \Leftrightarrow \quad \begin{array} { r l } 
{ \ell _ { 1 } }
\end{array} \left\{\begin{array} { r l } 
{ x - 2 y } & { = - 3 } \\
{ y + z } & { = 2 } \\
{ 3 y + 3 z } & { = 6 }
\end{array} \Leftrightarrow \left\{\begin{array}{rl}
x-2 y & =-3 \\
y+z & =2
\end{array}\right.\right.\right.
$$
L'ensemble des solutions est : $\{(1-2 z, 2-z, z) / z \in \mathbb{R}\}$. }}
