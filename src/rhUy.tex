\uuid{rhUy}
\chapitre{Statistique}
\sousChapitre{Tests d'hypothèses, intervalle de confiance}
\titre{Comparaison de moyennes}
\theme{tests d'hypothèses}
\auteur{}
\datecreate{2022-10-19}
\organisation{AMSCC}
\contenu{


  On étudie la consommation du chauffage pendant les mois d'hiver, dans deux régions $A$ et $B$. On suppose que la consommation dans ces régions est distribuée selon une loi normale. On prélève un échantillon aléatoire de ménages dans les deux régions. Après calcul, on observe sur ces deux échantillons les valeurs :

\begin{description}
	\item[région A :] échantillon de taille $n_1 = 200$, moyenne  $\overline{x}_1 = 600$, variance  $S^2_{(1)} = \numprint{20000}$
	\item[région B :] échantillon de taille $n_2 = 100$, moyenne  $\overline{x}_2 = 500$, variance  $S^2_{(2)} = \numprint{160000}$
\end{description}

L'écart de consommation entre les deux régions est-il significatif à $5\%$ près ?}
