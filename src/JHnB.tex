\uuid{JHnB}
\chapitre{Probabilité discrète}
\niveau{L2}
\module{Probabilité et statistique}
\sousChapitre{Probabilité conditionnelle}
\titre{Probabilité bayésienne}
\theme{probabilités conditionnelles}
\auteur{}
\datecreate{2023-07-05}
\organisation{AMSCC}

\difficulte{2}
\contenu{
    \texte{ Un poison vient d'être ingéré par une personne. L'examen des lieux laisse penser que trois poisons $p_1$, $p_2$ et $p_3$ seulement peuvent être incriminés. Le poison $p_1$ a une probabilité $\frac{1}{10}$ d'avoir été ingéré, le poison $p_2$ une probabilité de $\frac{1}{2}$ et le poison $p_3$ une probabilité de $\frac{2}{5}$. 
    
    De plus, on sait que chaque poison provoque chez la personne qui l'a ingéré un signe clinique $s$ particulier, mais avec des probabilités différentes car ils n'ont pas la même composition. Ainsi, le poison $p_1$ provoque le signe clinique $s$ avec une probabilité de $\frac{4}{5}$. Ce même signe est observable respectivement avec les probabilités $\frac{1}{50}$ pour le poison $p_2$ et $\frac{2}{5}$ pour le poison $p_3$. 
    }

    \question{ Quel est le poison qui a la plus grande probabilité d'avoir été absorbé, sachant que le signe clinique $s$ est observé sur le patient ? }

    \reponse{Soit $A$ (respectivement $B$ et $C$) l'événement \og{} la personne a ingéré le poison $p_1$ (respectivement $p_2$ et $p_3$)\fg. Soit $S$ l'événement \og{} la personne présente le signe clinique $s$\fg. \\
    On cherche à déterminer quel poison a la probabilité la plus élevée d'avoir été ingéré, sachant que le signe clinique $s$ a été observé. Autrement dit, cela revient à calculer les probabilités suivantes :
    \begin{itemize}
     \item $\prob(A|S)=\frac{\prob(A\cap S)}{p(S)}=\frac{\prob(A)\times \prob(S|A)}{\prob(S)}$
     \item $\prob(B|S)=\frac{\prob(B)\times \prob(S|B)}{\prob(S)}$
     \item $\prob(C|S)=\frac{\prob(C)\times \prob(S|C)}{\prob(S)}$
    \end{itemize}
    or par la formule des probabilités totales, on a :
    \begin{align*}
     \prob(S)&=\prob(S|A)\prob(A)+\prob(S|B)\prob(B)+\prob(S|C)\prob(C) \\
     &= \frac{4}{5}\times \frac{1}{10}+\frac{1}{50}\times \frac{1}{2}+\frac{2}{5}\times \frac{2}{5} \\
     &=\frac{1}{4}.
    \end{align*}
    Ainsi on a :
    \begin{align*}
     \prob(A|S)&=\frac{\frac{1}{10}\times\frac{4}{5}}{\frac{1}{4}}=\frac{8}{25} \\
     \prob(B|S)&= \frac{\frac{1}{2}\times \frac{1}{50}}{\frac{1}{4}}=\frac{1}{25} \\
     \prob(C|S)&= \frac{\frac{4}{10}\times \frac{2}{5}}{\frac{1}{4}}=\frac{16}{25}.
    \end{align*}
    Le poison $p_3$ est donc le plus probable.

    On peut remarquer qu'une partie des calculs n'étaient pas nécessaires ici : comme on a 
        $$\prob(A|S) = \frac {\prob(A\cap S)}{\prob(S)},~~ \prob(B|S) = \frac {\prob(B\cap S)}{\prob(S)},
            ~~ \text{et } \prob(C|S) = \frac {\prob(C\cap S)}{\prob(S)},$$
    il suffit en fait de comparer $\prob(A\cap S)$, $\prob(B\cap S)$ et $\prob(C\cap S)$ pour obtenir la conclusion.}
}

