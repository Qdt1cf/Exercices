\uuid{4Idt}
\chapitre{Probabilité discrète}
\sousChapitre{Lois de distributions}
\titre{Inégalités et loi de Poisson}
\theme{variables aléatoires, loi de Poisson}
\auteur{Maxime Nguyen}
\datecreate{2023-09-18}
\organisation{AMSCC}

\contenu{

\texte{ Soit $X$ une variable aléatoire de loi de Poisson de paramètre $\lambda>0$. }
\begin{enumerate}
	\item \question{ En utilisant l'inégalité de Markov, montrer que: $\prob(X\geq 2\lambda)\leq\frac{1}{2}$. }
	\reponse{Par l'inégalité de Markov, on a:
		\[ \prob(|X|\geq 2\lambda)\leq \frac{\E(|X|)}{2\lambda}\]
		or $X\sim\mathcal{P}(\lambda)$ donc $X(\Omega)=\N$ et on a
		\[ \prob(X\leq 2\lambda) \geq \frac{\E(X)}{2\lambda}=\frac{\lambda}{2\lambda}=\frac{1}{2}.\]
	}
	
	\item \question{ Montrer que $\prob(|X-\lambda| \geq \lambda) \leq \frac{1}{\lambda}$ et en déduire que $\prob(X\geq 2\lambda)\leq \frac{1}{\lambda}$. }
	\reponse{ Par l'inégalité de Bienaymé-Tchebychev, on a
		\[ \prob(|X-\E(X)|\geq \lambda)\leq \frac{\sigma^2(X)}{\lambda^2}=\frac{\lambda}{\lambda^2}=\frac{1}{\lambda}\]
		donc
		\[ \prob(|X-\lambda|)\geq \frac{1}{\lambda}.\]
		Or
		\[ \prob(X\geq 2\lambda)=\prob(X-\lambda \geq \lambda)
		\leq \prob(|X-\lambda|\geq \lambda)\leq \frac{1}{\lambda},
		\]
		d'où
		\[ \prob(X\geq 2\lambda) \leq \frac{1}{\lambda}.\]
	}
	
	\item \question{ Pour $\lambda$ assez grand, laquelle des deux inégalités est la meilleure ? }
	\reponse{ Pour $\lambda \geq 2$, $\frac{1}{\lambda}\leq \frac{1}{2}$ donc l'inégalité de Bienaymé-Tchebychev est meilleure que celle de Markov.
	}
	
\end{enumerate}
}