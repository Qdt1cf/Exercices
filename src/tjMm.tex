\uuid{tjMm}
\chapitre{Calcul d'intégrales}
\niveau{L1}
\module{Analyse}
\sousChapitre{Calcul approché d'intégrale}
\titre{Deux calculs d'intégrale}
\theme{méthode de Monte Carlo}
\auteur{Maxime Nguyen}
\datecreate{2022-09-24}
\organisation{AMSCC}
\contenu{

\question{ 	Proposer au moins deux méthodes de Monte Carlo permettant de fournir une valeur approchée de l'intégrale :
	$$I = \int_0^1 \cos(x^3)e^{-x}dx$$ }

\reponse{
	On voit que $I=  \int \cos(x^3)e^{-x} \chi_{[0;1]}(x)dx = \mathbb{E}\left( \cos(U^3)e^{-U} \right)$ où $U$ suit une loi $\mathcal{U}([0;1])$. Donc si $U_1,U_2...$ est une suite de VA iid selon la loi  $\mathcal{U}([0;1])$, alors la loi des grands nombres donne la convergence presque sûre :
	$$\frac{\cos(U_1^3)e^{-U_1}+...+\cos(U_n^3)e^{-U_n}}{n} \longrightarrow I$$
	
	Il suffit donc de programmer l'algorithme suivant :
	
	\begin{itemize}
		\item N=1000
		\item S = 0
		\item Pour i variant de 1 à N : \\ 
		U = simulation d'une loi $\mathcal{U}([0;1])$ \\
		$S = S+ \cos(U^3) \times \exp(-U)$
		\item Afficher $S/N$
	\end{itemize}
	
	On voit aussi que $I=  \int \cos(x^3) \chi_{[0;1]}(x)  \chi_{[0;+\infty[}(x) e^{-x}dx = \mathbb{E}\left(\cos(V^3) \chi_{[0;1]}(V) \right)  $ où $V$ suit une loi $\mathcal{E}(1)$.
}}
