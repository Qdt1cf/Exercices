\titre{ Autour d'une loi exponentielle }
\theme{ probabilités }
\auteur{Maxime Nguyen}
\organisation{AMSCC}

\contenu{
  Soient $X$ et $Y$ deux variables aléatoires indépendantes suivant chacune une loi exponentielle $\mathcal{E}(3)$. On rappelle qu'une densité de probabilité de la loi exponentielle $\mathcal{E}(\lambda)$ est donnée par : $$f(x) = \begin{cases} 
    \lambda e^{-\lambda x} & \text{ si } x \geq 0 \\
    0 & \text{ sinon }
\end{cases}.$$
  
  On note $Z = \min(X,Y)$ la variable aléatoire donnant le minimum de $X$ et $Y$. 

\begin{enumerate}
    \item \question{ Déterminer $\prob(X \geq t)$ pour tout $t \in \R$. }
    \item \question{ Déterminer $\prob(Z \geq t)$ pour tout $t \in \R$.}
    %\item \question{ Déterminer la loi de $Z$. }
\end{enumerate}
}