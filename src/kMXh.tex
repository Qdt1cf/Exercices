\titre{Couple de variables aléatoires, question de cours}
\theme{probabilités}
\auteur{}
\organisation{AMSCC}
\contenu{

\texte{ Supposons que $(X,Y)$ suive la loi uniforme sur le disque unité $D$ centré en $(0,0)$. }

\question{ Déterminer la loi de $X$. }

\reponse{Alors $(X,Y)$ admet pour densité $f(x,y)=\frac{1}{\pi} \textbf{1}_D(x,y)$. Alors $X$ admet pour densité $f_X$ définie par  $f_X(x) = \int_{\R}^{} f(x,y)dy = \frac{1}{\pi} \int_{\{y^2<1-x^2\}}^{} dy$.
	
	Or $ y^2<1-x^2 \iff \begin{cases}
		-\sqrt{1-x^2} < y < \sqrt{1-x^2} \\
		-1<x<1
	\end{cases}$ donc $$f_X(x) = \frac{1}{\pi} \textbf{1}_{]-1;1[}(x)\int_{-\sqrt{1-x^2}}^{+\sqrt{1-x^2}} dy =  \frac{2}{\pi} \sqrt{1-x^2} \textbf{1}_{]-1;1[}(x)$$}}
