\titre{Lois pour les statistiques}

\texte{La bestiole est un animal dont le poids est distribué selon une loi normale de moyenne 100 g et d'écart-type 5 g.

 On prélève un échantillon aléatoire de 16 bestioles. On note $X_i$ le poids de la bestiole numéro $i$ ($1 \leq i \leq 16$).}

 \begin{enumerate}
  \item \question{Déterminer la loi suivie par  $$\overline{X}=\frac{\sum\limits_{i=1}^{16} X_i}{16}$$}
  \item \question{Déterminer la loi suivie par $$Q=\frac{\sum\limits_{i=1} ^{16} (X_i-100)^2}{25}$$
Déterminer le réel $q$ tel que $\PP(Q > q) = 0.05$.}
  \item \question{Déterminer la loi suivie par $$V=\frac{\sum\limits_{i=1} ^{16} (X_i-\overline{X})^2}{25}$$
  puis déterminer le réel $v$ tel que $\PP(V > v) = 0.05$.}
  \item \question{Déterminer la loi suivie par $$W=\frac{(\overline{X}-100)4\sqrt{15}}{\sqrt{\sum\limits_{i=1} ^{16} (X_i-\overline{X})^2}}$$
    Déterminer le réel $w$ tel que $\PP(W > w) = 0.05$.}
 \end{enumerate}
