\uuid{attt}
\titre{Polynômes et décomposition en facteurs irréductibles}
\theme{Algèbre}
\auteur{Q. Liard}
\organisation{AMSCC}

\contenu{
\texte{L'enjeu de cet exercice est la factorisation du polynôme $R(X)=X^9+X^6+X^3+1$.
}
\begin{enumerate}
\item
\question{Déterminer les coefficients du polynôme $Q$ donné par $Q(X)=(X-1)P(X)$ où $P$ est le polynôme $P$ défini par $P(X)=X^3+X^2+X+1$ }
\reponse{$Q(X)=X^4+X^3+X^2+X-X^3-X^2-X-1=X^4-1.$}
\item \question{Donner les racines complexes de $Q$.}
\reponse{$1$ et $-1$ sont les racines réelles de $Q$ et $-i$ et $i$ sont les racines complexes de $Q$.}
\item \question{En déduire une factorisation du polynôme $P$ en produit de facteurs irréductibles dans $\mathbb{C}[X]$.}
\reponse{On peut écrire $Q$ comme un produit de facteurs de degré $1$ dans $\mathbb{C}[X]:$ $Q(X)=(X-1)(X+1)(X-i)(X+i).$}
\item \question{Établir une égalité entre le polynôme $P$ et $R$.}
\reponse{$R(X)=P(X^3)$}
\item \question{Donner la décomposition en produit de facteurs irréductibles sur $\mathbb{R}[X]$ de $X^3+1$.}
\reponse{$-1$ est une racine réelle de $X^3+1$ et $X^3+1=(X+1)(X^2-X+1)$. Ainsi $x_1=\exp(-i\pi/3)$ et $\overline{x_1}$ sont les racines complexes de $X^2-X+1$ donc de $X^3+1$.}
\item \question{En déduire la décomposition en produit de facteurs irréductibles de $R$ sur $\mathbb{C}[X]$.\\
\emph{Pour cette question on pourra utiliser que les racines complexes de $X^6+1$ sont les $z_k=\exp(i(\pi+2k\pi)/6)$ avec $k=0,\cdots{},5$.}}
\reponse{On peut écrire: 
$$R(X)=P(X^3)=(X^3+1)(X^6+1).$$ Les racines complexes de $X^6+1$ sont les $z_k=\exp(i(\pi+2k\pi)/6)$ avec $k=0,\cdots{},5$. On obtient:
$$R(X)=(X+1)(X-x_1)(X-\overline{x_1})\prod_{k=0}^{5}(X-z_k).$$}
\end{enumerate}
}
