\uuid{Syo2}
\titre{ {TITRE} }
\theme{calcul matriciel}
\auteur{Maxime NGUYEN}
\datecreate{19/12/2024}
\organisation{AMSCC}

\contenu{

\texte{ Soit la matrice $U_{3} = \begin{pmatrix} 1 & 1 & 1 \\ 1 & 1 & 1 \\ 1 & 1 & 1 \end{pmatrix} \in \mathcal{M}_{3}(\mathbb{R})$ et la matrice identité $I_{3} = \begin{pmatrix} 1 & 0 & 0 \\ 0 & 1 & 0 \\ 0 & 0 & 1 \end{pmatrix} \in \mathcal{M}_{3}(\mathbb{R})$. 

Pour tout réels $a,b$, on note $M(a,b) = aU + bI_{3}$.
 }

 \begin{enumerate}
	\item \question{Montrer que $M(1,-3)$ n'est pas inversible.}
	\indication{}
    \reponse{}
    \item \question{Montrer que $M(1,1)$ est inversible et déterminer son inverse.}
    \indication{}
    \item \question{ Montrer que $M(a,b)$ est inversible dans $\mathcal{M}_{3}(\mathbb{R})$ si et seulement si $b(b+3a) \neq 0$ et exrimer $M(a,b)^{-1}$ sous la forme $a'U + b'I_{3}$ où $a'$ et $b'$ sont des réels à déterminer.}
\end{enumerate}

}