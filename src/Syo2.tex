\chapitre{Matrice}
\sousChapitre{Propriétés élémentaires, généralités}
\uuid{Syo2}
\titre{\'Etude d'une matrice à paramètres}
\theme{calcul matriciel}
\auteur{Maxime NGUYEN}
\datecreate{2024-12-19}
\organisation{AMSCC}

\contenu{

\texte{ Soit la matrice $U= \begin{pmatrix} 1 & 1 & 1 \\ 1 & 1 & 1 \\ 1 & 1 & 1 \end{pmatrix} \in \mathcal{M}_{3}(\mathbb{R})$ et la matrice identité $I_{3} = \begin{pmatrix} 1 & 0 & 0 \\ 0 & 1 & 0 \\ 0 & 0 & 1 \end{pmatrix} \in \mathcal{M}_{3}(\mathbb{R})$. 

Pour tous réels $a,b$, on note $M(a,b) = aU + bI_{3}$.
 }

 \begin{enumerate}
	\item \question{Montrer que $M(1,-3)$ n'est pas inversible.}
	%\indication{}
    \reponse{La matrice $M(1,-3)$ s'\'ecrit :
    \[
    M(1,-3) = U - 3I_{3} = \begin{pmatrix} -2 & 1 & 1 \\ 1 & -2 & 1 \\ 1 & 1 & -2 \end{pmatrix}.
    \]
    Pour d\'eterminer si $M(1,-3)$ est inversible, calculons son d\'eterminant :
    \[
    \det(M(1,-3)) = \det\begin{pmatrix} -2 & 1 & 1 \\ 1 & -2 & 1 \\ 1 & 1 & -2 \end{pmatrix}.
    \]
    En sommant la colonne $1$ et $2$ on obtient une troisième colonne nulle. En cons\'equence, $\det(M(1,-3)) = 0$. La matrice n'est donc pas inversible.
}
    \item \question{Montrer que $M(1,1)$ est inversible et déterminer son inverse.}
  %  \indication{}
  \reponse{   La matrice $M(1,1)$ s'\'ecrit :
    \[
    M(1,1) = U+ I_{3} = \begin{pmatrix} 2 & 1 & 1 \\ 1 & 2 & 1 \\ 1 & 1 & 2 \end{pmatrix}.
    \]
    Calculons son d\'eterminant :
    \[
    \det(M(1,1)) = \det\begin{pmatrix} 2 & 1 & 1 \\ 1 & 2 & 1 \\ 1 & 1 & 2 \end{pmatrix} = 2(2 \cdot 2 - 1 \cdot 1) - 1(1 \cdot 2 - 1 \cdot 1) + 1(1 \cdot 1 - 1 \cdot 2).
    \]
    Simplifions :
    \[
    \det(M(1,1)) = 2(4 - 1) - 1(2 - 1) + 1(1 - 2) = 2 \cdot 3 - 1 \cdot 1 - 1 = 6 - 1 - 1 = 4.
    \]
    Comme $\det(M(1,1)) \neq 0$, $M(1,1)$ est inversible. 
    Avec le méthode du pivot de Gauss appliquée également à la matrice augmentée avec $I_{3}$ on obtient:
    $M(1,1)^{-1}=\begin{pmatrix} \frac{3}{4} & -\frac{1}{4} & -\frac{1}{4} \\ -\frac{1}{4} &\frac{3}{4} & -\frac{1}{4} \\ -\frac{1}{4} & -\frac{1}{4} & \frac{3}{4} \end{pmatrix}$
    
    }
    \item \question{ Montrer que $M(a,b)$ est inversible dans $\mathcal{M}_{3}(\mathbb{R})$ si et seulement si $b(b+3a) \neq 0$ et exprimer $M(a,b)^{-1}$ sous la forme $a'U + b'I_{3}$ où $a'$ et $b'$ sont des réels à déterminer.}
    \reponse{
    La matrice $M(a,b)$ s'\'ecrit :
    \[
    M(a,b) = aU+ bI_{3}.
    \]
    Calculons le d\'eterminant:
    \[
    \det(M(a,b)) = \det(aU + bI_{3}).
    \]
    Le d\'eterminant est nul si $b = 0$ ou si $b + 3a = 0$. En effet:\\
En sommant la colonne $1$ et $2$ puis en factorisant par $b+3a$ on obtient un déterminant avec une colonne de $1$. En utilisant le pivot de Gauss on se retrouve avec 
$$ \det(M(a,b))=(3a+b)b^2.$$
    Si $b(b + 3a) \neq 0$, alors $M(a,b)$ est inversible. Pour exprimer $M(a,b)^{-1}$, on cherche des coefficients $a'$ et $b'$ tels que :
    \[
    M(a,b) \cdot (a' + b'I_{3}) = I_{3}.
    \]
    Le calcul montre que :
    \[
    a' = -\frac{1}{b(b + 3a)}, \quad b' = \frac{1}{b}.
    \]
    Ainsi, l'inverse est donn\'e par :
    \[
    M(a,b)^{-1} = -\frac{1}{b(b + 3a)}U_{3} + \frac{1}{b}I_{3}.
    \]

    }
\end{enumerate}

}
