\titre{ Temps de transport }
\theme{ probabilités }
\auteur{ Maxime Nguyen }
\organisation{ AMSCC }

\contenu{
    \texte{ Une commune a lancé une étude concernant les problèmes liés au transport. Sur une ligne de bus, une enquête a permis de révéler que le retard (ou l'avance) sur l'horaire officiel du bus à une station donnée est donné par une variable aléatoire $X$, en minutes, suivant une loi normale $\mathcal{N}(5,\sigma)$ où $\sigma >0$ est pour l'instant indéterminé. Cependant, on sait que la probabilité que le retard soit inférieur à $7$ minute est de $p=84{,}13\%$. }

    \begin{enumerate}
        \item \question{  Déterminer la valeur de $\sigma$.  }
        \item \question{  Quelle est la probabilité que le regard soit supérieur à $9$ minutes ? }
        \item \question{ 
            Sachant que le retard est supérieur à $3$ minutes, quelle est la probablité que ce retard soit inférieur à $7$ minutes ?
        }
        \item \question{ Une dame fréquente cette ligne de bus tous les jours pendant 10 jours. On suppose que les retards journaliers sont indépendants. et on note $Y$ le nombre de jours où la dame a attendu moins de $7$ minutes. Déterminer la loi de $Y$, son espérance et sa variance.
        }
        \item 
        \question { 
            Soit $U$ le rang $k$ du jour où pour la première fois, la dame attend plus de $7$ minute, si cet événement se produit. Dans le cas contraire, $U$ prend la valeur $0$. Déterminer, en fonction de $p$, la loi de probabilité de $Z$. 
        }
    \end{enumerate}
}