\uuid{r5Yy}
\chapitre{Statistique}
\niveau{L2}
\module{Probabilité et statistique}
\sousChapitre{Tests d'hypothèses, intervalle de confiance}
\titre{Test du comportement des clients}
\theme{tests d'hypothèses}
\auteur{}
\datecreate{2022-09-28}
\organisation{AMSCC}
\difficulte{}
\contenu{

\texte{ 	Soit $X$ le nombre d'incidents de paiements pour un crédit à la consommation observés sur la durée du prêt. On suppose que $X$ suit une loi de Poisson de paramètre $\lambda$ positif. On dispose d'un échantillon de $n$ clients appartenant à la banque A. }
	
	\begin{enumerate}
		\item \question{ Donner un estimateur de $\lambda$ centré et convergent. }
		\item \texte{ On désire tester l'hypothèse nulle selon laquelle les clients de la banque A sont faiblement risqués sous la forme suivante : $\begin{cases}
		 \lambda = 1 \\
		 \lambda > 1
		\end{cases}$. 

Pour un échantillon de 100 clients de la banque A, nous observons les résultats suivants :

\begin{center}
	\begin{tabular}{|c|c|c|c|c|c|c|c|c|}
		\hline Incidents : & 0 & 1 & 2 & 3 & 4 & 5 & 6 & Total \\ 
		\hline Effectifs : & 38 & 31 & 16 & 10 & 2 & 2 & 1 & 100 \\ 
		\hline 
	\end{tabular} 
\end{center}	
} 

\question{ Que peut-on conclure pour un risque de première espèce de $5\%$ ? Sur un même graphique, superposer une représentation graphique de la loi de Poisson de paramètre $\lambda = 1$ et l'histogramme des effectifs observés. }
\end{enumerate}}
