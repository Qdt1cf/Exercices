\chapitre{Déterminant, système linéaire}
\sousChapitre{Calcul de déterminants}
\uuid{kOVx}
\titre{Déterminant d'un produit}
\theme{calcul déterminant}
\auteur{}
\datecreate{2023-01-11}
\organisation{AMSCC}
\contenu{

\begin{enumerate}
	\item 
\question{ En utilisant le déterminant, justifier l'existence d'une matrice $A$ vérifiant :
$$
\begin{pmatrix}
	5 & 4 & 3 \\
	7 & 6 & 5 \\
	2 & 0 & 1
\end{pmatrix}=\begin{pmatrix}
	0 & 1 & -1 \\
	1 & 1 & 1 \\
	1 & 0 & 1
\end{pmatrix} \times A
$$
puis déterminer $A$. }

\indication{Calculer et utiliser l'inverse de la matrice $\begin{pmatrix}0 & 1 & -1 \\ 1 & 1 & 1 \\ 1 & 0 & 1\end{pmatrix}$. }

\reponse{ Remarque :
Compte-tenu des règles du produit matriciel, si elles existent, les matrices $A$ et $B$ sont des matrices $3 \times 3$.
Puisque $\begin{vmatrix}0 & 1 & -1 \\ 1 & 1 & 1 \\ 1 & 0 & 1\end{vmatrix} \neq 0$, la matrice $M=\begin{pmatrix}0 & 1 & -1 \\ 1 & 1 & 1 \\ 1 & 0 & 1\end{pmatrix}$ est inversible et :
$$
\begin{pmatrix}
	0 & 1 & -1 \\
	1 & 1 & 1 \\
	1 & 0 & 1
\end{pmatrix}^{-1} \times \begin{pmatrix}
	5 & 4 & 3 \\
	7 & 6 & 5 \\
	2 & 0 & 1
\end{pmatrix}=\underbrace{\left(\begin{array}{ccc}
		0 & 1 & -1 \\
		1 & 1 & 1 \\
		1 & 0 & 1
	\end{array}\right)^{-1} \cdot\left(\begin{array}{ccc}
		0 & 1 & -1 \\
		1 & 1 & 1 \\
		1 & 0 & 1
	\end{array}\right)}_{I d} \cdot A=A
$$
On calcule $\left(\begin{array}{ccc}0 & 1 & -1 \\ 1 & 1 & 1 \\ 1 & 0 & 1\end{array}\right)^{-1}=\left(\begin{array}{ccc}1 & -1 & 2 \\ 0 & 1 & -1 \\ -1 & 1 & -1\end{array}\right)$ par l'une ou l'autre méthode d'inversion.
Méthode 1 :
Par la méthode du pivot de Gauss. Posons :
$$
M=\left(\begin{array}{ccc}
	0 & 1 & -1 \\
	1 & 1 & 1 \\
	1 & 0 & 1
\end{array}\right) \quad I_3=\left(\begin{array}{ccc}
	1 & 0 & 0 \\
	0 & 1 & 0 \\
	0 & 0 & 1
\end{array}\right)
$$
Permutons la première et la troisième ligne :
$$
\left(\begin{array}{ccc}
	1 & 0 & 1 \\
	1 & 1 & 1 \\
	0 & 1 & -1
\end{array}\right) \quad\left(\begin{array}{lll}
	0 & 0 & 1 \\
	0 & 1 & 0 \\
	1 & 0 & 0
\end{array}\right)
$$
Retranchons la première ligne de la deuxième :
$$
\left(\begin{array}{ccc}
	1 & 0 & 1 \\
	0 & 1 & 0 \\
	0 & 1 & -1
\end{array}\right) \quad\left(\begin{array}{ccc}
	0 & 0 & 1 \\
	0 & 1 & -1 \\
	1 & 0 & 0
\end{array}\right)
$$
Retranchons la deuxième ligne de la troisième :
$$
\left(\begin{array}{ccc}
	1 & 0 & 1 \\
	0 & 1 & 0 \\
	0 & 0 & -1
\end{array}\right) \quad\left(\begin{array}{ccc}
	0 & 0 & 1 \\
	0 & 1 & -1 \\
	1 & -1 & 1
\end{array}\right)
$$
Changeons le signe de la troisième ligne :
$$
\left(\begin{array}{lll}
	1 & 0 & 1 \\
	0 & 1 & 0 \\
	0 & 0 & 1
\end{array}\right) \quad\left(\begin{array}{ccc}
	0 & 0 & 1 \\
	0 & 1 & -1 \\
	-1 & 1 & -1
\end{array}\right)
$$
Retranchons la troisième ligne de la première :
$$
I_3=\left(\begin{array}{lll}
	1 & 0 & 0 \\
	0 & 1 & 0 \\
	0 & 0 & 1
\end{array}\right) \quad\left(\begin{array}{ccc}
	1 & -1 & 2 \\
	0 & 1 & -1 \\
	-1 & 1 & -1
\end{array}\right)=M^{-1}
$$ }



\item \question{ Peut-on déterminer une matrice $B$ telle que :
$$
\begin{pmatrix}
	5 & 4 & 3 \\
	7 & 6 & 5 \\
	2 & 0 & 1
\end{pmatrix}=B \times \begin{pmatrix}
	1 & 2 & 3 \\
	2 & 3 & 4 \\
	1 & 2 & 3
\end{pmatrix} \text{ ?}
$$ }

\reponse{ En revanche, puisque $\forall M, N \in \mathcal{M}_n(\mathbb{R}) \operatorname{det}(M \times N)=\operatorname{det}(M) \times \operatorname{det}(N)$, il n'existe pas de matrice $B \in \mathcal{M}_3(\mathbb{R})$ telle que $\left(\begin{array}{lll}5 & 4 & 3 \\ 7 & 6 & 5 \\ 2 & 0 & 1\end{array}\right)=B \cdot\left(\begin{array}{lll}1 & 2 & 3 \\ 2 & 3 & 4 \\ 1 & 2 & 3\end{array}\right)$ car $\left|\begin{array}{lll}5 & 4 & 3 \\ 7 & 6 & 5 \\ 2 & 0 & 1\end{array}\right| \neq 0$ et $\left|\begin{array}{lll}1 & 2 & 3 \\ 2 & 3 & 4 \\ 1 & 2 & 3\end{array}\right|=0$. }
\end{enumerate}}
