\chapitre{Matrice}
\sousChapitre{Propriétés élémentaires, généralités}
\uuid{svRW}
\titre{Trace de matrices}
\theme{calcul matriciel}
\auteur{}
\datecreate{2023-01-03}
\organisation{AMSCC}
\contenu{

\texte{ On rappelle que si $A=\left(a_{i j}\right)_{\substack{1 \leq i \leq n \\ 1 \leq j \leq n}}$, alors la trace de $A$ est : $\operatorname{Tr} A=\sum_{i=1}^n a_{i i}$ (autrement dit $\operatorname{Tr} A$ est la somme des termes de la diagonale de $A$ ). 

Soit la matrice $A=\begin{pmatrix}3 & 2 \\ 2 & 3\end{pmatrix}$. }

\begin{enumerate}
	\item \question{ Calculer $\operatorname{Tr} A$ et $\operatorname{det} A$. Vérifier que $A^2-(\operatorname{Tr} A) \cdot A+(\operatorname{det} A) I_2=0$, où $I_2$ est la matrice identité $2 \times 2$. }
	\item \question{ En déduire que $A$ est inversible et exprimer $A^{-1}$. }
\end{enumerate} 

\reponse{ On a $\operatorname{Tr} A=3+3=6$ et $\operatorname{det} A=3 \times 3-2 \times 2=5$.
	Par ailleurs :
	$$
	A^2-(\operatorname{Tr} A) \cdot A+(\operatorname{det} A) I_2=A^2-6 \cdot A+5 I_2=\left(\begin{array}{ll}
		13 & 12 \\
		12 & 13
	\end{array}\right)-6\left(\begin{array}{ll}
		3 & 2 \\
		2 & 3
	\end{array}\right)+5\left(\begin{array}{ll}
		1 & 0 \\
		0 & 1
	\end{array}\right)=\left(\begin{array}{ll}
		0 & 0 \\
		0 & 0
	\end{array}\right)
	$$
	Comme $\operatorname{det} A=5 \neq 0$, on sait que $A$ est inversible, donc $A^{-1}$ existe.
	On multiplie l'égalité ci-dessus à gauche par $A^{-1}$, on obtient :
	$$
	A^{-1} \cdot\left(A^2-(\operatorname{Tr} A) \cdot A+(\operatorname{det} A) I_2\right)=A-(\operatorname{Tr} A) \cdot I_2+(\operatorname{det} A) A^{-1}=0
	$$
	Finalement, ceci donne que $A^{-1}=\frac{1}{\operatorname{det} A}\left(\operatorname{Tr}(A) \cdot I_2-A\right)=\frac{1}{5}(6 I-A)=\frac{1}{5}\left(\begin{array}{cc}3 & -2 \\ -2 & 3\end{array}\right)$.
 }}
