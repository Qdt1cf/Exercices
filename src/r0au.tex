\uuid{r0au}
\chapitre{Fonction de plusieurs variables}
\sousChapitre{Autre}
\titre{Courbes de niveau}
\theme{fonctions de plusieurs variables}
\auteur{}
\datecreate{2023-03-01}
\organisation{AMSCC}
\contenu{

	
		A chaque fonction définie ci-dessous (1-6), associer ses courbes de niveaux (A-F).
\begin{enumerate}
	\item \question{  $f(x,y) = \sin(xy)$ }
	\reponse{1-B : la fonction est périodique en $x$ et en $y$ ; $f$ ne change pas quand on échange $x$ et $y$, i.e. le graphe est symétrique par rapport au plan d’équation $y = x$ ; $f (0, y) = f (x,0) = 0$.}
	\item \question{  $f(x,y) = \sin(x-y)$ }
	\reponse{2-A : la fonction est périodique en $x$ et en $y$ ; $f$ est constante si $y = x +k$ (lignes de niveau parallèles à la droite d'équation $y=x$}
	\item \question{ $f(x,y) = (1-x^2)(1-y^2)$ }
	\reponse{3-F : $f (\pm1, y) = f (x,\pm1) = 0$ ; la trace dans le plan $xz$ est $z = 1-x^2$ et dans le plan $yz$ est $z = 1-y^2$.}
	\item \question{ $f(x,y) = \frac{x-y}{1+x^2+y^2}$ }
	\reponse{4-E : $f (x,x) = 0$ ; $f (x, y) > 0$ si $x > y$ ; $f (x, y) < 0$ si $x < y$.}
	\item \question{ $f(x,y) = e^x \cos(y)$ }
	\reponse{5-D : la fonction est périodique en $y$ ;}
	\item \question{ $f(x,y) = \sin(x)-\sin(y)$ }
	\reponse{6-C : la fonction est périodique en $x$ et en $y$.}
\end{enumerate}

\begin{minipage}{0.5\textwidth}
	\begin{tikzpicture}
\begin{axis}[
title={A},
enlarge x limits,
view={0}{90},
xlabel=$x$, ylabel=$y$,
small,
axis equal,
axis x line=center,
axis y line=center,
%grid=major,
]
\addplot3[
domain=-3:3,
domain y=-3:3,
contour gnuplot={number=14,labels=false},%levels={0.2,0.4,0.6, 0.8,1,2,3,4,5,6,7,8,9,10,11,12}},
samples=40,
thick,
]
{sin((x-y) r)};
\end{axis}
\end{tikzpicture}
\end{minipage}
\hfill
\begin{minipage}{0.5\textwidth}
	\begin{tikzpicture}
\begin{axis}[
title={B},
enlarge x limits,
view={0}{90},
xlabel=$x$, ylabel=$y$,
small,
axis equal,
axis x line=center,
axis y line=center,
%grid=major,
]
\addplot3[
domain=-3:3,
domain y=-3:3,
contour gnuplot={number=14,labels=false},%levels={0.2,0.4,0.6, 0.8,1,2,3,4,5,6,7,8,9,10,11,12}},
samples=40,
thick,
]
{sin((x*y) r)};
\end{axis}
\end{tikzpicture}
\end{minipage}

\begin{minipage}{0.5\textwidth}
	\begin{tikzpicture}
\begin{axis}[
title={C},
enlarge x limits,
view={0}{90},
xlabel=$x$, ylabel=$y$,
small,
axis equal,
axis x line=center,
axis y line=center,
%grid=major,
]
\addplot3[
domain=-3:3,
domain y=-3:3,
contour gnuplot={number=14,labels=false},%levels={0.2,0.4,0.6, 0.8,1,2,3,4,5,6,7,8,9,10,11,12}},
samples=40,
thick,
]
{sin((x) r) - sin((y) r)};
\end{axis}
\end{tikzpicture}
\end{minipage}
\hfill
\begin{minipage}{0.5\textwidth}
	\begin{tikzpicture}
\begin{axis}[
title={D},
enlarge x limits,
view={0}{90},
xlabel=$x$, ylabel=$y$,
small,
axis equal,
axis x line=center,
axis y line=center,
%grid=major,
]
\addplot3[
domain=-3:3,
domain y=-3:3,
contour gnuplot={number=14,labels=false},%levels={0.2,0.4,0.6, 0.8,1,2,3,4,5,6,7,8,9,10,11,12}},
samples=40,
thick,
]
{exp(x)*cos((y) r)};
\end{axis}
\end{tikzpicture}
\end{minipage}

\begin{minipage}{0.5\textwidth}
	\begin{tikzpicture}
\begin{axis}[
title={E},
enlarge x limits,
view={0}{90},
xlabel=$x$, ylabel=$y$,
small,
axis equal,
axis x line=center,
axis y line=center,
%grid=major,
]
\addplot3[
domain=-3:3,
domain y=-3:3,
contour gnuplot={number=14,labels=false},%levels={0.2,0.4,0.6, 0.8,1,2,3,4,5,6,7,8,9,10,11,12}},
samples=40,
thick,
]
{(x-y)/(1+x^2+y^2)};
\end{axis}
\end{tikzpicture}
\end{minipage}
\hfill
\begin{minipage}{0.5\textwidth}
	\begin{tikzpicture}
\begin{axis}[
title={F},
enlarge x limits,
view={0}{90},
xlabel=$x$, ylabel=$y$,
small,
axis equal,
axis x line=center,
axis y line=center,
%grid=major,
]
\addplot3[
domain=-3:3,
domain y=-3:3,
contour gnuplot={number=14,labels=false},%levels={0.2,0.4,0.6, 0.8,1,2,3,4,5,6,7,8,9,10,11,12}},
samples=40,
thick,
]
{(1-x^2)*(1-y^2)};
\end{axis}
\end{tikzpicture}
\end{minipage}

}
