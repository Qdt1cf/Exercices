\uuid{64Uj}
\chapitre{Equation différentielle}
\niveau{L2}
\module{Analyse}
\sousChapitre{Résolution d'équation différentielle}
\titre{Résolution d'une EDO par un schéma numérique}
\theme{méthodes numériques, équations différentielles}
\auteur{}
\datecreate{2024-04-29}
\organisation{AMSCC}
\difficulte{}
\contenu{
\texte{ Soit l'équation différentielle sur $[0,1]$:
$$(E)	\begin{cases} 
	y'(t) = f(t,y(t)) \\
	y(0) = 0,
\end{cases} $$
où
$$
f(t,y)=\sin\left((t+y)^2\right).
$$ }
\begin{enumerate}
	\item \question{ Justifier l'existence et l'unicité locale d'une solution $y$ de classe $\mathcal{C}^1$.  }
	\reponse{$f$ est $C^1$: Théorème de Cauchy-Lipschitz.
	}
	
	\item \question{ Justifier le caractère borné de $y$. }
	\reponse{$|y'(t)|\leq 1$, donc $|y(t)|\leq a+T=1$.
	}
	
	\item Justifier l'existence et l'unicité globale d'une solution $y$ de classe $\mathcal{C}^1$. 
	\reponse{Principe de prolongement.
	}
	
	
\texte{ 	Pour approcher la solution de $(E)$, on propose le schéma numérique suivant : 
	$h=1/N$, $t_n=nh$, $y_0=0$ et 
	$$(S) \colon y_{n+1} = y_n +  \frac{h}{4}\left( f(t_n,y_n)+3f\left(t_n+ \frac{2h}{3},y_n+\frac{2h}{3}f(t_n,y_n) \right) \right)$$ }
	
%	Pour caractériser la consistance d'ordre $p \geq 2$, on peut utiliser le résultat suivant : la méthode $y_{n+1}=y_n + hF(t_n,y_n,h)$ est consistante d'ordre $p$ si pour tout $ 0 \leq k \leq p-1$, 
%	$$\frac{\partial^{k}F}{\partial h^{k}}(t,y,0) = \frac{1}{k+1}f^{[k]}(t,y)$$
%	où $f^{[0]} = f$ et $f^{[k+1]} = \frac{\partial f^{[k]}}{\partial t} + f \frac{ \partial f^{[k]}}{\partial y}$.
	
	
	\item \question{ Démontrer que ce schéma est convergent. }
	\reponse{On vérifie qu'il est consistant en appliquant le résultat du cours : on écrit le schéma sous la forme standard $y_{n+1} = y_n+hF(t_n,y_n,h)$ et on vérifie que $F(t,y,0) = f(t,y)$.
		
		Par ailleurs, on vérifie facilement que $F$ est lipschitzienne par rapport à la deuxième variable $y$, c'est une condition suffisante de stabilité du schéma. 
	}
	
	\item \question{ Vérifier que ce schéma est consistant d'ordre 2. }
	\reponse{On applique le critère énoncé et admis ci-dessus en calculant $\frac{1}{2}f^{[1]}(t,y)$.
	}

    \item Proposer un algorithme qui donne une approximation de $y(1)$ pour $h = 10^{-6}$ en utilisant le schéma numérique ci-dessus. 
\end{enumerate}
}