\titre{Loi normale, question de cours}
\theme{probabilité}
\auteur{Maxime NGUYEN}
\organisation{AMSCC}

\texte{ Soit  $Z$ une variable aléatoire suivant une loi normale centrée réduite $\mathcal{N}(0,1)$. } 

\question{ Justifier que $\PP(Z>0) = \frac{1}{2}$ et que pour tout réel $a \in \R$, $\PP(Z<-a) = \PP(Z>a)$. }

\reponse{ Il suffit de remarquer que la fonction densité est paire et que son intégrale vaut $1$. }
