\uuid{eFy7}
\titre{ Analyse de covariance et moments d'une loi normale }
\theme{loi normale}
\auteur{}
\organisation{AMSCC}
\contenu{


\texte{ On considère deux variables aléatoires réelles \( X \) et \( U \)  indépendantes, \( X \) suivant la loi normale \( \mathcal{N}(0, 1) \) et \( U \) suivant la loi discrète uniforme sur \( \{-1,1\} \).

On pose \( Y = UX \) et on admet que \( Y \) est une variable aléatoire absolument continue.  }

\begin{enumerate}
	\item  \question{ En utilisant la formule des probabilités totales, montrer que :
$$ \forall x \in \mathbb{R}, \quad \mathbb{P}(Y \leq x) = \mathbb{P}(U = 1) \mathbb{P}(X \leq x) + \mathbb{P}(U = -1) \mathbb{P}(X > -x) $$

et en déduire que \( Y \) suit la même loi que \( X \). }

\item \question{  Calculer l'espérance de \( U \), puis montrer que \( \mathbb{E}(XY) = 0 \). En déduire que \( \text{Cov}(X, Y) = 0 \). }

\item \question{ Rappeler la valeur de \( \mathbb{E}(X^2) \) et en déduire que :
$$ \int_{0}^{+\infty} x^2 e^{-\frac{x^2}{2}} \, dx = \frac{\sqrt{\pi}}{2} $$ }

\item \question{ En déduire, s'il existe, le moment d'ordre $4$ de $X$. }

%b) Montrer, grâce à une intégration par parties, que :
%\[ \forall A \in \mathbb{R}_+, \int_{0}^{A} x^4 e^{-\frac{x^2}{2}} \, dx = -A^3 e^{-\frac{A^2}{2}} + 3 \int_{0}^{A} x^2 e^{-\frac{x^2}{2}} \, dx \]
%
%c) En déduire que l'intégrale \( \int_{0}^{+\infty} x^4 e^{-\frac{x^2}{2}} \, dx \) converge et vaut \( \frac{3}{2} \sqrt{2\pi} \).
%
%d) Établir que \( X \) possède un moment d'ordre 4 et que \( \mathbb{E}(X^4) = 3 \).
\end{enumerate}
}