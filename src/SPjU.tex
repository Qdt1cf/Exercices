\uuid{SPjU}
\chapitre{Fonction de plusieurs variables}
\niveau{L2}
\module{Analyse}
\sousChapitre{Limite}
\titre{Limite et continuité d'une fonction définie sur $\R^2$}
\theme{fonctions de plusieurs variables}
\auteur{}
\datecreate{2023-03-01}
\organisation{AMSCC}
\difficulte{}
\contenu{

\question{ 	\'Etudier l'existence et le cas échéant, calculer
$$\lim_{(x,y) \to (0,0)} \frac{x^2-y^2}{x^2+y^2}$$ }

\indication{ On peut se ramener à la limite pour une seule variable en étudiant pour $(x,y)$ sur un chemin particulier et constater que la limite n'est pas toujours la même selon le chemin choisi. }

\reponse{On pose $f(x,y)=\frac{x^2-y^2}{x^2+y^2}$ : cette fonction n'est pas continue en $(0,0)$ et on peut étudier sa limite ; or pour tout $x \in \R$, $y \in \R$, $f(x,0) = 1$ et $f(0,y) = -1$ donc les limites de $f$ le long de chaque axe en $(0,0)$ n'ont pas la même valeur : $f(x,0) \xrightarrow[x \to 0]{} 1$ et $f(0,y) \xrightarrow[y \to 0]{} -1 \neq 1$. 

Cela suffit pour affirmer que $f$ n'admet pas de limite en $(0,0)$.}}
