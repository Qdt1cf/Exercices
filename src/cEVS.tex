\chapitre{Série numérique}
\sousChapitre{Série à  termes positifs}
\uuid{cEVS}
\titre{Séries à termes réels positifs}
\theme{séries}
\auteur{}
\datecreate{2023-05-17}
\organisation{AMSCC}
\contenu{


\texte{ Étudier la convergence des séries $\sum u_n$ suivantes:  }
\colonnes{\solution}{3}{1}
\begin{enumerate}
	\item \question{$u_n=\frac{n}{n^3+1}$} 
	\reponse{  Alors $u_n \geq 0$ pour tout $n \in \N$ et $u_n \underset{+\infty}\sim \frac{n}{n^3} = \frac{1}{n^2}$ donc par comparaison à une série de Riemann convergente ($\alpha = 2>1$), la série $\sum u_n$ converge. } 
	\item \question{$u_n=\frac{\sqrt{n}}{n^2+\sqrt{n}}$}
	\reponse{ Alors $u_n \geq 0$ pour tout $n \in \N$ et $u_n \underset{+\infty}\sim \frac{\sqrt{n}}{n^2} = \frac{1}{n^{3/2}}$ donc par comparaison à une série de Riemann convergente ($\alpha = 3/2>1$), la série $\sum u_n$ converge. }
	\item \question{$u_n=n\sin\Big(\frac{1}{n}\Big)$}
	\reponse{ Alors $u_n \geq 0$ pour tout $n \in \N^*$ et $u_n \underset{+\infty}\sim n \times \frac{1}{n} = 1$ donc la série $\sum u_n$ diverge grossièrement. }
	\item \question{$u_n=\frac{1}{\sqrt{n}}\ln\Big(1+\frac{1}{\sqrt{n}}\Big)$}
	\reponse{ Alors $u_n \geq 0$ pour tout $n \in \N^*$ et $u_n \underset{+\infty}\sim \frac{1}{\sqrt{n}} \times \frac{1}{\sqrt{ n}} = \frac{1}{n}$ donc par comparaison à une série de Riemann divergente ($\alpha = 1<1$), la série $\sum u_n$ diverge. }
	\item \question{$u_n=\frac{\sqrt{n+1}-\sqrt{n}}{n}$}
	\reponse{ Alors $u_n \geq 0$ pour tout $n \in \N^*$ et $u_n = \frac{\sqrt{n+1}-\sqrt{n}}{n} = \frac{(\sqrt{n+1}+ \sqrt{n})(\sqrt{n+1}-\sqrt{n})}{n(\sqrt{n+1}+ \sqrt{n})} = \frac{1}{n(\sqrt{n+1}+ \sqrt{n})} = \frac{1}{n\sqrt{n}\left(\sqrt{1+\frac{1}{n}}+1\right)} \underset{+\infty}\sim \frac{1}{n \times 2\sqrt{n}} = \frac{1}{2n^{3/2}}$ donc par comparaison à une série de Riemann convergente ($\alpha = 3/2>1$), la série $\sum u_n$ converge. }
	\item \question{$u_n=\frac{(-1)^n+n}{n^2+1}$}
	\solution{ Alors $u_n \geq 0$ pour tout $n \in \N^* $ et $u_n = \frac{(-1)^n+n}{n^2+1} \underset{+\infty}\sim \frac{n}{n^2} = \frac{1}{n}$ donc par comparaison à une série de Riemann divergente ($\alpha = 1<1$), la série $\sum u_n$ diverge. }
	\item \question{$u_n=\frac{1}{n!}$}
	\reponse{ Alors $u_n \geq 0$ pour tout $n \in \N^*$ et $u_n \leq \frac{1}{2^{n-1}}$ pour tout $n \in \N^*$ donc par comparaison à une série géométrique convergente ($q = \frac{1}{2}<1$), la série $\sum u_n$ converge. }
	\item \question{$u_n=\ln\left(\frac{n^2+n+1}{n^2+n-1}\right)$}
	\reponse{ Alors $u_n \geq 0$ pour tout $n \in \N^*$ et $u_n = \ln\left(\frac{n^2+n+2-1}{n^2+n-1}\right) = \ln\left(1+\frac{2}{n^2+n-1}\right) \underset{+\infty}\sim \frac{2}{n^2+n-1}$ donc par comparaison à une série de Riemann convergente ($\alpha = 2>1$), la série $\sum u_n$ converge. }
\end{enumerate}
\fincolonnes{\solution}{3}{1}

\texte{ \textit{Indications: Pour la question $5.$, utiliser la quantité conjuguée, i.e. multiplier le numérateur et le dénominateur par $\sqrt{n+1}+\sqrt{n}$. Pour la question $7.$, utiliser l'inégalité $n!\geq 2^{n-1}$. Pour la question $8.$, utiliser un équivalent de $\ln(1+x)$ en $0$.} }
}
