\uuid{S8kg}
\titre{Différentiabilité d'une fonction de deux variables}
\theme{}
\auteur{Grégoire MENET}
\datecreate{2025-04-16}
\organisation{AMSCC}

\contenu{
	
	\texte{ 
		On donne \( f : \mathbb{R}^2 \to \mathbb{R} \) définie par :
$$
		f(x,y) =
		\begin{cases} 
			\frac{3xy^2+y^3}{x^2 + y^2} & \text{si } (x,y) \neq (0,0), \\[1mm]
			0 & \text{sinon}.
		\end{cases}
$$
	}
	
	\begin{enumerate}
		\item \question{Déterminer la courbe de niveau 0 de \(f\).}
		\indication{Il s'agit de trouver l'ensemble des \((x,y)\) pour lesquels \(f(x,y)=0\). Pour \((x,y)\neq (0,0)\), résoudre l'équation \(\frac{3xy^2+y^3}{x^2+y^2}=0\) afin d'identifier les conditions sur \(x\) et \(y\).}
		\reponse{    On a $f(x, y) = 0 \Leftrightarrow y = 0$ ou $3x + y = 0$.\\
			Ainsi, la courbe de niveau 0 est l'union de la droite $y = 0$ et de la droite $y = -3x$.}
		
		\item \question{Montrer que \( f \) est continue en \((0,0)\).}
		\indication{Utiliser notamment le passage en coordonnées polaires pour étudier la limite de \(f(x,y)\) lorsque \((x,y)\to(0,0)\) et vérifier que cette limite est égale à \(f(0,0)=0\).}
		\reponse{On utilise les coordonnées polaires : $x = r\cos\theta$, $y = r\sin\theta$.\\
			On a : $f(x, y) = \frac{3xy^2 + y^3}{x^2 + y^2} = \frac{r^3 \cos\theta \sin^2\theta \cdot 3 + r^3 \sin^3\theta}{r^2} = r \cdot (3\cos\theta \sin^2\theta + \sin^3\theta)$.\\
			Donc : $$\left|f(x,y)\right|\leq r\left(\left|3\cos\theta \sin^2\theta\right|+ \left|\sin^3\theta\right|\right)\leq4r.$$
			On a $4r\rightarrow 0$ quand $r\rightarrow0$.
			Donc $\lim\limits_{(x, y) \to (0, 0)} f(x, y) = 0 = f(0,0)$, donc $f$ est continue en $(0,0)$.}
		
		\item \question{Vérifier l'existence des dérivées partielles \(\frac{\partial f}{\partial x}(0,0)\) et \(\frac{\partial f}{\partial y}(0,0)\) et les calculer si elles existent.}
		\indication{Utiliser la définition des dérivées partielles, c'est-à-dire calculer les limites \(\lim_{h\to 0}\frac{f(h,0)-f(0,0)}{h}\) et \(\lim_{h\to 0}\frac{f(0,h)-f(0,0)}{h}\) pour obtenir \(\frac{\partial f}{\partial x}(0,0)\) et \(\frac{\partial f}{\partial y}(0,0)\) respectivement.}
		\reponse{On considère les fonctions partielles. On a $f(x,0)=0$ si $x\neq0$ et $f(0,0)=0$, donc $f(x,0)=0$ pour tout $x\in\R$.
			Ainsi $\frac{\partial f}{\partial x}(0,0)=0$.
			De même, on a $f(0,y)=y$ si $y\neq0$ et  $f(0,0)=0$, donc $f(0,y)=y$ pour tout $y\in\R$.
			Ainsi $\frac{\partial f}{\partial y}(0,0)=1$.}
		
		\item \question{La fonction \(f\) est-elle différentiable en \((0,0)\) ?}
		\indication{Étudier la différentiabilité en \((0,0)\) en vérifiant si l'approximation linéaire de \(f\) en ce point représente correctement le comportement de \(f\) près de \((0,0)\), c'est-à-dire si le reste est négligeable devant \(\sqrt{x^2+y^2}\).}
		\reponse{Pour tester la différentiabilité de $f$ en $(0,0)$, on doit examiner si :
			\[
			\lim_{(x, y) \to (0,0)} \frac{f(x,y) - f(0,0) - \frac{\partial f}{\partial x}(0,0)x - \frac{\partial f}{\partial y}(0,0)y}{\sqrt{x^2 + y^2}} = 0.
			\]
			
			On a vu précédemment que :
			\[
			f(0,0) = 0,\quad \frac{\partial f}{\partial x}(0,0) = 0,\quad \frac{\partial f}{\partial y}(0,0) = 1.
			\]
			
			Donc, on étudie la limite :
			\[
			\lim_{(x, y) \to (0,0)} \frac{f(x, y) - y}{\sqrt{x^2 + y^2}}.
			\]
			
			On remplace $f(x,y)$ :
			\[
			f(x, y) = \frac{3xy^2 + y^3}{x^2 + y^2} \Rightarrow f(x,y) - y = \frac{3xy^2 + y^3 - y(x^2 + y^2)}{x^2 + y^2}.
			\]
			
			Simplifions le numérateur :
			\[
			3xy^2 + y^3 - yx^2 - y^3 = 3xy^2 - yx^2.
			\]
			
			Donc :
			\[
			f(x,y) - y = \frac{3xy^2 - x^2 y}{x^2 + y^2}.
			\]
			
			On utilise maintenant les coordonnées polaires : $x = r\cos\theta$, $y = r\sin\theta$.\\
			On a :
			
			\[
			\frac{f(x,y) - y}{\sqrt{x^2 + y^2}} = \frac{r(3\cos\theta \sin^2\theta - \cos^2\theta \sin\theta)}{r} = 3\cos\theta \sin^2\theta - \cos^2\theta \sin\theta.
			\]
			
			La limite de cette expression quand $r\rightarrow0$ dépend de $\theta$ donc le taux d'accroissement n'admet pas de limite en $(0,0)$. La fonction $f$ n'est donc pas différentiable en $(0,0)$. }
		
		\item \question{La fonction \(f\) est-elle de classe \(\mathcal{C}^1\) en \((0,0)\) ?}

		\reponse{Non car $f$ n'est pas différentiable en $(0,0)$.}
	\end{enumerate}
	
}
