\uuid{aRzj}
\titre{Recherche d'un estimateur}
\theme{statistiques, estimateurs, maximum de vraisemblance}
\auteur{}
\datecreate{2023-01-11}
\organisation{AMSCC}
\contenu{


\texte{ Soient $\theta$ un réel strictement positif et $X_1, X_2, \ldots, X_n$ un échantillon dont la loi mère a pour densité la fonction $f$ définie sur $\R$ par : $$f \colon x \mapsto \frac{1}{\theta^2} x e^{-\frac{x}{\theta}} {\textbf{1}}_{]0;+\infty[}(x)$$. }

\begin{enumerate}
	\item \question{ Déterminer un estimateur de $\theta$ issu de la méthode du maximum de vraisemblance. } 
	\reponse{ On définit un échantillon $(X_1,...,X_n)$ de cette loi et on considère une réalisation quelconque $(x_1,...,x_n)$ de cet échantillon. Le support de la loi étant l'intervalle $]0;+\infty[$, on peut supposer que pour tout $i$, $x_i > 0$. 
		
		On exprime maintenant la log vraisemblance de cet échantillon : 
		\begin{align*}
		\ln \mathcal{L}(x_1,...,x_n,\theta) &= \ln \prod_{i=1}^n f(x_i) \\
		&= -2n \ln(\theta) + \sum_{i=1}^{n} \ln(x_i) -\frac{1}{\theta} \sum_{i=1}^{n} x_i \\
		\frac{\partial \ln \mathcal{L}}{\partial \theta}(x_1,...,x_n,\theta) = 0 &\iff \theta = \frac{1}{2 n} \sum_{i=1}^n x_i \\
		\frac{\partial^2 \ln \mathcal{L}}{\partial \theta^2}(x_1,...,x_n,\theta) &= \frac{2n}{\theta^2} - \frac{2}{\theta^3} \sum_{i=1}^n x_i
		\end{align*}
	En posant $\theta_0 = \frac{1}{2 n} \sum_{i=1}^n x_i$, on a :
	\begin{align*}
		\frac{\partial^2 \ln \mathcal{L}}{\partial \theta^2}(x_1,...,x_n,\theta_0) &= \frac{2}{\theta_0} \left(\frac{n}{\theta_0} - \frac{1}{\theta_0} \times (2n)\right) \\
		&= \frac{-2n}{\theta_0^2} < 0
	\end{align*}
Donc la fonction $\theta \mapsto \ln \mathcal{L}(x_1,...,x_n,\theta)$ admet bien un maximum en $\theta = \theta_0$ pour tout $x_1,...,x_n$. On en déduit un estimateur de $\theta$ : 	 \fbox{$\hat{\theta}=\frac{1}{2 n} \sum_{i=1}^n X_i$}. }
	\item \question{ Déterminer le biais de cet estimateur.  \\(Indication : on admet que pour tout $n \in \N$, $\int_0^{+\infty} x^n e^{-x} \, \mathrm{d}x = n!$.)}
	\reponse{ On calcule l'espérance de cette loi : 
\begin{align*}
\E(X_1) &= \int xf(x)dx \\
        &= \int_0^{+\infty} \frac{x^2}{\theta^2}e^{-\frac{x}{\theta}} dx \\
        &= \theta \int_0^{+\infty} {u^2}e^{-u} du \\
        &= 2 \theta 
\end{align*}	
Donc par linéarité, l'espérance de $\hat{\theta}$ est $$\E\left(\hat{\theta}\right) = \frac{1}{2n} \times n \times 2 \theta = \theta$$
donc le biais de $\hat{\theta}$ est $$B(\hat{\theta}) = \E(\hat{\theta} - \theta) = 0$$
 }
\end{enumerate}}
