\titre{ Estimation par intervalle de confiance}
\theme{statistiques}
\auteur{Maxime Nguyen}
\organisation{AMSCC}
\contenu{
\texte{ On s'intéresse au taux de glucose dans une population de 768 patients atteints de diabète. On note $m$ le taux moyen de glucose dans cette population et à $\sigma$ son écart type.  }

\begin{enumerate}
	\item \question{ Donner une estimation de $m$ à l'aide de l'échantillon fourni \href{en suivant ce lien}{lien}, en précisant la taille de l'échantillon donné et l'estimateur choisi. }
	\item \question{ Donner une estimation de $m$ par intervalle de confiance au niveau $95\%$ et $99\%$, en donnant les valeurs numériques des calculs intermédiaires. }
	\item \question{ Le fichier "diabetes.csv" contient les données des 768 patients. Convertir les données pour pouvoir les afficher avec le tableur et donner la valeur réelle de $m$. Quel niveau de confiance avait-on besoin de prendre, a posteriori, pour que l'intervalle de confiance de la question précédente contienne bien la valeur recherché ? }
\end{enumerate}
}