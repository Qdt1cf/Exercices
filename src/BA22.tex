\titre{}

\texte{ Une entreprise utilise des camions pour transporter sa production. Elle dispose de 100 camions. Elle repère sur un échantillon de 25 jours choisis au hasard le nombre de camions en panne. Voici les résultats:
	
\begin{center}
		\begin{tabular}{|c|c|c|c|c|c|c|c|c|c|c|c|c|c|c|c|c|c|c|c|c|c|c|c|c|}
		\hline
5&5&6&4&6&6&8&3&5&5&5&4&3&6&5&6&4&7&6&6&5&4&3&6&5  \\
		\hline
	\end{tabular}
\end{center}
}

\begin{enumerate}
\item \question{ Calculer la moyenne $\bar{x}$ et l'écart-type $\sigma$ du nombre de camions en panne chaque jour pour l'échantillon étudié. }
\reponse{ On trouve $\bar{x} = 5.12$ et $\sigma = 1.21$. (\href{https://stcyrterrenetdefensegouvf-my.sharepoint.com/:x:/g/personal/maxime_nguyen_st-cyr_terre-net_defense_gouv_fr/EeDHiet0b2JOpLX-9rFeI-kBsj_o9NtQ-Qz8hJvy6rlIpw?e=463Dks}{feuille de calcul}).}
\item \question{ A partir des résultats obtenus pour cet échantillon, proposer une estimation ponctuelle sans biais de la moyenne $\mu$ et de l'écart type $s$ du nombre de camions en panne chaque jour pour la population correspondant aux jours ouvrables de l'année. }
\item \question{ Déterminer un intervalle de confiance de la moyenne $\mu$ de la population avec un niveau de confiance $95 \%$. }
\item \question{ Au garage où sont stationnés les camions, le responsable affirme qu'il y a, en moyenne, 4 camions en panne par jour. Un des chauffeurs prétend qu'il y en a 6. 
	
Construire un test avec les hypothèses $\begin{cases}H_0 \colon \mu = 4 \\ H_1 \colon \mu > 4
	\end{cases}$ et un risque de première espèce de $5 \%$ puis proposer une interprétation du résultat. }
\end{enumerate}