\titre{Convergence de la descente de gradient}
\theme{Optimisation}
\auteur{Erwan HILLION}
\organisation{AMSCC}
\contenu{

On va prouver le théorème suivant : 

\medskip

\textbf{Théorème : } Soit $f : \Rr \rightarrow \Rr$ une fonction de classe $\mathcal{C}^2$ telle qu'il existe deux constantes $K,c >0$ vérifiant $c < f''(x) \leq K$ pour tout $x \in \Rr$. On considère la suite $(x_n)_{n \geq 0}$ d\'efinie par $x_0 \in \Rr$ et par $x_{n+1} = x_n - \gamma f'(x_n)$, où le pas $\gamma$ vérifie $0 < \gamma < 2/K$. Alors :
\begin{itemize}
\item La suite $(f(x_n))_{n \ge 0}$ est décroissante.
\item La suite $(x_n)_{n \ge 0}$ converge vers l'unique point critique de $f$.
\end{itemize}

\begin{enumerate} \item Pour $n \geq 0$ fix\'e, on pose $\phi(t) = f(x_n- t f'(x_n))$.
\begin{enumerate}
\item Montrer qu'il existe $\theta \in ]0,t[$ tel que $\phi(t) = \phi(0)+t\phi'(0)+\frac{t^2}{2} \phi''(\theta)$.
\item Montrer que $\phi'(0) = -f'(x_n)^2$ et $\phi''(\theta) \leq f'(x_n)^2 K$.
\item Montrer que $\phi(t) \leq \phi(0)$ pour tout $0 \leq t < 2/K$.
\item En d\'eduire que $f(x_{n+1}) \leq f(x_n)$.
\end{enumerate} 
\item On introduit la fonction $g(x) = x - \gamma f'(x)$.
\begin{enumerate}
\item On pose $M = \sup_{x \in \Rr} |g'(x)|$. Montrer que $M \leq \max\left( |1-\gamma c|, |1- \gamma K|\right)$, puis que $M < 1$.
\item Montrer que pour tout $n \geq 1$, on a $|x_{n+1}-x_n| = |g(x_n)-g(x_{n-1})| < M |x_n-x_{n-1}|$.
\item Montrer que la suite $(x_n)_{n \geq 0}$ est convergente (on pourra considérer la série $\sum x_{n+1}-x_n$).
\item Montrer que la limite $l$ de la suite $(x_n)_{n \geq 0}$. Montrer que $f'(l)=0$.
\item Montrer que $f$ ne possède qu'un seul point critique.
\end{enumerate}
\end{enumerate}}
