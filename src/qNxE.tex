\titre{Simulation de loi : Rademacher, Laplace, Géométrique}
\theme{probabilités}
\auteur{}
\organisation{AMSCC}
\contenu{
%qyPc raccourci
\texte{ On donne ou on rappelle la définition de quelques lois usuelles :
	
	\underline{Définition} : 
	Soit $p \in ]0;1[$ : une variable $X$ suit une loi de Rademacher $\mathcal{R}(p)$ si :
	\begin{itemize}
		\item $\PP(X=1)=p$
		\item $\PP(X=-1)=1-p$
	\end{itemize}
	
	
	\underline{Définition} : 
	Soit $\lambda >0$ : une variable $X$ suit une loi de Laplace $\mathcal{L}(\lambda)$ si elle admet pour densité :
	$$f_X(x) = \frac{\lambda}{2} e^{-\lambda |x|}$$
	
	Soient $X$ et $Y$ deux variables aléatoires indépendantes telles que $X$ suit une loi Rademacher $\mathcal{R}(1/2)$ et $Y$ suit une loi uniforme sur $[0;1]$. Soit $\lambda >0$. On pose $U = \frac{1}{\lambda} X \ln(Y)$.  }

	\reponse{ Soit $\lambda>0$ et $U = \frac{1}{\lambda} X \ln(Y)$ }
	\begin{enumerate}
		\item \question{ Soit $a \in \mathbb{R}$. Calculer $\PP(\ln(Y) \leq a, X=1)$ et $\PP(\ln(Y) \geq a, X=-1)$ }
		\reponse{ Soit $a \in \mathbb{R}$. Par indépendance de $X$ et $Y$, on a $\PP(\ln(Y) \leq a, X=1) = \PP(\ln(Y) \leq a) \times \PP(X=1) = \PP(Y \leq e^a) \times \frac{1}{2}$. Or 
			$\PP(Y \leq t) = 1$ si $t >1$ et $\PP(Y \leq t) = t$ si $0<t<1$ étant donnée la loi suivie par $Y$. Par conséquent, on a  $\PP(\ln(Y) \leq a, X=1) = \begin{cases} \frac{1}{2} \text{ si } a>0 \\ \frac{1}{2} e^a \text{ sinon}\end{cases} $.
			
			De même, $\PP(\ln(Y) \geq a, X=-1) = \begin{cases} 0 \text{ si } a>0 \\ \frac{1}{2} (1-e^a) \text{ sinon}\end{cases} $ }
		\item \question{ Déterminer la fonction de répartition de la variable $U$. }
		\reponse{ Soit $F_U$ la fonction de répartition de la variable $U$. Par définition, pour tout réel $t$, 
			$$F_U(t) = \PP(\frac{1}{\lambda} X\, \ln(Y) \leq t) = \PP(X \, \ln(Y) \leq \lambda t)$$
			
			Par application du théorème des probabilités totales au système d'événements $\{(X=1), (X=-1)\}$, 
			$$F_U(t) = \PP(X=1,Y \leq e^{\lambda t}) + \PP(X=-1,Y \geq e^{-\lambda t})$$
			D'après le calcul précédent, on obtient 
			$$F_U(t) =   \begin{cases} 1-\frac{1}{2} e^{-\lambda t} \text{ si } t>0 \\ \frac{1}{2} e^{\lambda t} \text{ sinon}\end{cases} $$ }
		\item \question{ En déduire que $U$ suit une loi de Laplace $\mathcal{L}(\lambda)$. }
		\reponse{ On dérive la fonction de répartition pour obtenir la densité : $F_U'(t) = \begin{cases} \frac{1}{2} \lambda e^{-\lambda t} \text{ si } t>0 \\ \frac{1}{2} \lambda e^{\lambda t} \text{ sinon}\end{cases} = \frac{1}{2} \lambda e^{-\lambda |t|}$. On reconnaît la fonction densité d'une loi de Laplace de paramètre $\lambda$. }
		\item A partir de la fonction \texttt{rand()} qui permet de simuler une loi uniforme sur $[0;1]$ et en utilisant les résultats des questions précédentes, écrire un programme qui permet de simuler une loi de Laplace 
	\end{enumerate}
}
