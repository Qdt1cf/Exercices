\uuid{H30c2}
\titre{Test d'adéquation}
\theme{statistiques}
\auteur{}
\organisation{AMSCC}
\contenu{


\texte{ En lançant successivement 60 fois un dé, un joueur obtient les résultats suivants :
	
\begin{center}
	\begin{tabular}{|c|c|c|c|c|c|c|}
	\hline Faces $x_i$ & 1 & 2 & 3 & 4 & 5 & 6 \\
	\hline Effectifs $n_i$ & 15 & 7 & 4 & 11 & 6 & 17 \\
	\hline
\end{tabular}
\end{center} }

\question{ Le dé est-il truqué? }

\reponse{On réalise un test d'adéquation du $\chi^2$ pour répondre à la question. Un dé non truqué devrait produire une distribution uniforme des effectifs de chaque face. Cela donne le tableau des effectifs théoriques : 

\begin{center}
	\begin{tabular}{|c|c|c|c|c|c|c|}
		\hline Faces $x_i$ & 1 & 2 & 3 & 4 & 5 & 6 \\
		\hline Effectifs $n_i$ & 15 & 7 & 4 & 11 & 6 & 17 \\
		\hline Effectifs $e_i$ & 10 & 10 & 10 & 10 & 10 & 10 \\
		\hline
	\end{tabular}
\end{center} 

On pose pour variable de décision $$Q=\sum_{i=1}^6 \frac{(O_i-e_i)^2}{e_i}$$ où $O_i$ est la variable aléatoire donnant l'effectif de la $i$-ème classe pour l'échantillon de taille $60$. D'après le cours, $Q$ suit une loi $\chi^2(5)$. 

D'après la table de $\chi^2(5)$, l'intervalle de rejet pour $\alpha=5\%$ est $[11.07~;~+\infty[$.

On trouve $Q_{obs} = 13.6 \in W$ (\href{https://stcyrterrenetdefensegouvf-my.sharepoint.com/:x:/g/personal/maxime_nguyen_st-cyr_terre-net_defense_gouv_fr/EfmljPG_LipCiIlDwESb5DsBtQvsb9gziaK95ni0MXfezg?e=mUpc0c}{fichier tableur})donc on peut rejeter l'hypothèse $H_0$ avec un risque de première espèce de $5\%$ : on peut suspecter que le dé est truqué. 
} 
}