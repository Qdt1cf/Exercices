\titre{Courbes de niveaux}
\theme{fonctions de plusieurs variables}
\auteur{}
\organisation{AMSCC}

\texte{ Déterminer les courbes de niveaux des fonctions suivantes et esquisser rapidement une représentation graphique d'un ensemble de celles-ci. }
	\begin{enumerate}
		\item \question{ $f(x,y) = x+y-1$ }
		\reponse{Soit $k \in \mathbb{R}$ : $f(x,y) = k \iff y = -x+k+1$. Pour tout $k \in \R$, les lignes de niveau $k$ sont des droites de coefficient directeur $-1$. 
		
\begin{center}
	\begin{tikzpicture}
\begin{axis}[
title={$f(x,y)=x+y-1$},
enlarge x limits,
view={0}{90},
xlabel=$x$, ylabel=$y$,
small,
axis equal,
axis x line=center,
axis y line=center,
%grid=major,
]
\addplot3[
domain=-3:3,
domain y=-3:3,
contour gnuplot={number=14},
thick,
]
{x+y-1};
\end{axis}
\end{tikzpicture}
\end{center}
}
		\item \question{ $f(x,y) = e^{y-x^2}$ }
		\reponse{Soit $k \in \mathbb{R}$ : $f(x,y) = k \Rightarrow k>0$. Pour tout $k>0$, $f(x,y) = k \iff y = x^2 + \ln(k)$. Les lignes de niveau $k>0$ sont des paraboles, vides si $k \leq 0$. 
\begin{center}
	\begin{tikzpicture}
	\begin{axis}[
	%title={$f(x,y)=e^{y-x^2}$},
	enlarge x limits,
	view={0}{90},
	xlabel=$x$, ylabel=$y$,
	small,
	axis equal,
	axis x line=center,
	axis y line=center,
	%grid=major,
	]
	\addplot3[
	domain=-3:3,
	domain y=-3:3,
	contour gnuplot={levels={0.2,0.4,0.6, 0.8,1,2,3,4,5,6,7,8,9,10,11,12}},
	samples=40,
	thick,
	]
	{exp(y-x^2)};
	\end{axis}
	\end{tikzpicture}
\end{center}		
	
}
		\item \question{ $f(x,y) = \ln(x+2y)$ }
		\reponse{Soit $k \in \mathbb{R}$ : $f(x,y) = k \iff y = -\frac{x+e^{k}}{2} $. Les lignes de niveau $k$ sont des droites de coefficient directeur $-\frac{1}{2}$. 

\begin{center}
	\begin{tikzpicture}
	\begin{axis}[
	%title={$f(x,y)=\ln(x+2y)$},
	enlarge x limits,
	view={0}{90},
	xlabel=$x$, ylabel=$y$,
	small,
	axis equal,
	axis x line=center,
	axis y line=center,
	%grid=major,
	]
	\addplot3[
	domain=-5:5,
	domain y=-5:5,
	contour gnuplot={levels={-1,0,1,1.5,2}},
	samples=20,
	thick,
	]
	{ln(x+2*y)};
	\end{axis}
	\end{tikzpicture}
\end{center}		
	
}
	%	\item $f(x,y,z) = \frac{\ln(x^2+1)}{yz}$
	\end{enumerate}