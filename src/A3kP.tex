\uuid{A3kP}
\titre{ Répartition des ressources militaires}
\niveau{L2}
\module{Probabilités}
\chapitre{Couples de variables aléatoires discrètes}
\sousChapitre{Loi conjointe, marginales, covariance}
\theme{Loi marginale, indépendance, espérance, covariance, dépendance}
\auteur{}
\datecreate{2025-09-16}
\organisation{}
\difficulte{3}
\contenu{
	\texte{
		Un bataillon militaire dispose de deux types de véhicules : des véhicules blindés (notés \( X \)) et des drones de reconnaissance (notés \( Y \)). Lorsque l'on considère un véhicule au hasard, les probabilités conjointes sont données par le tableau suivant :
		
\begin{center}
	\begin{tabular}{c c c c c}
	\hline
	\(X \setminus Y\) & 0 & 1 & 2 & \(\prob(X=x)\) \\
	\hline
	0 & 0.1 & 0.1 & 0.1 &  \\
	1 & 0.1 & 0.2 & 0.1 &  \\
	2 & 0.0 & 0.1 & 0.2 &  \\
	\hline
	\(\prob(Y=y)\) &  &  &  &  \\
	\hline
\end{tabular}
\end{center}

	}
	\begin{enumerate}
		\item   \question{Déterminer les lois marginales du couple $(X,Y)$.}
		
		\item   \question{Les variables \( X \) et \( Y \) sont-elles indépendantes ?}
		\indication{Vérifier si \(\PP(X=x, Y=y) = \PP(X=x)\PP(Y=y)\) pour toutes les valeurs de \(x\) et \(y\).}
		
		\item   \question{Calculer l'espérance \( \E[X] \), \( \E[Y] \) et l'espérance du nombre total de véhicules (blindés + drones) : \( \E[X + Y] \).}
		\indication{Utiliser la linéarité de l'espérance : \(\E[X + Y] = \E[X] + \E[Y]\).}
		
		%\item   \question{Calculer la covariance \( \text{Cov}(X, Y) =  \E[XY] - \E[X]E[Y] \). Quel est le signe du résultat ? Interpréter. }
		\item Calculer la probabilité que le nombre de drones soit strictement supérieur à celui des blindés. 
	\end{enumerate}
}
