\titre{Méthode itérative d'inversion d'une matrice}
\theme{}
\auteur{}
\organisation{AMSCC}
\contenu{

\texte{ Soient $n \in \mathbb{N}$ tel que $n \geq 3$ et $b \in \mathbb{R}^n$, de composantes $(b_1, \ldots, b_n)$. On cherche $x \in \mathbb{R}^n$, de composantes $(x_1, \ldots, x_n)$, solution de :

$$
\begin{cases}	
	4x_1 + x_2 &= b_1, \\
	x_{i-1} + 4x_i + x_{i+1} &= b_i, \quad \forall i \in [\![ 2, n - 1 ]\!] \\	
	x_{n-1} + 4x_n &= b_n.	
\end{cases}
$$ }

\begin{enumerate}
	\item \question{ Montrer que le syst\`eme ci-dessus peut s'\'ecrire sous la forme $Ax=b$ avec une matrice $A$ que l'on donnera pour $n=4$. %Donner un code Python qui permet de générer la matrice $A$ pour $n$ quelconque. 
	}
	
	\item On suppose, dans cette question uniquement, que $b = 0$.
	
	\begin{enumerate}
		\item \question{  Montrer que : $\forall i \in [\![1, n]\!]\,, \quad 4 |x_i| \leq 2 \|x\|_\infty$. }
		\item \question{ En déduire que $x = 0$. }
	\end{enumerate}
	
	
	\item \question{ Montrer que dans le cas d’un second membre quelconque $b$, il existe une unique $x \in \mathbb{R}^n$ solution du système linéaire. }
	
	\item \texte{ Afin de résoudre le système, on considère la méthode itérative suivante : $x^{(0)} = 0 \in \mathbb{R}^n$ et
	
	$$	
	\begin{cases}	
		x^{(k+1)}_1 = \alpha x^{(k)}_1 + \frac{\alpha - 1}{4} (x^{(k)}_2 - b_1), \\	
		x^{(k+1)}_i = \alpha x^{(k)}_i + \frac{\alpha - 1}{4} (x^{(k)}_{i-1} + x^{(k)}_{i+1} - b_i), & \forall i \in [\![ 2, n - 1 ]\!] \\	
		x^{(k+1)}_n = \alpha x^{(k)}_n + \frac{\alpha - 1}{4} (x^{(k)}_{n-1} - b_n).	
	\end{cases}
	$$
	
	avec pour paramètre $\alpha \in \mathbb{R}$.  }
	
	\begin{enumerate}
		
		\item \question{ Montrer que pour tout $\alpha \in \mathbb{R}$, on a
		
		$$\|x^{(k+1)} - x\|_\infty \leq \left(|\alpha| + \frac{|\alpha - 1|}{2}\right) \|x^{(k)} - x\|_\infty.$$ }
		
		\item \question{ Trouver $\alpha_{\text{min}}, \alpha_{\text{max}} \in \mathbb{R}$, tels que $\alpha \in ]\alpha_{\text{min}}, \alpha_{\text{max}}[$ si et seulement si $|\alpha| + \frac{|\alpha - 1|}{2} < 1$. }
		
		\item \question{ Montrer que la méthode itérative converge vers $x$ pour $\alpha \in ]\alpha_{\text{min}}, \alpha_{\text{max}}[$. }
		
		\item\question{  Trouver $\alpha_0 \in \mathbb{R}$ qui minimise la quantité $|\alpha| + \frac{|\alpha - 1|}{2}.$ Que peut-on d\'eduire de la convergence du syst\`eme pour ce $\alpha_0$ ? Quelle méthode du cours reconnaissez-vous ? }
		
		
		
	\end{enumerate}
	
\end{enumerate}
}