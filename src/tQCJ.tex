\titre{Déterminant nul}
\theme{calcul déterminant}
\auteur{}
\organisation{AMSCC}
\contenu{




\question{ Sans calculs (ou presque), justifier que les déterminants suivants sont nuls. 
$$
\Delta_1=\left|\begin{array}{cccc}
	2 & 3 & 0 & 7 \\
	16 & 25 & 31 & 12 \\
	0 & 0 & 0 & 0 \\
	1 & -11 & 1 & 0
\end{array}\right|, \quad \Delta_2=\left|\begin{array}{cccc}
	1 & 6 & 7 & 14 \\
	-1 & 1 & 0 & 0 \\
	2 & 4 & 6 & 4 \\
	0 & 1 & 1 & 2
\end{array}\right|
$$ }

\reponse{ La matrice comporte une ligne de zéro donc par propriété du déterminant : $$
	\Delta_1=\left|\begin{array}{cccc}
		2 & 3 & 0 & 7 \\
		16 & 25 & 31 & 12 \\
		0 & 0 & 0 & 0 \\
		1 & -11 & 1 & 0
	\end{array}\right|=0
	$$
	Pour le calcul de $\Delta_2$, on remarque que $\mathcal{C}_3=\mathcal{C}_1+\mathcal{C}_2$, donc :
	$$
	\Delta_2=\left|\begin{array}{cccc}
		c_1 & c_2 & c_3 & c_4 \\
		1 & 6 & 7 & 14 \\
		-1 & 1 & 0 & 0 \\
		2 & 4 & 6 & 4 \\
		0 & 1 & 1 & 2
	\end{array}\right|=0
	$$ }}
