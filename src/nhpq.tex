\uuid{nhpq}
\titre{ Estimation par intervalle de confiance}
\theme{estimateurs, intervalle de confiance}
\auteur{Maxime Nguyen}
\organisation{AMSCC}
\contenu{


\texte{ Dans une base militaire, un nouveau type de radio est en cours de test pour évaluer sa fiabilité en conditions opérationnelles. Un échantillon de 150 radios a été testé durant un exercice, et il a été constaté que 135 de ces radios ont fonctionné sans défaillance tout au long de l'exercice. }

\begin{enumerate}
	\item \question{Donner une estimation de la proportion de ces nouvelles radios fonctionnant sans défaillance, en précisant l'estimateur utilisé et son biais. }
	\reponse{On utilise l'estimateur de fréquence empirique $F = \frac{1}{150}\sum_{i=1}^{150} X_i$ avec $X_i \sim \mathcal{B}(p)$, sans biais pour estimer la proportion $p$ de radios sans défaillance : sa réalisation ici est $p_{obs} = \frac{130}{150} = 0{,}90$. }
	
	\item \question{Donner cette estimation à l'aide d'un intervalle de confiance à 90\%, 95\% et 99\%.}
	\reponse{On utilise la formule du cours : 
		$$I_{conf}(F(\omega))=\left[f_{obs}-u_{\alpha/2} \sqrt{\frac{f_{obs}(1-f_{obs})}{n}} ~;~ f_{obs} + u_{\alpha/2} \sqrt{\frac{f_{obs}(1-f_{obs})}{n}} \right]$$
	avec $\alpha = 0.1$ : $I_{conf} = [0,859709479 ; 0,940290521]$
	
		avec $\alpha = 0.05$ : $I_{conf} = [0,851990883 ; 0,948009117]$
		
		avec $\alpha = 0.01$ : $I_{conf} = [0,836905325 ; 0,963094675]$
}
	
	\item \question{Quelle taille d'échantillon devrait-on choisir pour que l'amplitude de l'intervalle de confiance ne dépasse pas $0.01$ avec une erreur de première espèce de $5\%$ ?}
	\reponse{On utilise la formule simplifiée du cours : 
		$$I_{conf}(F(\omega))=\left[ f_{obs}-u_{\alpha/2} \frac{1}{2 \sqrt{n}} ~;~ f_{obs} + u_{\alpha/2} \frac{1}{2 \sqrt{n}} \right]$$
	et on cherche $n$ tel que $u_{\alpha/2} \frac{1}{\sqrt{n}} \leq 0.01 \iff \sqrt{n} \geq \frac{1.96}{0.01}$ soit $n \geq 38415$. }
\end{enumerate}
}