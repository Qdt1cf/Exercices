\titre{ Estimation par intervalle de confiance}
\theme{statistiques}
\auteur{Maxime Nguyen}
\organisation{AMSCC}
\contenu{


\texte{ Dans une base militaire, un nouveau type de radio est en cours de test pour évaluer sa fiabilité en conditions opérationnelles. Un échantillon de 150 radios a été testé durant un exercice, et il a été constaté que 135 de ces radios ont fonctionné sans défaillance tout au long de l'exercice. }

\begin{enumerate}
	\item \question{Donner une estimation de la proportion de ces nouvelles radios fonctionnant sans défaillance, en précisant l'estimateur utilisé et son biais. }
	\reponse{}
	
	\item \question{Donner cette estimation à l'aide d'un intervalle de confiance à 90\%, 95\% et 99\%.}
	\reponse{}
	
	\item \question{Quelle taille d'échantillon devrait-on choisir pour que l'amplitude de l'intervalle de confiance ne dépasse pas $0.01$ avec une erreur de première espèce de $5\%$ ?}
	\reponse{}
\end{enumerate}
}