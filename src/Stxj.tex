\uuid{Stxj}
\chapitre{Série entière}
\sousChapitre{Développement en série entière}
\titre{Développement en série entière}
\theme{séries entières}
\auteur{}
\datecreate{2023-06-05}
\organisation{AMSCC}
\contenu{


\question{ 	Développer en série entière la fonction de la variable réelle suivante : $$f \colon x \mapsto \ln(x^2-5x+6)$$  }

\reponse{Pour tout réel $x$, $x^2 -5x+6=(x-2)(x-3)$ et donc si $x < 2$, $x^2-5x+6>0$. Pour $x\in ]-2,2[$,
	
	\begin{center}
		$\ln(x^2-5x+6)=\ln(2-x)+\ln(3-x)=\ln(6)+\ln\left(1-\frac{x}{2}\right)+\ln\left(1-\frac{x}{3}\right)$,
	\end{center}
	
	et puisque pour $x$ dans $]-2,2[$, $\frac{x}{2}$ et $\frac{x}{3}$ sont dans $]-1,1[$, 
	
	\begin{center}
		$\ln(x^2-5x+6)=\ln(6)-\sum_{n=1}^{+\infty}\left(\frac{1}{2^n}+\frac {1}{3^n}\right)\frac{x^n}{n}$,
	\end{center}
	
	En résumé, la fonction $f$ est développable en série entière et on vérifie avec la règle de d'Alembert que son rayon de convergence est $R = 2$.}}
