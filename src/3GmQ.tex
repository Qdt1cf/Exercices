\titre{Contrôle qualité}
\theme{statistiques}
\auteur{}
\organisation{AMSCC}

\texte{ Une entreprise fabrique  des pièces en sous-traitance. Au sein d'une démarche qualité,  toutes  les  machines  ont  été  systématiquement révisées  et  on  a  défini  une  nouvelle  organisation  dans  l'atelier  :  les  tâches  de  contrôle  sont  réparties  à  chaque  étape  du  processus  de  fabrication et le taux de pièces défectueuses est tombé à 1\%. 

Quelques  mois  plus  tard,  une  opération  de  contrôle  est  effectuée  pour  vérifier  si  la  norme  de  1\%  (hypothèse  $H_0$)  de  pièces  défectueuses reste valable. Sur les 5 000 pièces contrôlées 100 s'avèrent défectueuses, soit 2\% (hypothèse $H_1$). 

Mme de Mainard, chef d'entreprise, décide que si l'hypothèse nulle est vérifiée, elle ne modifiera plus son processus de production (décision D0) et au contraire, si c'est l'hypothèse alternative, elle entreprendra une action de sensibilisation des salariés de cet atelier au problème de la qualité (décision D1). 

Pour choisir entre ces deux hypothèses, elle tire un échantillon de 1 500 pièces.  }
\begin{enumerate}
	
	\item \question{  Si la chef d'entreprise se fixe un risque de 1\% d'entreprendre une action de sensibilisation des salariés à tort, quel sera le taux critique de pièces défectueuses qui fera prendre une décision ? }
	
	\item \question{  Si dans l'échantillon prélevé, le nombre de pièces défectueuses est 18, quelle sera la décision de Mme Granzer ? }
	\item   \question{ Calculer alors le risque de l'acheteur, c'est-à-dire ne pas modifier le processus de production alors qu'on le devrait. Comment s'appelle ce risque ? }
	
\end{enumerate}