\chapitre{Polynôme, fraction rationnelle}
\sousChapitre{Fraction rationnelle}
\uuid{yNs9}
\titre{Division euclidienne et fractions rationnelles}
\theme{fractions rationnelles}
\auteur{}
\datecreate{2023-01-24}
\organisation{AMSCC}
\contenu{
	
	\texte{ Soit les polynômes à coefficients réels $A(X)=X^3+2 X-5$ et $B(X)=X-1$.  }

\begin{enumerate}
	\item \question{ Effectuer la division euclidienne du polynôme $A$ par le polynôme $B$. }
	\reponse{ On trouve : 
$$
A(X)=X^3+2 X-5=\underbrace{(X-1)}_{B(X)} \left(X^2+X+3\right)-2
$$	
 }
	\item \question{  En déduire les coefficients $a, b, c$ et $d$ de la décomposition en éléments simples de $$\frac{A(X)}{B(X)}=\frac{X^3+2 X-5}{X-1}=a X^2+b  X+c+\frac{d}{X-1}$$ }.
\reponse{ 
	On en déduit :
	$$
	\begin{aligned}
	F(X) & =\frac{A(X)}{B(X)}=\frac{X^3+2 X-5}{X-1}=\frac{(X-1) \left(X^2+X+3\right)-2}{X-1} \\
	& =\frac{(X-1) \cdot\left(X^2+X+3\right)}{X-1}-\frac{2}{X-1}=X^2+X+3-\frac{2}{X-1}
	\end{aligned}
	$$ }
\end{enumerate}
}
