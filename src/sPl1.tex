\uuid{sPl1}
\titre{Inversibilité de matrices}
\theme{calcul matriciel}
\auteur{}
\organisation{AMSCC}
\contenu{


\texte{ Soit la matrice $A=\left(\begin{array}{ccc}3 & 2 & -2 \\ -1 & 0 & 1 \\ 1 & 1 & 0\end{array}\right)$.  }

\question{ Calculer $A^3-3A^2+3A$. En déduire que $A$ est inversible et déterminer $A^{-1}$. }


\reponse{ $$
	\begin{aligned}
		& A=\left(\begin{array}{ccc}
			3 & 2 & -2 \\
			-1 & 0 & 1 \\
			1 & 1 & 0
		\end{array}\right) \\
		& A^2=A \cdot A=\left(\begin{array}{ccc}
			3 & 2 & -2 \\
			-1 & 0 & 1 \\
			1 & 1 & 0
		\end{array}\right) \cdot\left(\begin{array}{ccc}
			3 & 2 & -2 \\
			-1 & 0 & 1 \\
			1 & 1 & 0
		\end{array}\right)=\left(\begin{array}{ccc}
			5 & 4 & -4 \\
			-2 & -1 & 2 \\
			2 & 2 & -1
		\end{array}\right) \\
		& A^3=A^2 \cdot A=\left(\begin{array}{ccc}
			5 & 4 & -4 \\
			-2 & -1 & 2 \\
			2 & 2 & -1
		\end{array}\right) \cdot\left(\begin{array}{ccc}
			3 & 2 & -2 \\
			-1 & 0 & 1 \\
			1 & 1 & 0
		\end{array}\right)=\left(\begin{array}{ccc}
			7 & 6 & -6 \\
			-3 & -2 & 3 \\
			3 & 3 & -2
		\end{array}\right) \\
		& A^3-3 \cdot A^2+3 \cdot A=\left(\begin{array}{ccc}
			7 & 6 & -6 \\
			-3 & -2 & 3 \\
			3 & 3 & -2
		\end{array}\right)-3\left(\begin{array}{ccc}
			5 & 4 & -4 \\
			-2 & -1 & 2 \\
			2 & 2 & -1
		\end{array}\right)+3\left(\begin{array}{ccc}
			3 & 2 & -2 \\
			-1 & 0 & 1 \\
			1 & 1 & 0
		\end{array}\right)=\left(\begin{array}{lll}
			1 & 0 & 0 \\
			0 & 1 & 0 \\
			0 & 0 & 1
		\end{array}\right)=I_3
	\end{aligned}
	$$
	On en déduit :
	$$
	\begin{gathered}
		A^3-3 \cdot A^2+3 \cdot A=I_3 \Leftrightarrow A \cdot \underbrace{\left(A^2-3 \cdot A+3 \cdot I_3\right)}_{=A^{-1}}=I_3 \\
		\Rightarrow A^{-1}=A^2-3 \cdot A+3 \cdot I_3
	\end{gathered}
	$$ }}
