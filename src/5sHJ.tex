\titre{Gain au loto}
\theme{Probabilités}
\auteur{}
\organisation{AMSCC}

\contenu{ 
    \texte{ Le loto est un jeu consistant à choisir six numéros différents compris entre 1 et 49.
Au dernier tirage, il fallait jouer les numéros: $4$, $8$, $21$, $23$, $42$ et $49$. Le gain était de $\nombre{700000}$ euros avec les six bons numéros, de $\nombre{1100}$ euros avec cinq bons numéros, de $20$ euros avec quatre bons numéros et de $1$ euro avec trois bons numéros. 
    }

    \question{Pour un jeu simple, calculer l'espérance de gain.}
\reponse{ Soit $X$ le nombre de bons numéros trouvés par le joueur. Alors $X$ suit une loi hypergéométrique de paramètres: $\mathcal{H}(6,\frac{1}{46},49)$. \\
Soit $Y$ le gain du joueur. Alors on a
\begin{center}
\begin{tabular}{|c|c|c|c|c|c|c|c|}
 \hline
 $\omega$ & 0 & 0 & 0 & 1 & 20 & 1100 & $700\ 000$ \\
 \hline
 $\prob(\omega)$ & $\prob(X=0)$ & $\prob(X=1)$ & $\prob(X=2)$ & $\prob(X=3)$ & $\prob(X=4)$ & $\prob(X=5)$ & $\prob(X=6)$ \\
 \hline
\end{tabular}
\end{center}
On a donc
\[ \E(Y)=\prob(X=3)+20\times \prob(X=4)+\nombre{1100}\times \prob(X=5) + {700000}\times \prob(X=6).\]
Or
\begin{align*}
 & \prob(X=3)=\frac{\binom{6}{3}\binom{43}{3}}{\binom{49}{6}}= \frac{246\ 820}{13\ 983\ 816} \\
 & \prob(X=4)=\frac{\binom{6}{4}\binom{43}{2}}{\binom{49}{6}}= \frac{13\ 545}{13\ 983\ 816} \\
 & \p(robX=5)=\frac{\binom{6}{5}\binom{6}{1}}{\binom{49}{6}}= \frac{258}{13\ 983\ 816} \\
 & \prob(X=6)=\frac{\binom{6}{6}\binom{43}{0}}{\binom{49}{6}}= \frac{1}{13\ 983\ 816} 
\end{align*}
donc 
\[ \E(Y)=\frac{1}{13\ 983\ 816} (246\ 820+ 20\times 13\ 545+ 1\ 100\times 258+700\ 000)
 =\frac{1\ 501 520}{13\ 983\ 816}
 \simeq 0.1074
\]
L'espérance du gain du joueur (sans compter sa mise de départ) est d'environ $0.11$ euros.
}