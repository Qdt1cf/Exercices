\titre{Consistance d'un $\theta$-schéma}
\theme{analyse numérique}
\auteur{}
\organisation{AMSCC}



\texte{ Pour une fonction $f$ continue par morceaux sur $[a;b]$, on pose $\theta \in [0;1]$ on considère l'approximation :
$$\int_a^b f(s)ds \approx (b-a)(\theta f(a) + (1-\theta)f(b))$$ }
\begin{enumerate}
	\item \question{ \'A partir de ce choix d'approximation, construire un schéma de résolution d'une EDO $y'(t)=f(t,y)$. }
	\item \question{ Reconnaître des schémas usuels pour les valeurs $\theta \in \{0,\frac{1}{2},1\}$. }
	\item \question{ Montrer que le schéma est consistant d'ordre 1 si $\theta \neq \frac12$. }
\end{enumerate}