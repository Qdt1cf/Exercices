\uuid{btWx}
\titre{Consistance d'un $\theta$-schéma}
\theme{analyse numérique}
\auteur{}
\organisation{AMSCC}
\contenu{



\texte{ Pour une fonction $f$ continue par morceaux sur $[a;b]$, on pose $\theta \in [0;1]$ on considère l'approximation :
$$\int_a^b f(s)ds \approx (b-a)(\theta f(a) + (1-\theta)f(b))$$ }
\begin{enumerate}
<<<<<<< HEAD
	\item \question{ \'A partir de ce choix d'approximation, construire un schéma de résolution d'une EDO $y'(t)=f(t,y)$. }
=======
	\item \question{ A partir de ce choix d'approximation, construire un schéma de résolution d'une EDO $y'(t)=f(t,y)$. }
	\reponse{ On a $y(t_{n+1}) = y(t_n) + \int_{t_n}^{t_{n+1}} f\left(s,y(s)\right) ds$ d'où le schéma défini par :
		$$y_{n+1} = y_n + h \left( \theta f(t_n,y(t_n)) + (1-\theta) f(t_{n+1}, y(t_{n+1}))\right)$$
	}
>>>>>>> 375a5e41094b13767807c5e246dbdce0810354c6
	\item \question{ Reconnaître des schémas usuels pour les valeurs $\theta \in \{0,\frac{1}{2},1\}$. }
	\reponse{ 
		\begin{itemize}
			\item 	Si $\theta =0$ : méthode d'Euler implicite (rectangle à droite) ;
			\item   Si $\theta = 1$ : méthode d'Euler explicite (rectangle à gauche) ;
			\item   Si $\theta = 1/2$ : méthode de Carnk Nikolson.
		\end{itemize}
	  }
	\item \question{ Montrer que le schéma est consistant d'ordre 1 si $\theta \neq \frac12$. }
	\reponse{ On calcule l'erreur de consistance : 
\begin{align*}
  h\, e_n(h) &= \underbrace{y(t_{n+1}) - y(t_n)}_{\text{développement de Taylor}} - h \theta \underbrace{f(t_n, y(t_n))}_{y'(t_n)} - h(1-\theta) \underbrace{f(t_{n+1}, y(t_{n+1}))}_{y'(t_{n+1})} \\
             &= h y'(t_n) + \frac{h^2}{2}y''(t_n) + \frac{h^3}{6} y'''(c_n) - h \theta y'(t_n) - h(1-\theta) \left( y'(t_n) + hy''(t_n) + \frac{h^2}{2}y'''(c'_n)  \right) \\
             &= h y'(t_n)(1-\theta - (1-\theta)) + h^2 \left( \frac{y''(t_n)}{2} - (1-\theta)y''(t_n)  \right) + h^3 \underbrace{\left( ... \right)}_{\text{borné}} \\
\end{align*}	
Donc $$|e_n(h)| \leq h \left| \frac{y''(t_n)}{2} - (1-\theta)y''(t_n)  \right| + h^2 M$$
ce qui permet de conclure que l'ordre de consistance est $1$ si $\theta \neq \frac{1}{2}$, l'ordre de consistance est $2$ si $\theta = \frac{1}{2}$.
 }
\end{enumerate}}
