\uuid{MZkM}
\chapitre{Autre}
\niveau{L3}
\module{Topologie}
\sousChapitre{Autre}
\titre{ Classification linéaire par un perceptron }
\theme{réseaux de neurones}
\auteur{ Maxime NGUYEN }
\datecreate{2024-11-17}
\organisation{ AMSCC }

\contenu{

\question{ Décrire ce que permet de réaliser ce réseau de neurones :

\begin{center}
	\begin{tikzpicture}[scale=1.5]
		\def\layersep{2cm}
		\tikzstyle{every pin edge}=[thick]
		\tikzstyle{neuron}=[circle,fill=black!25,minimum size=12pt,inner sep=0pt]
		\tikzstyle{entree}=[];
		\tikzstyle{input neuron}=[neuron, fill=green!50];
		\tikzstyle{output neuron}=[neuron, fill=red!50];
		\tikzstyle{hidden neuron}=[neuron, fill=blue!50];
		\tikzstyle{annot} = [text width=4em, text centered]
		
		% Entree
		\node[entree,blue] (E) at (-\layersep,-2.5) {$x$};
		
		% Premiere couche
		\node[input neuron] (I-1) at (0,-1) {};
		\node[input neuron] (I-2) at (0,-2) {};
		\node[input neuron] (I-3) at (0,-3) {};
		\node[input neuron] (I-4) at (0,-4) {};
		
		\node[above right=0.8ex,scale=0.7] at (I-1) {$H$};
		\node[above right=0.8ex,scale=0.7] at (I-2) {$H$};
		\node[below right=0.8ex,scale=0.7] at (I-3) {$H$};
		\node[below right=0.8ex,scale=0.7] at (I-4) {$H$};
		
		%Seconde couche et sortie
		\node[output neuron] (O) at (\layersep,-2.5 cm) {};
		\node[below right=0.8ex,scale=0.7] at (O) {id};
		
		% Arrete et poids
		\path[thick] (E) edge node[pos=0.8,above,scale=0.7]{$1$} (I-1) ;
		\draw[-o,thick] (I-1) to node[midway,below right,scale=0.7]{$-1$} ++ (-120:0.6);
		
		\path[thick] (E) edge node[pos=0.8,above,scale=0.7]{$-1/3$} (I-2);
		\draw[-o,thick] (I-2) to node[midway,below right,scale=0.7]{$1$} ++ (-120:0.6);
		
		\path[thick] (E) edge node[pos=0.8,above,scale=0.7]{$-1/5$} (I-3) ;
		\draw[-o,thick] (I-3) to node[midway,below right,scale=0.7]{$1$} ++ (-120:0.6);
		
		\path[thick] (E) edge node[pos=0.8,below left,scale=0.7]{$1/4$} (I-4);
		\draw[-o,thick] (I-4) to node[midway,below right,scale=0.7]{$-1$} ++ (-120:0.6);
		
		\path[thick] (I-1) edge node[pos=0.8,above,scale=0.7]{$1$} (O);
		\path[thick] (I-2) edge node[pos=0.8,above,scale=0.7]{$1$}(O);
		%\draw[-o,thick] (O) to node[midway,right,scale=0.7]{$-3$} ++ (120:0.8) ;
		
		\path[thick] (I-3) edge node[pos=0.8,above,scale=0.7]{$2$} (O);
		\path[thick] (I-4) edge node[pos=0.8,above,scale=0.7]{$2$}(O);
		\draw[-o,thick] (O) to node[midway,below right,scale=0.7]{$-3$} ++ (-120:0.8) ;
		% Sortie
		\draw[->,thick] (O)-- ++(1,0) node[right,blue]{$F(x)$};
		
	\end{tikzpicture} 
\end{center}  }
}