\chapitre{Dénombrement}
\sousChapitre{Autre}
\uuid{2yzb}
\titre{Factorielles}
\theme{dénombrement}
\auteur{}
\datecreate{2023-01-24}
\organisation{AMSCC}
\contenu{



\texte{ 	Les questions suivantes sont indépendantes : }
	\begin{enumerate}
		%		\item Soit $n \in \N$, $n \geq 1$. Simplifier l'expression $\frac{(n+2)!}{n!}$.
		%		\item Sans faire de longs calculs, démontrer de tête que $6! \times 7! = 10!$.
		\item \question{ Soit $n$ un entier naturel, $n \geq 2$. Simplifier la fraction $\dfrac{(n+1)!}{(n-1)!}$. }
		\item \question{ Démontrer que pour tout entier naturel $k \in \mathbb{N}$, $(k+1)! - k! = k \times k!$ }
		\item \question{ Ecrire ce produit à l'aide de factorielles : $100 \times 99 \times \cdots \times 80$. }
		\item \question{ De combien de façons peut-on répartir 7 personnes sur 7 chaises ? }
		\item \question{ L'alphabet est un ensemble de 26 lettres. On appelle  << mot >> un ensemble de lettres. On choisit 5 lettres distinctes de l'alphabet. Combien de mots différents peut-on former avec ces 5 lettres ? }
	\end{enumerate}
}
