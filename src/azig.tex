\uuid{azig}
\titre{}
\theme{AM}
\auteur{Q. Liard}
\organisation{AMSCC}
\contenu{
\texte{


L'état-major d'une armée souhaite optimiser l'entretien de ses véhicules blindés en fonction de leur durée opération en environnement hostile. Pour cela, on cherche à modéliser la relation entre :

\begin{itemize}
    \item  X : Nombre de mois en mission dans une zone de conflit.
    \item  Y : Coût moyen des réparations après mission (en milliers d'euros).
\end{itemize}

Les valeurs suivantes ont été observées sur un échantillon de véhicules après différentes missions :

\begin{center}
    \begin{tabular}{|c|c|}
        \hline
        \textbf{Mois en mission X} & \textbf{Coût de réparation Y (k€)} \\
        \hline
        2  & 56 \\
        3 & 61 \\
        4 & 72 \\
        6 & 95 \\
        8  & 150 \\
        9 & 207 \\
        10 & 312 \\
        12 & 560 \\
      
        \hline
    \end{tabular}
\end{center}


\begin{enumerate}
    \item Déterminer la nature des variables étudiées \( X \) et \( Y \).
    \item Calculer les caractéristiques statistiques suivantes pour \( X \) et \( Y \) :
    \begin{itemize}
        \item La moyenne \( \bar{X} \) et \( \bar{Y} \).
        \item L’écart-type \( \sigma_X \) et \( \sigma_Y \).
        \item La covariance de $(X,Y),$ noté $Cov(X,Y)$.
    \end{itemize}
\end{enumerate}

\begin{enumerate}
    \setcounter{enumi}{2}
    \item Tracer le nuage de points \( (X, Y) \) et calculer le coefficient de corrélation $\rho(X,Y)$.
\end{enumerate}

\begin{enumerate}
    \setcounter{enumi}{3}
    \item On pose $Z = \ln(Y)$. Calculer $Cov(X,Z)$ et calculer le coefficient de corrélation linéaire $\rho(X,Z)$ entre $X$ et $Z$. Interpréter.
   \item  Déterminer l’équation de la droite de régression des moindres carrés entre \( X \) et $ Z$, $$Z = aX + b.$$
    En déduire l’équation de la relation entre \( X \) et \( Y \) sous forme exponentielle: $Y = e^{aX + b}$.
    \item Prédire le coût moyen des réparations pour un véhicule ayant effectué 7 mois en mission.

 
\end{enumerate}






}
}