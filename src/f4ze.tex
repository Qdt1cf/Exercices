\uuid{f4ze}
\chapitre{Série entière}
\niveau{L2}
\module{Analyse}
\sousChapitre{Equations différentielles}
\titre{Résolution d'une équation différentielle}
\theme{équations différentielles, séries entières}
\auteur{}
\datecreate{2024-06-17}

\organisation{AMSCC}
\difficulte{}
\contenu{
\texte{ Soit l'équation différentielle $(E)$ définie pour $\alpha>0$ par :
	$$
	(1+x)y'-\alpha\,y=0.
	$$
	
	On note $f$ la solution de cette équation différentielle vérifiant la condition initiale $f(0) = 1$.  }
\begin{enumerate}
	\item \question{ On suppose que $f$ s'écrit sous la forme d'une série entière $f(x)=\displaystyle\sum_{n=0}^{\infty}a_n\,x^{n}$. Déterminer les coefficients $(a_n)$ pour $n$ entier et déterminer le rayon de convergence de $f$. }
	\reponse{
		On pose $f(x)=\displaystyle\sum_{n=0}^{\infty}a_n\,x^{n}$. Alors $f'(x)=\displaystyle\sum_{n=1}^{\infty}n a_n\,x^{n-1}$.
		L'équation différentielle $(1+x)y'-\alpha\,y=0$ devient :
		$$ (1+x)\sum_{n=1}^{\infty}n a_n\,x^{n-1} - \alpha \sum_{n=0}^{\infty}a_n\,x^{n} = 0 $$
		$$ \sum_{n=1}^{\infty}n a_n\,x^{n-1} + \sum_{n=1}^{\infty}n a_n\,x^{n} - \alpha \sum_{n=0}^{\infty}a_n\,x^{n} = 0 $$
		On effectue un changement d'indice pour le premier terme. Soit $k=n-1$, donc $n=k+1$. Quand $n=1$, $k=0$.
		$$ \sum_{k=0}^{\infty}(k+1) a_{k+1}\,x^{k} + \sum_{n=1}^{\infty}n a_n\,x^{n} - \alpha \sum_{n=0}^{\infty}a_n\,x^{n} = 0 $$
		On peut faire commencer la deuxième somme à $n=0$ car le terme pour $n=0$ est $0 \cdot a_0 \cdot x^0 = 0$.
		En remplaçant $k$ par $n$ dans la première somme :
		$$ \sum_{n=0}^{\infty}(n+1) a_{n+1}\,x^{n} + \sum_{n=0}^{\infty}n a_n\,x^{n} - \alpha \sum_{n=0}^{\infty}a_n\,x^{n} = 0 $$
		$$ \sum_{n=0}^{\infty} \left[ (n+1) a_{n+1} + n a_n - \alpha a_n \right] x^{n} = 0 $$
		Par unicité des coefficients d'une série entière, on a pour tout $n \ge 0$ :
		$$ (n+1) a_{n+1} + (n-\alpha) a_n = 0 $$
		$$ a_{n+1} = -\frac{n-\alpha}{n+1} a_n = \frac{\alpha-n}{n+1} a_n $$
		La condition initiale $f(0)=1$ donne $a_0 = 1$ (car $f(0) = \sum a_n 0^n = a_0$).
		Calculons les premiers termes :
		$a_0 = 1$
		$a_1 = \frac{\alpha-0}{1} a_0 = \alpha$
		$a_2 = \frac{\alpha-1}{2} a_1 = \frac{\alpha(\alpha-1)}{2}$
		$a_3 = \frac{\alpha-2}{3} a_2 = \frac{\alpha(\alpha-1)(\alpha-2)}{3 \cdot 2 \cdot 1}$
		Par récurrence, on montre que $a_n = \frac{\alpha(\alpha-1)\dots(\alpha-n+1)}{n!}$.
		Ceci est la définition du coefficient binomial généralisé $\binom{\alpha}{n}$.
		Donc, $a_n = \binom{\alpha}{n}$.
		La solution est $f(x) = \displaystyle\sum_{n=0}^{\infty} \binom{\alpha}{n} x^n$.
		
		Pour déterminer le rayon de convergence $R$, on utilise la règle de d'Alembert :
		$$ \lim_{n\to\infty} \left| \frac{a_{n+1}}{a_n} \right| = \lim_{n\to\infty} \left| \frac{\alpha-n}{n+1} \right| = \lim_{n\to\infty} \left| \frac{-n(1-\alpha/n)}{n(1+1/n)} \right| = \lim_{n\to\infty} \left| \frac{-(1-\alpha/n)}{1+1/n} \right| = |-1| = 1. $$
		Le rayon de convergence est $R = \frac{1}{1} = 1$.
		La série converge pour $|x|<1$.
	}
	
	\item \question{ Donner une expression alternative de $f,$ solution de $(E)$, en résolvant directement l'équation différentielle avec la condition initiale $f(0)=1$.  }
	\reponse{
		L'équation différentielle est $(1+x)y' - \alpha y = 0$. C'est une équation différentielle linéaire du premier ordre à variables séparables.
		Pour $x \neq -1$ et $y \neq 0$:
		$$ (1+x)\frac{dy}{dx} = \alpha y $$
		$$ \frac{dy}{y} = \frac{\alpha}{1+x} dx $$
		On intègre les deux membres :
		$$ \int \frac{1}{y} dy = \int \frac{\alpha}{1+x} dx $$
		$$ \ln|y| = \alpha \ln|1+x| + C_1 $$
		où $C_1$ est une constante d'intégration.
		$$ \ln|y| = \ln|(1+x)^\alpha| + C_1 $$
		$$ |y| = e^{C_1} |(1+x)^\alpha| $$
		Soit $C = \pm e^{C_1}$. On peut inclure la solution $y=0$ en autorisant $C=0$.
		La solution générale est $y(x) = C(1+x)^\alpha$.
		
		On utilise la condition initiale $f(0)=1$.
		$f(0) = C(1+0)^\alpha = C \cdot 1^\alpha = C$.
		Donc $C=1$.
		L'expression alternative de la solution $f$ est $f(x) = (1+x)^\alpha$.
		
		On remarque que la série entière trouvée à la question 1, $f(x) = \displaystyle\sum_{n=0}^{\infty} \binom{\alpha}{n} x^n$, est le développement en série entière de $(1+x)^\alpha$, ce qui est cohérent.
	}
\end{enumerate}
}