\uuid{td24}
\titre{Loi exponentielle}
\theme{}
\auteur{}
\datecreate{2025-03-20}
\organisation{}

\contenu{

\texte{
On considère un échantillon i.i.d. $(X_i)_{1 \leq i \leq n}$ avec $n \geq 3$ et $X_1$ de densité $f_\theta(x) = \theta \exp(-\theta x) \mathbf{1}_{[0,+\infty[}(x)$, où $\theta > 0$ est le paramètre inconnu.
}

\begin{enumerate}
    \item \question{On propose d’estimer $\theta$ par $Y_n = \frac{1}{n} \sum_{i=1}^n X_i$. (a) Montrer que la v.a. $Y_n$ est bien définie. (b) Expliquer pourquoi il est logique de choisir $Y_n$ comme estimateur de $\theta$. (c) Déterminer la loi limite de $\sqrt{n}(Y_n - \theta)$. (d) Donner la loi de $\sum_{i=1}^n X_i$. En déduire la valeur de $E[(Y_n - \theta)^2]$.}
    \indication{}
    \reponse{}
    \item \question{Pour estimer $\theta$, on propose d’utiliser $Z_n = \frac{n-1}{n} Y_n$. (a) $Z_n$ vérifie-t-il des propriétés de convergence similaires à celles de $Y_n$ ? (b) Qui de $Y_n$ ou $Z_n$ choisiriez-vous pour estimer $\theta$ ?}
    \indication{}
    \reponse{}
    \item \question{Soit $\alpha \in ]0, 1[$. Donner un intervalle de confiance bilatère de niveau asymptotique $(1-\alpha)$ pour $\theta$.}
    \indication{}
    \reponse{}
    \item \question{Proposer un test de niveau asymptotique $\alpha$ pour tester $H_0 : \theta \geq 1$ contre $H_1 : \theta < 1$.}
    \indication{}
    \reponse{}
    \item \question{Proposer un test de niveau asymptotique $\alpha$ pour tester $H_0 : \theta = 1$ contre $H_1 : \theta \neq 1$.}
    \indication{}
    \reponse{}
\end{enumerate}

}