\titre{}

\texte{ L'équipe médicale d'une entreprise fait ses
 statistiques sur le taux de cholestérol de ses employés; les
observations sur 100 employés tirés au sort sont les suivantes.

\begin{center}
	\begin{tabular}{c|c}
	taux de cholestérol en cg:(centre classe) & effectif d'employés: \\
	\hline $120$ & $9$ \\ 
	$160$ & $22$ \\ 
	$200$ & $25$ \\ 
	$240$ & $21$ \\ 
	$280$ & $16$ \\ 
	$320$ & $7$
\end{tabular}
\end{center} }

\begin{enumerate}
	\item \question{ Calculer la moyenne $m_{e}$ et la variance $\sigma_{e}^2$ sur l'échantillon. }
	\reponse{On obtient, sur l'échantillon, la moyenne $m_{e}=213{,}6$, l'écart-type $\sigma _{e}=55.7767$. (\href{https://stcyrterrenetdefensegouvf-my.sharepoint.com/:x:/g/personal/maxime_nguyen_st-cyr_terre-net_defense_gouv_fr/EWYi-azbRsdHtG21-VGJXbcBqFKfHOj8SPV3FMXeKLBM1g?e=fBHC4S}{feuille de calcul})}
	\item \question{ Estimer sans biais la moyenne et l'écart-type pour le taux de cholestérol dans toute l'entreprise. }
	\reponse{ La moyenne sur l'entreprise est estimée sans biais par l'estimateur $\overline{X} = \frac{1}{n}\sum_{i=1}^{n} X_i$ qui se réalise en $m_{e}=213{,}6$.
		La variance est estimée sans biais par l'estimateur $S^2 = \frac{1}{n-1} \sum_{i=1}^{n} (X_i - \overline{X})^2$ donc l'écart-type est estimé par: $s_{e}=\sqrt{\frac{100}{99}}55.77\simeq 56.06$. }
	\item \question{ Déterminer un intervalle de confiance  permettant d'estimer la moyenne du taux de cholestérol de tous les employés de cette entreprise avec une confiance de $90\%$.  }
	\reponse{ On en déduit, au seuil 90\%, un intervalle de confiance pour la
		moyenne par lecture de table de la loi normale centrée réduite avec $u_{\alpha/2 } \approx 1{,}644854$ : 
		$$\left[m_{e} - u_{\alpha/2 }\frac{s_{e}}{\sqrt{n}};
		m_{e} + u_{\alpha/2 }\frac{s_{e}}{\sqrt{n}}]=[204.379;222.821\right]$$.
		Ainsi le taux moyen de cholestérol est, à un seuil de confiance $90$\%, 
		située entre $204$ et $223$ cg. }
\end{enumerate}

