\uuid{V97L}
\titre{\'Etude d'un gain}
\theme{variables aléatoires discrètes}
\auteur{}
\datecreate{2023-09-20}
\organisation{AMSCC}

\contenu{
    \texte{
        On lance un dé équilibré. On gagne 1 euro si le résultat est pair, on pert 1 euro si le résultat est impair. Ŝoit $n \geq 1$ le nombre de parties. On note $X$ le nombre le nombre de lancers pairs obtenus au bout de $n$ parties et $G$ le gain obtenu au bout de $n$ parties. 
    }
    \begin{enumerate}
        \item \question{ Donner la loi de $X$, son espérance et sa variance. }
        \reponse{
            La variable aléatoire $X$ donne le nombre de succès à l'issue de $n$ expériences indépendantes de Bernoulli où le succès est l'obtention d'un résultat pair, de probabilité $p=0.5$. On a donc $X\sim\mathcal{B}(n,0.5)$, $\E(X) = n\times0.5 = \frac{n}{2}$ et $\V(X) = n\times0.5\times0.5 = \frac{n}{4}$.
        }
        \item \question{ Exprimer $G$ en fonction de $X$. }
        \reponse{
            On a $G = X - (n-X) = 2X-n$.
        }
        \item \question{ Exprimer l'événement \og le gain ou la perte n'excède pas 20 euros \fg{} en fonction de $X$. }
        \reponse{
            On a $-20 \leq G \leq 20 \iff -20 \leq 2X-n \leq 20 \iff -10 \leq X-\frac{n}{2} \leq 10$. Donc l'événement considéré est $\{|X - \frac{n}{2}| \leq 10\}$ ou encore $\{-10 \leq X - \frac{n}{2} \leq 10\}$.
        }
        \item \question{ En utilisant le théorème central limite sans correction de continuité, déterminer le nombre maximal de lancers $n$ à effectuer pour que la probabilité de l'événement \og le gain ou la perte n'excède pas 20 euros \fg{} soit supérieure à $0.9544$. }
        \reponse{
            On cherche $n$ tel que $\prob(|X - \frac{n}{2}| \leq 10) \geq 0.9544$. On sait que $\E(X) = \frac{n}{2}$ et $\V(X) = \frac{n}{4}$. D'après le théorème central limite, la variable aléatoire $Z = \frac{X-\frac{n}{2}}{\frac{\sqrt{n}}{2}}$ suit approximativement une loi normale centrée réduite. On a donc : 
            $$\prob(|X - \frac{n}{2}| \leq 10) = \prob\left(\frac{|X - \frac{n}{2}|}{\frac{\sqrt{n}}{2}} \leq \frac{10}{\frac{\sqrt{n}}{2}}\right) = \prob\left(|Z| \leq \frac{10}{\frac{\sqrt{n}}{2}}\right)$$
            On cherche donc $n$ tel que $\prob(|Z| \leq \frac{10}{\frac{\sqrt{n}}{2}}) = 2 \times \Phi(\frac{10}{\frac{\sqrt{n}}{2}}) - 1 = 2 \times \Phi(\frac{20}{\sqrt{n}}) - 1 \geq 0.9544$ soit encore $\Phi(\frac{20}{\sqrt{n}}) \geq 0.9772$.

            Par lecture de table, on trouve $\frac{20}{\sqrt{n}} \geq 2$ soit $n \leq 100$.
        }
    \end{enumerate}
}