\titre{Dé truqué}
\theme{Probabilités}
\auteur{}
\organisation{AMSCC}

\contenu{
\texte{On considère un dé truqué cubique dont les faces sont numérotées de $1$ à $6$. On suppose que le dé est truqué de sorte que la probabilité d'obtenir une face est proportionnelle au numéro inscrit sur cette face.

Soit $X$ la variable aléatoire donnant le numéro obtenu à l'issue d'un lancer.}
\begin{enumerate}
 \item \question{Déterminer la loi de $X$.}
 \reponse{ D'après les informations, pour tout $k \in \{1,2,3,4,5,6\}$, on a $\prob(X=k)=k\prob(X=1)$. Or, par définition d'une loi de probabilité, on a $\sum_{k=1}^6 \prob(X=k)=1$. Ainsi, on a :
\begin{align*}
    1 &= \sum_{k=1}^6 \prob(X=k) \\
    &= \sum_{k=1}^6 k\prob(X=1) \\
    &= \prob(X=1) \sum_{k=1}^6 k \\
    &= \prob(X=1) \times \frac{6\times 7}{2} \\
    &= 21 \prob(X=1).
\end{align*}
On en déduit que $\prob(X=1)=\frac{1}{21}$ et donc que pour tout $k \in \{1,2,3,4,5,6\}$, on a $\prob(X=k)=\frac{k}{21}$.
}
 \item \question{Calculer l'espérance de $X$.}
 \reponse{
    Par définition de l'espérance, on a : 
    \begin{align*}
        \E(X) &= \sum_{k=1}^6 k\prob(X=k) \\
        &= \sum_{k=1}^6 k \times \frac{k}{21} \\
        &= \frac{1}{21} \sum_{k=1}^6 k^2 \\
        &= \frac{7 \times 13}{21} \\
        &= \frac{13}{3}.
    \end{align*}
 }
 \item \question{On pose $Y=\frac{1}{X}$. Déterminer la loi de $Y$ et calculer son espérance.}
 \reponse{
    Les valeurs prises par $Y$ sont les inverses des valeurs prises par $X$. Ainsi, on a pour tout $k \in \{1,2,3,4,5,6\}$, $\prob\left( Y=\frac{1}{k} \right) = \prob(X=k)$.

    Pour calculer l'espérance, on utilise la définition :

    \begin{align*}
        \E(Y) &= \sum_{k=1}^6 \frac{1}{k} \prob\left(Y=\frac{1}{k}\right) \\
        &= \sum_{k=1}^6 \frac{1}{k} \times \frac{k}{21} \\
        &= \frac{1}{21} \sum_{k=1}^6 1 \\
        &= \frac{6}{21} \\
        &= \frac{2}{7}.
    \end{align*}
 }
\end{enumerate}

}