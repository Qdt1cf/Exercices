\titre{Moments de la loi exponentielle}
\theme{probabilité}
\auteur{Maxime NGUYEN}
\organisation{AMSCC}

\question{ Soit $X$ une variable aléatoire suivant une loi exponentielle  de paramètre $\lambda>0$. Démontrer l'existence des moments d'ordre $n$ de $X$ pour tout $n \in \mathbb{N}$ puis les calculer. }
\reponse{On sait par comparaison que la fonction $x\mapsto x e^{-\lambda x}$ est intégrable au voisinage de $+\infty$ donc on peut calculer par théorème de transfert pour tout $n \in \N^*$ : 
	\begin{align*}
		\mathbb{E}(X^n) &= \lambda \int_0^{+\infty} x^n e^{-\lambda x} dx \\
		&= \lambda \left[ \frac{x^n e^{-\lambda x}}{-\lambda} \right]_{0}^{+\infty} + \frac{1}{\lambda} n \int_{0}^{+\infty} x^{n-1} \lambda e^{-\lambda x} dx \\
		&= \frac{n}{\lambda } \mathbb{E}(X^{n-1})
	\end{align*}
	De plus, on a immédiatement que $\mathbb{E}(X^0) = 1$ donc par récurrence, on obtient :
	$$\mathbb{E}(X^n) = \frac{n!}{\lambda^n}$$
	
}
