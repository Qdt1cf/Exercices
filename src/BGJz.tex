\uuid{BGJz}
\chapitre{Probabilité continue}
\sousChapitre{Densité de probabilité}
\titre{Densité et transformation affine}
\theme{variables aléatoires à densité}
\auteur{}
\datecreate{2022-09-21}
\organisation{AMSCC}
\contenu{

\texte{Soit $X$ une variable aléatoire absolument continue admettant pour densité $f$. Soit $(a,b) \in \mathbb{R}^2$ avec $a>0$ et $Y=aX+b$. }

\question{Démontrer que $Y$ est une variable aléatoire absolument continue en exprimant sa densité en fonction de $f$.}

\indication{Exprimer la fonction de répartition de la variable $aX+b$ ou utiliser un théorème d'identification de densité pour cette variable aléatoire. }

\reponse{
	On utilise le théorème d'identification. La densité $f_Y$ de $Y$ est l'unique fonction telle que pour toute fonction $h$ continue bornée, on ait
	\[ \E(h(Y))=\int_\R h(y)f_Y(y)\ dy.\]
	Or, étant donné $h$, on a
	\begin{align*}
		\E(h(Y))&=\E(h(aX+b)) \\
		&= \int_\R h(ax+b) f_X(x) \ dx \quad \text{(par le théorème de transfert)} \\
		&= \int_\R h(y) f_X\left(\frac{y-b}{a}\right) \frac{1}{a} dy \quad \text{(par changement de variable } y=ax+b \text{)}
	\end{align*}
	d'où son identification
	\[ f_Y(y)=\frac{1}{a}f_X\left(\frac{y-b}{a}\right).\]
}}
