\chapitre{Polynôme, fraction rationnelle}
\sousChapitre{Division euclidienne}
\uuid{pZq6}
\titre{Calcul de division euclidienne}
\theme{polynômes}
\auteur{}
\datecreate{2023-01-23}
\organisation{AMSCC}
\contenu{

\texte{ Effectuer les divisions euclidiennes des polynômes définis ci-dessous. }

\begin{enumerate}
	\item \question{ $A(X)=X^4-3 X^3+4 X^2-6 X+8$, par $B(X)=X-1$; }


\reponse{ \begin{tabular}{r|l}
$X^4-3 X^3+4 X^2-6 X+8$ & $X-1$ \\
\cline { 2 - 2 }$-\left(X^4-X^3\right)$ & $X^3-2 X^2+2 X-4$ \\
\hline$-2 X^3+4 X^2-6 X+8$ & \\
$-\left(-2 X^3+2 X^2\right)$ & \\
\hline $2 X^2-6 X+8$ & \\
$-\left(2 X^2-2 X\right)$ & \\
\hline$-4 X+8$ & \\
$-(-4 X+4)$ & 4
\end{tabular}
$$
A(X)=X^4-3 X^3+4 X^2-6 X+8=\underbrace{(X-1)}_{B(X)} \cdot\left(X^3-2 X^2+2 X-4\right)+4
$$ }

\item \question{ $C(X)=3 X^5+2 X^4-2 X^2+1$, par $D(X)=X^3+X+2$. }
\reponse{ $$
\begin{array}{r|r}
3 X^5+2 X^4-2 X^2+1 & X^3+X+2 \\
-\left(3 X^5+3 X^3+6 X^2\right) & 3 X^2+2 X-3 \\
\hline 2 X^4-3 X^3-8 X^2+1 & \\
-\left(2 X^4+2 X^2+4 X\right) & \\
\hline-3 X^3-10 X^2-4 X+1 & \\
-\left(-3 X^3-3 X-6\right) & \\
\hline-10 X^2-X+7 &
\end{array}
$$

$$
C(X)=3 X^5+2 X^4-2 X^2+1=\underbrace{\left(X^3+X+2\right)}_{D(X)} \cdot\left(3 X^2+2 X-3\right)+\left(-10 X^2-X+7\right)
$$ }
\end{enumerate}}
