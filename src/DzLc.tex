\uuid{DzLc}
\chapitre{Statistique}
\niveau{L2}
\module{Probabilité et statistique}
\sousChapitre{Tests d'hypothèses, intervalle de confiance}
\titre{Test d'indépendance}
\theme{tests d'hypothèses}
\auteur{}
\datecreate{2022-10-19}
\organisation{AMSCC}
\contenu{

Dans le cadre d'une enquête sur le SIDA réalisée en Allemagne durant l'été 1990 (A. Hahn, W.H. Eirmbter et R. Jacob), on a interrogé 2089 personnes. Le questionnaire comportait notamment l'item suivant : Le sida est un péril omniprésent. Indiquez si vous êtes : d'accord, indécis, pas d'accord. Le croisement de la réponse du sujet avec son âge donne le tableau de contingence suivant :

\begin{center}
	\begin{tabular}{|c|c|c|c|c|}
	\hline Classe d'âge & d'accord & indécis & pas d'accord & Total \\
	\hline $18$ à $<30$ & 43 & 116 & 365 & 524 \\
	$30$ à $<40$ & 36 & 116 & 273 & 425 \\
	$40$ à $<50$ & 32 & 95 & 217 & 344 \\
	$50$ à $<60$ & 38 & 114 & 167 & 319 \\
	$60$ et plus & 67 & 160 & 250 & 477 \\
	\hline Total & 216 & 601 & 1272 & 2089 \\
	\hline
\end{tabular}
\end{center}

\begin{enumerate}
\item \question{ Réaliser un test du $\chi^2$ permettant de répondre à la question suivante: << Les réponses des sujets dépendent-elles de leur âge ? >> }
\reponse{ Le tableau des effectifs théoriques est donné par:
	Celui des contributions au $\chi^2$ est donné par:
\begin{center}
		\begin{tabular}{|c|c|c|c|}
		\hline Classe d'âge & d'accord & indécis & pas d'accord \\
		\hline $18$ à $<30$ & $2.31$ & $8.01$ & $6.61$ \\
		$30$ à $<40$ & $1.44$ & $0.32$ & $0.78$ \\
		$40$ à $<50$ & $0.36$ & $0.16$ & $0.27$ \\
	$50$ à $<60$ & $0.76$ & $5.58$ & $3.82$ \\
		$60$ et plus & $6.33$ & $3.78$ & $5.63$ \\
		\hline
	\end{tabular}
\end{center}
	On obtient $\chi_{o b s}^2=46$. Or pour un seuil de $1 \%$ et $d d l=(5-1)(3-1)=8$, on $a: \chi_c^2=20.09$. La réponse du sujet est donc dépendante de son âge. }
\item \question{ Comparer de même à l'aide d'un test du $\chi^2$ les réponses des deux dernières classes d'âge, puis les réponses des moins de 30 ans à celles des 60 ans et plus. }
\reponse{ En ne considérant que les deux dernières classes d'âge, on obtient le tableau d'effectifs théoriques suivant:
\begin{center}
		\begin{tabular}{|c|c|c|c|}
		\hline Classe d'âge & d'accord & indécis & pas d'accord \\
		\hline $50$ à $<60$ & $42.1$ & $109.8$ & $167.1$ \\
		$60$ et plus & $62.9$ & $164.2$ & $249.9$ \\
		\hline
	\end{tabular}
\end{center}
	On obtient alors $\chi_{obs}^2=0.92$. Or, pour $\alpha=5 \%$ et $d d l=2, \chi_c^2=5.99$. Les réponses des sujets sont cette fois indépendantes de leur appartenance à l'une ou l'autre classe d'âge. Dans le dernier cas, $\chi_{o b s}^2=31.62$, ce qui est significatif d'une dépendance. }
\end{enumerate}
}
