\uuid{aebc}
\titre{}
\theme{AM}
\auteur{Q. Liard}
\organisation{AMSCC}
\contenu{
\texte{

Le tableau suivant contient la liste de 9 pays d'Amérique du Nord et d'Amérique centrale, dont la population dépassait un million d'habitants en 1980. Pour chaque pays, le tableau indique son taux de natalité (nombre de naissances par année pour 1000 personnes) ainsi que son taux d'urbanisation (pourcentage de la population vivant dans des villes de plus de 100 000 habitants).

\begin{table}[H]
    \centering
    \renewcommand{\arraystretch}{1.3}
    \begin{tabular}{|l|c|c|}
        \hline
        \textbf{Pays} & \textbf{Taux de natalité} & \textbf{Taux d'urbanisation} \\
        \hline
        Canada & 16.2 & 55 \\
        Costa Rica & 30.5 & 27.3 \\
        Cuba & 16.9 & 33.3 \\
        États-Unis & 16.0 & 56.5 \\
        El Salvador & 40.2 & 11.5 \\
        Guatemala & 38.4 & 14.2 \\
        Haïti & 41.3 & 13.9 \\
        Honduras & 43.9 & 19 \\
        Jamaïque & 28.3 & 33.1 \\
        \hline
    \end{tabular}
    \caption{Taux de natalité et d'urbanisation en 1980}
    \label{tab:taux_natalite_urbanisation}
\end{table}

On note \( X \) le taux de natalité et \( Y \) le taux d'urbanisation. 

%\begin{align*}
 %%   \sum_{i=1}^{9} x_i &= 271.7, \quad & \sum_{i=1}^{9} y_i &= 263.8, \\
  %  \sum_{i=1}^{9} x_i^2 &= 9258.69, \quad & \sum_{i=1}^{9} y_i^2 &= 10055.14, \\
%    \sum_{i=1}^{9} x_i y_i &= 6542.9.
%\end{align*}
}

\begin{enumerate}
    \item Calculer la variance de \( X \) et la variance de \( Y \), en déduire l'écart type de \( X \) et celui de \( Y \).
    \item Calculer $Cov(X,Y),$ la covariance entre la variable $X$ et $Y$  puis le coefficient de corrélation linéaire $\rho(X,Y)$.
    \item Peut-on raisonnablement établir un lien entre le taux de natalité et le taux d'urbanisation? 
    \item Donner l'équation d'une droite de régression entre les variables $X$ et $Y$. Tracer le nuage de points et la droite de régression linéaire sur le fichier Excel fourni.
\end{enumerate}









}