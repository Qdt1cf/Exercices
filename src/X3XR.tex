\chapitre{Probabilité continue}
\sousChapitre{Densité de probabilité}
\uuid{X3XR}
\titre{Fonction de répartition}
\theme{variables aléatoires à densité}
\auteur{}
\datecreate{2022-10-07}
\organisation{AMSCC}


\contenu{

\texte{ Soit $\lambda>0$ et $X$ une variable aléatoire admettant pour densité $f(x)=\lambda e^{-\lambda x}1_{[0;+\infty[}(x)$.  }
\question{ Vérifier que $f$ définit bien une fonction densité, puis déterminer la fonction de répartition $F_X$ de $X$. 

}
 \reponse{Il suffit de vérifier que $f(x) \geq 0$ pour tout $x \in \R$ puis de calculer :
 	\begin{align*}
 	\int_{-\infty}^{+\infty} f(x)dx &= \int_0^{+\infty} \lambda e^{-\lambda x} dx \\
 	                               &= \left[-e^{-\lambda x}\right]_0^{+\infty} \\
 	                               &= 1
 	\end{align*}
On détermine maintenant la fonction de répartition : soit $t \in \R$ ;
\begin{itemize}
	\item si $t<0$, alors $F_X(t) = \int_{-\infty}^t f(x)dx = \int_{-\infty}^t 0 dx = 0$ ;
	\item si $t \geq 0$, alors $F_X(t) = \int_{-\infty}^t f(x)dx = \int_{-\infty}^0 0 dx + \int_0^t \lambda e^{-\lambda x} dx = 0 + \left[-e^{-\lambda x}\right]_0^t = 1 - e^{-\lambda t}$.
\end{itemize} 
}}
