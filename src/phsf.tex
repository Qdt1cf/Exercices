\uuid{phsf}
\chapitre{Probabilité discrète}
\niveau{L2}
\module{Probabilité et statistique}
\sousChapitre{Probabilité et dénombrement}
\titre{Calcul de probabilité}
\theme{probabilités}
\auteur{}
\datecreate{2023-01-24}
\organisation{AMSCC}
\contenu{


\texte{ 	On  jette un dé à 6 faces et on observe le résultat. }
	\begin{enumerate}
		\item \question{ Quel univers peut-on définir pour modéliser cette expérience aléatoire ? }
		\item \question{ On observe que $\PP(1)=0.3$, $\PP(2)=0.15$, $\PP(3)=0.1$, $\PP(4)=\PP(2)$, $\PP(5)=\PP(6)$. Le dé est-il truqué ? Déterminer $\PP(5)$ et $\PP(6)$. }
		\item \texte{ On considère les deux événements suivants :
		\begin{enumerate}
			\item $A$ : \og le nombre obtenu est impair \fg{}
			\item $B$ : \og le nombre obtenu est supérieur ou égal à 3 \fg{}.
		\end{enumerate} }
		\question{ Calculer les probabilités $\PP(A)$, $\PP(B)$, $\PP(A \cap B)$. }
		\item \question{ Calculer $\PP(A \cup B)$ de deux manières différentes. }
		\item \question{ Décrire à l'aide d'une phrase les événements $\bar A$ et $\bar B$ puis calculer leur probabilité. }
		\item\question{  Donner un exemple de deux événements incompatibles $C$ et $D$ puis calculer $\PP(C)$, $\PP(D)$, $\PP(C \cap D)$, $\PP(C \cup D)$. }
	\end{enumerate}
}
