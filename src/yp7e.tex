\titre{Calcul de sommes}
\theme{séries}
\auteur{}
\organisation{AMSCC}

\texte{ Sachant que $\displaystyle e=\sum_{n\geq 0} \frac{1}{n!}$, déterminer la valeur des sommes suivantes : }
\begin{enumerate}
	\item \question{$\sum_{n\geq0} \frac{n+1}{n!}$}
	%\reponse{La série est équivalente à $e\sum_{n\geq0} \frac{n}{n!} + e\sum_{n\geq0} \frac{1}{n!} = e(e+1)$, en utilisant le fait que $\sum_{n\geq0} \frac{n}{n!}$ est la dérivée de la série exponentielle et $\sum_{n\geq0} \frac{1}{n!} = e$.}
	
	\item \question{$\sum_{n\geq0} \frac{n^2-2}{n!}$}
	%\reponse{La série est équivalente à $e\sum_{n\geq0} \frac{n^2}{n!} - 2e = e^2 - 2e$, en utilisant le fait que $\sum_{n\geq0} \frac{n^2}{n!}$ est la dérivée seconde de la série exponentielle et $\sum_{n\geq0} \frac{1}{n!} = e$.}
	
	\item \question{$\sum_{n\geq0} \frac{n^3}{n!}$}
	%\reponse{La série est équivalente à $e\sum_{n\geq0} \frac{n^3}{n!} = e^3$, en utilisant le fait que $\sum_{n\geq0} \frac{n^3}{n!}$ est la dérivée troisième de la série exponentielle.}
\end{enumerate}
