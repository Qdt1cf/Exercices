\titre{ Saturation d'un standard téléphonique }
\theme{ probabilités }
\auteur{ }
\organisation{AMSCC}

\contenu{
    \texte{
        Une entreprise compte $300$ employés. Chaque employé utilise son téléphone de manière aléatoire, en moyenne $6$ minutes par heure. Cela implique qu'à un instant donné, la probabilité qu'il soit au téléphone est de $\frac{6}{60} =  0{,}1$. On suppose que l'utilisation du téléphone par un employé est indépendante de celle des autres employés.  
    }
    \begin{enumerate}
        \item \question{ Il est 10h00. Soit $X$ le nombre d'employés qui téléphonent à cet instant. Déterminer la loi de $X$. }
        \item \question { Justifier que la loi de $X$ peut être approchée par une loi normale $\mathcal{N}(30;\sqrt{27})$. }
        \item \question{ Estimer le nombre $\ell$ de lignes que l'entreprise doit installer pour que la probabilité que toutes les lignes soient occupées soit au plus égale à $2{,}5\%$. }
    \end{enumerate}
}