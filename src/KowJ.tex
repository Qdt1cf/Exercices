\chapitre{Probabilité continue}
\sousChapitre{Loi normale}
\uuid{KowJ}
\titre{ Saturation d'un standard téléphonique }
\theme{loi binomiale, loi normale, théorème central limite}
\auteur{ }
\datecreate{2023-10-16}
\organisation{AMSCC}

\contenu{
    \texte{
        Une entreprise compte $300$ employés. Chaque employé utilise son téléphone de manière aléatoire, en moyenne $6$ minutes par heure. Cela implique qu'à un instant donné, la probabilité qu'il soit au téléphone est de $\frac{6}{60} =  0{,}1$. On suppose que l'utilisation du téléphone par un employé est indépendante de celle des autres employés.  
    }
    \begin{enumerate}
        \item \question{ Il est 10h00. Soit $X$ le nombre d'employés qui téléphonent à cet instant. Déterminer la loi de $X$. }
        \reponse{
            On a $X \sim \mathcal{B}(300,0{,}1)$.
        }
        \item \question { Justifier que la loi de $X$ peut être approchée par une loi normale $\mathcal{N}(30;\sqrt{27})$. }
        \reponse{
            On a $np = 300 \times 0{,}1 = 30$ et $np(1-p) = 300 \times 0{,}1 \times 0{,}9 = 27$. Le paramètre $n$ est considéré comme grand ($>30$) donc les conditions d'application du théorème de Moivre-Laplace sont vérifiées. On peut donc approcher la loi de $\frac{X-30}{\sqrt{27}}$ par une loi normale centrée réduite, ce qui revient à approcher $X$ par une loi normale $\mathcal{N}(30;\sqrt{27})$. 
        }
        \item \question{ Estimer le nombre $\ell$ de lignes que l'entreprise doit installer pour que la probabilité que toutes les lignes soient occupées soit au plus égale à $2{,}5\%$. }
        \reponse{
            On cherche $\ell$ tel que $\prob(X \geq \ell) \leq 0{,}025$. On a : 
            \begin{align*}
                \prob\left( X \geq \ell \right) &= \prob\left( \frac{X - 30}{\sqrt{27}} \geq \frac{\ell - 30}{\sqrt{27}} \right) \\
                &= 1- \prob\left( \frac{X - 30}{\sqrt{27}} \leq \frac{\ell - 30}{\sqrt{27}} \right) \\
            \end{align*}
            On cherche donc $\ell$ tel que $\prob \left( \frac{X - 30}{\sqrt{27}} \leq \frac{\ell - 30}{\sqrt{27}} \right) \geq 0{,}975$. Par lecture inverse de table, on en déduit que $\frac{\ell - 30}{\sqrt{27}} \geq 1{,}96$ donc $\ell \geq 30 + 1{,}96 \times \sqrt{27} \approx 38{,}8$. On en déduit que $\ell = 39$. 

        }
    \end{enumerate}
}