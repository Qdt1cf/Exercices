\chapitre{Polynôme, fraction rationnelle}
\sousChapitre{Racine, décomposition en facteurs irréductibles}
\uuid{gLXG}
\titre{Divisibilité et racines}
\theme{polynômes}
\auteur{}
\datecreate{2023-01-23}
\organisation{AMSCC}
\contenu{

\question{ Soit $n \in \mathbb{N} \backslash\{0,1\}$. Montrer que $P_n(X)=(X+1)^{2 n}-X^{2 n}-2 X-1$ est divisible par $Q(X)=X(X+1)(2 X+1)$. }

\indication{ Traduire cette information de divisibilité de polynômes en termes de racines : si $P_n$ est divisible par $Q$ alors les racines de $Q$ sont également des racines de $P_n$. }


\reponse{ On a :
\begin{align*}
P_n(0) & =(1)^{2 n}-0-0-1=0 \\
P_n(-1) & =(-1+1)^{2 n}-(-1)^{2 n}-2 \cdot(-1)-1=0-1+2-1=0 \\
P_n\left(-\frac{1}{2}\right) & =\left(-\frac{1}{2}+1\right)^{2 n}-\left(-\frac{1}{2}\right)^{2 n}-2 \cdot\left(-\frac{1}{2}\right)-1=\left(\frac{1}{2}\right)^{2 n}-\left(\frac{1}{2}\right)^{2 n}+1-1=0
\end{align*}

$0$ racine signifie que $X$ divise $P_n(X): P_n(X)=X Q_1(X)$;
$-1$ racine signifie que $(X+1)$ divise $P_n(X): P_n(X)=(X+1) \underbrace{Q_2(X)}_{X \cdot Q_3(X)}=X \cdot(X+1)  Q_3(X)$;
$-\frac{1}{2}$ racine signifie que $(2 X+1)$ divise $P_n(X)$ :
$$
P_n(X)=(2 X+1) \cdot Q_4(X)=\underbrace{X (X+1) (2 X+1)}_{Q(X)} \cdot Q_5(X)
$$
donc $Q(X)=X(X+1)(2 X+1)$ divise $P_n(X)=(X+1)^{2 n}-X^{2 n}-2 X-1$. }}
