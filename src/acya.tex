\chapitre{Déterminant, système linéaire}
\sousChapitre{Système linéaire, rang}
\uuid{acya}
\titre{Système linéaire avec paramètre et déterminant}
\theme{Algèbre linéaire}
\auteur{}
\organisation{AMSCC}

\contenu{
\texte{
Considérer le système suivant dépendant d'un paramètre réel \(m\) :
\begin{center}
$$
\begin{cases}
x + my + (m-1)z = m+1, \\
3x + 2y + mz = 3, \\
(m-1)x + my + (m+1)z = m-1.
\end{cases}
$$
\end{center}
}
\begin{enumerate}
    \item \question{Écrire le système sous forme matricielle, c'est-à-dire :
    $$
    A\,\mathbf{X} = \mathbf{B},
    $$
    où \(A\) est la matrice des coefficients, \(\mathbf{X}\) le vecteur des inconnues, et \(\mathbf{B}\) le vecteur des constantes.}
\reponse{La matrice $A$ est $\begin{pmatrix}
1 & m & m-1 \\ 3 & 2 & m \\ m-1 & m & m-1 \\

\end{pmatrix}$, le vecteur d'inconnue est $\mathbf{X}=\begin{pmatrix} x \\ y \\ z \end{pmatrix}$ et le vecteur d'inconnue est $\mathbf{B}=\begin{pmatrix} m+1 \\ 3 \\ m-1 \end{pmatrix}$}
    \item \question{Calculer le déterminant de la matrice \(A\) et montrer que
    $
    \det(A) = m^2(m-4).
    $
}
\reponse{Le déterminant est $A$ est:
$$\det(A)=\begin{vmatrix}
1 & m & m-1 \\ 3 & 2 & m \\ m-1 & m & m-1 \\
\end{vmatrix}=\begin{vmatrix}
-m+2 & 0 & -2 \\ 3 & 2 & m \\ m-1 & m & m-1 \\
\end{vmatrix}
$$
$$=(2-m)(-m^2+2m+2)-6m+4m-4=m^2(m-4).$$


}
    \item \question{Déterminer les valeurs de \(m\) pour lesquelles le système n'a pas de solution unique.}
    \reponse{Pour $m=0$ et $m=4$ la matrice $A$ pour déterminant $0$, la matrice $A$ n'est pas inversible donc il existe une infinité de solutions ou pas de solution.}
 \item \question{Le système a-t-il une solution pour $m=4$? Si oui, la déterminer.}
 \reponse{Le système s'écrit:
 \begin{center}
$$
\begin{cases}
x + 4y + 3z = 5, \\
3x + 2y + 4z = 3, \\
3x + 4y + 5z = 3.
\end{cases}
$$
\end{center}
Il est équivalent au système  
 \begin{center}
$$
\begin{cases}
x=5+2y \\
z=-2y \\
15=3.
\end{cases}
$$
\end{center}
Ce système n'a donc aucune solution!
  }
    \item \question{Si \(m \neq 0\) et \(m \neq 4\), résoudre le système.}
    
    \reponse{Si $m\neq 4$ et $m\neq 0$ alors le système a une unique solution:
  $$x=-2\frac{m-1}{m(m-4)},\quad
    y=\frac{m^2-4m-3}{m(m-4)},\quad z=\frac{m+2}{m(m-4)}.$$
    }

   
\end{enumerate}








}