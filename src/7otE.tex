\titre{  }
\theme{statistiques}
\auteur{}
\organisation{AMSCC}


\texte{ Lors d'un référendum, un sondage aléatoire simple avec remise pratiqué sur $1000$ personnes a
donné $55\%$ pour le Oui et $45\%$ pour le Non. }

\begin{enumerate}
	\item Est-il plus précis de faire un sondage sur $1000$ personnes dans une population de $1$ million de personnes ou un sondage sur $2000$ personnes dans une population de $10$ millions de personnes ? Justifier.
	\item Concernant le référendum cité ci-dessus, déterminer un intervalle contenant le pourcentage de Oui avec une probabilité de $0{,}95$.
	\item Peut-on considérer, avec une confiance de $95\%$, que le Oui l'emporte ? La réponse est-elle
	la même avec un niveau de confiance de $99\%$ ?
	\item Si, pour un référendum, on sait que << oui >> se situe autour de $50\%$, combien de personnes
	faudrait-il interroger pour que la proportion de << Oui >> soit connue à $1\%$ près (en plus ou en
	moins), avec un niveau de confiance de $0{,}95$.
\end{enumerate}