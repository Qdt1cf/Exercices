\chapitre{Statistique}
\sousChapitre{Autre}
\uuid{t2sS}
\titre{Variable quantitative à classer}
\theme{statistiques, tableur}
\auteur{}
\datecreate{2022-09-30}
\organisation{AMSCC}
\contenu{

\texte{ 	A partir des données du fichier suivant : \href{https://github.com/smaxx73/Exercices/blob/main/data/ronfle.txt}{ronfle.txt} qui résume des relevés effectués lors d'une étude sur le ronflement : }

\begin{enumerate}
	\item \question{ Importer les données dans un fichier tableur. }
	\item \question{ Classer la variable <<taille>> par intervalles de longueur 10 cm. }
	\item \question{ Peut-on dire que les âges sont uniformément représentés dans cette étude ? On pourra justifier la réponse à l'aide d'un histogramme bien choisi. }
	\item \question{ Afficher un graphique représentant la taille en fonction du poids et commenter. }
	\item \question{  Proposer un graphique qui fait apparaître la répartition H/F entre différentes catégories de consommation d'alcool.  }
\end{enumerate}

\reponse{\href{https://stcyrterrenetdefensegouvf-my.sharepoint.com/:x:/g/personal/maxime_nguyen_st-cyr_terre-net_defense_gouv_fr/EQRAlehIJHFOoA4WmO3ZMFQBHFCbwlSMyE8PWER0Vq4T6w?e=Hl08gY}{Lien vers le fichier tableur}}}
