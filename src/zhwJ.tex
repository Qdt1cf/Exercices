\titre{Etude d'extrema}
\theme{calcul différentiel}
\auteur{}
\organisation{AMSCC}
\contenu{


	On étudie la fonction de deux variables $$f \colon (x,y) \mapsto x\ln(y)-y\ln(x)$$
	
	\begin{enumerate}
		\item \question{ Déterminer l'ensemble de définition de la fonction $f$ et vérifier que $f$ est de classe $\mathcal{C}^2$ sur son ensemble de définition. }
		\reponse{L'ensemble de définition de $f$ est $D_f = \{(x,y) \in \R^2\, , \, x>0, y>0 \}$}
		\item \question{ On pose $h(t) = t-\ln(t)-\frac{1}{t}$ : déterminer l'ensemble de définition de $h$ et étudier ses variations. }
		\reponse{L'ensemble de définition de $h$ est $D_h = \{t \in \R\, , \, t>0\}= \R_+^*$. La fonction $h$ est dérivable sur son ensemble de définition et $\forall t \in \R_+^*$ : 
			$$h'(t) = 1-\frac{1}{t}+\frac{1}{t^2} = \frac{t^2-t+1}{t^2} = \frac{(t+1)^2+t}{t^2} > 0$$
			Par conséquent, $h$ est strictement croissante sur $]0;+\infty[$. 
		}
		\item \question{ Démontrer que $\mathrm{grad}_f(x,y)=(0,0) \iff 	\begin{cases}
			h\left(\frac{x}{y}\right) = 0\\
			\frac{x}{y} - \ln(x) =0
		\end{cases}$ }
		\reponse{ 	\begin{align*}
				\begin{cases}
					\frac{\partial f}{\partial x}(x,y) = 0\\
					\frac{\partial f}{\partial y}(x,y) =0
				\end{cases}
				\Leftrightarrow
				\begin{cases}
					\ln(y) - \frac{y}{x} = 0\\
					\frac{x}{y} - \ln(x)=0
				\end{cases}		
				\Leftrightarrow
				\begin{cases}
					\ln(y) - \ln(x) +  \frac{x}{y}  - \frac{y}{x} = 0            \\
					\frac{x}{y} - \ln(x) = 0
				\end{cases}		
				\Leftrightarrow
				\begin{cases}
					h\left(\frac{x}{y}\right) = 0            \\
					\frac{x}{y} - \ln(x) = 0
				\end{cases}			
		\end{align*}}
		\item \question{ En déduire l'ensemble des points stationnaires de $f$. }
		\reponse{La fonction $h$ s'annule une et une seule fois sur son ensemble définition en $t=1$ donc $(x,y)$ est un point stationnaire si et seulement si $x=y=e$. Il existe un unique point stationnaire qui est le point $(e,e)$. }
		\item \question{ Déterminer l'ensemble des points extrémaux (locaux et globaux) de $f$. }
		\reponse{Il est clair que $f$ n'admet pas d'extremum global. En effet, on peut voir par exemple que $\lim\limits_{x \to +\infty}f(x,1) = -\infty$ et $\lim\limits_{y \to +\infty}f(1,y) = +\infty$.
			
			De plus, au voisinage du point stationnaire $(e,e)$, on peut étudier les conditions du second ordre en formant la matrice hessienne : 
			
			$$Hess_f(x,y)=\begin{pmatrix} 
				\frac{\partial^2 f}{\partial x^2}(x,y) = 1/x^2 & \frac{\partial^2 f}{\partial x \partial y}(x,y) = 1/y-1/x \\
				\frac{\partial^2 f}{\partial y \partial x}(x,y) = 1/y-1/x & \frac{\partial^2 f}{\partial y^2}(x,y) = -1/y^2 
			\end{pmatrix}$$
			d'où 
			$$Hess_f(e,e)=\begin{pmatrix} 
				1/e^2 &  0 \\
				0 & -1/e^2 
			\end{pmatrix}$$
			
			On a un déterminant négatif, donc le point $(e,e)$ est un point selle. 
		}
	\end{enumerate}
}
