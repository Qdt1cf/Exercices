\uuid{dXNn}
\titre{Estimation du paramètre d'une loi}
\theme{estimateurs}
\auteur{Maxime NGUYEN}
\datecreate{2022-10-04}
\organisation{AMSCC}
\contenu{

\texte{ 	On fixe un réel $a>0$ et on définit une variable aléatoire $X$ sur $\mathbb{N}$ dont la loi de probabilité est définie par 
$$\PP(X=k)=\frac{a^k}{(1+a)^{k+1}}$$
pour tout $k \in \mathbb{N}$. }

\question{ A l'aide de la méthode du maximum de vraisemblance, définir un estimateur de $a$ et déterminer son biais. }

\reponse{
On définit un échantillon $X_1,...,X_n$ et on considère la probabilité que cet échantillon réalise un ensemble de valeurs $V=\{x_1,...,x_n\}$. La fonction de vraisemblance s'écrit alors 
$$L(x_1,...,x_n,a) = \PP(X_1=x_1,...,X_n=x_n) = \prod_{k=1}^{n} \PP(X=x_k) = \prod_{k=1}^n \frac{a^{x_k}}{(1+a)^{x_k+1}} = \frac{a^{\sum x_k}}{(1+a)^{\sum x_k+1}}$$

On dérive ce quotient par rapport à $a$, on factorise par $a^{\sum x_k-1}(1+a)^{n+\sum x_k -1}$ et on trouve que 
$$\frac{\partial L}{\partial a}(x_1,...,x_n,a) = 0 \iff \sum_{k=0}^n x_k(1+a) - a(n+\sum_{k=0}^n x_k) = 0 \iff a = \frac{1}{n} \sum_{k=0}^n x_k$$

On a donc trouvé un meilleur estimateur du paramètre $a$ selon la méthode du maximum de vraisemblance : il s'agit de l'estimateur $T = \frac{1}{n} \sum_{k=0}^n X_k$.

Reste à calculer le biais de cet estimateur, autrement dit à calculer $B(T) = \mathbb{E}(T-a)$. Or pour tout entier $i$, $X_i$ suit la loi définie ci-dessus et on espérance se calcule de la manière suivante :
$$\mathbb{E}(X_i) = \sum_{k=0}^n k\PP(X_i=k) = \sum_{k=0}^n k \left( \frac{a}{1+a}  \right)^{k-1} \frac{a}{(1+a)^2} = \frac{1}{\left(1-\frac{a}{1+a}\right)^2} \times \frac{a}{(1+a)^2} = a$$
Donc $B(T) = \frac{1}{n} \times n \times a - a = 0$ : la variable $T$ est donc un estimateur sans biais du paramètre $a$.
}}
