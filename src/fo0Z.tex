\uuid{fo0Z}
\titre{Recherche des extremums globaux}
\theme{}
\auteur{Grégoire MENET}
\datecreate{2025-04-16}
\organisation{AMSCC}

\contenu{
	
	\texte{ 
		On considère la fonction \(f\) définie par \(f(x,y) = -x^2y + \frac{1}{2}y^2 + y\) pour tout \((x,y)\in\R^2\).
	}
	
	\begin{enumerate}
		\item \question{Déterminer les points critiques de \(f\).}
		\indication{Calculer les dérivées partielles par rapport à \(x\) et \(y\) puis résoudre le système \(\frac{\partial f}{\partial x}=0\) et \(\frac{\partial f}{\partial y}=0\).}
		\reponse{}
		\item \question{Déterminer la nature de ces points critiques.}
		\indication{Étudier la nature des points critiques à l'aide du critère de la matrice hessienne.}
		\reponse{}
		\item \question{La fonction \(f\) admet-elle des extremums globaux ? Justifier votre réponse.}
		\indication{Analyser le comportement de \(f\) pour \(|x|\to\infty\) et \(|y|\to\infty\) afin de déterminer la bornitude de la fonction.}
		\reponse{}
	\end{enumerate}
	
}
