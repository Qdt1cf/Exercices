\uuid{bz11}
\titre{Test d'adéquation}
\chapitre{Statistique}
\sousChapitre{Tests d'hypothèses, intervalle de confiance}
\theme{Statistique}
\auteur{Erwan L'Haridon}
\organisation{AMSCC}
\contenu{


\texte{
L’observation de 1883 familles de 7 enfants donne les résultats suivants :

\begin{center}
\begin{tabular}{|c|c|c|c|c|c|c|c|c|}
    \hline
    Nombre de garçons & 0 & 1 & 2 & 3 & 4 & 5 & 6 & 7 \\
    \hline
    Effectifs des familles & 27 & 111 & 287 & 480 & 529 & 304 & 126 & 19 \\
    \hline
\end{tabular}
\end{center}

Peut-on admettre au risque d’erreur de 5\% que le nombre de garçons suit une loi binomiale de paramètre \( p \) ? Si non, donner une estimation de \( p \).
}
}