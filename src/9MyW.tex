\uuid{9MyW}
\chapitre{Probabilité discrète}
\niveau{L2}
\module{Probabilité et statistique}
\sousChapitre{Lois de distributions}
\titre{Loi de Poisson}
\theme{variables aléatoires discrètes, loi de Poisson}
\auteur{}
\datecreate{2023-02-07}
\organisation{AMSCC}
\contenu{

\texte{ 	Soit $\lambda >0$ et $X$ suivant une loi de Poisson $\mathcal{P}(\lambda)$.  }
	
\question{ 	Démontrer que sa fonction de répartition est définie par 
	$$\forall n \in \N \qquad \PP(X \leq n) = \frac{1}{n!}\int_{\lambda}^{+\infty} e^{-x} x^n \dx$$ }

\reponse{ Par récurrence sur $n$ et intégration par parties.  }}
