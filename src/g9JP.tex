\uuid{g9JP}
\chapitre{Analyse numérique}
\sousChapitre{Autre}
\titre{Etude d'un schéma à un pas}
\theme{analyse numérique}
\auteur{}
\datecreate{2023-04-21}
\organisation{AMSCC}
\difficulte{}
\contenu{

\texte{ 	Soit l'équation différentielle sur $[0,T]$ :
$$(E)	\begin{cases} 
	y'(t) = f(t,y(t)) \\
	y(0) = a,
\end{cases} $$
où
$$
f(t,y)=\cos(t^2+y),\quad a=0,\quad T=1.
$$



Pour approcher la solution de $(E)$, on propose le schéma numérique suivant : 
$h=T/N$, $t_n=nh$, $y_0=a$ et 
$$(S) \colon y_{n+1} = y_n +  \frac{h}{3}\left( f(t_n,y_n)+2f\left(t_n+ \frac{3h}{4},y_n+\frac{3h}{4}f(t_n,y_n) \right) \right)$$ }


\begin{enumerate}	
	\item \question{ Montrer que $f$ est globablement lipschitzienne par rapport à la variable $y$. } 
	\reponse{ On a $\left|\frac{\partial f }{\partial y}(t,y)\right| = |\cos(t^2 + y)| \leq 1$ pour tout $(t,y) \in [0,T]\times\R$ donc par théorème des accroisements finis, $f$ est $1$-lipschitzienne par rapport $y$.}
	\item\question{  En déduire que le schéma numérique proposé est zéro-stable. }
	\reponse{ On pose $F(t,y,h) = \frac{1}{3}\left( f(t,y)+2f\left(t+ \frac{3h}{4},y+\frac{3h}{4}f(t,y) \right) \right)$. 
		On a \begin{align*}
			|F(t,y,h) - F(t,u,h)| &\leq \frac{1}{3}|\left(f(t,y,h) - f(t,u,h)\right)|\\  &+ \frac{2}{3}\left|f\left(t+ \frac{3h}{4},y+\frac{3h}{4}f(t,y) \right)- f\left(t+ \frac{3h}{4},u+\frac{3h}{4}f(t,u) \right) \right| \\
			& \leq \frac{1}{3}|y-u| + \frac{2}{3} \left|y+\frac{3h}{4}f(t,y) - u+\frac{3h}{4}f(t,u)\right| \\
			& \leq \frac{1}{3}|y-u| + \frac{2}{3}|y-u| + \frac{6h}{12}|y-u|
			\end{align*}
		Puisque $h$ est borné (par exemple par $1$), on en déduit que la fonction $F$ définissant le schéma numérique est globalement lipschizienne par rapport à la deuxième variable. 
		
		Par propriété du cours, le schéma numérique est donc zéro-stable. }
	\item \question{ Montrer que le schéma numérique est consistant d'ordre au moins $2$. }
	\reponse{On vérifie qu'il est consistant d'ordre 1 en appliquant le résultat du cours : on écrit le schéma sous la forme standard $y_{n+1} = y_n+hF(t_n,y_n,h)$ et on vérifie que $F(t,y,0) = f(t,y)$. 
		
	Pour voir qu'il est au moins d'ordre $2$, on applique le critère de consistance en calculant $\frac{1}{2}f^{[1]}(t,y)$.
	}
	
	\item \question{ Le schéma est-il convergent ? }
	\reponse{ Le schéma est consistant et stable, donc convergent. }
\end{enumerate}}
