\titre{Contrôle de qualité sur des pièces d'un euro}
\theme{ probabilités }
\auteur{ } 
\organisation{ AMSCC }
\contenu{
On effectue un contrôle sur des pièces d'un euro dont une proportion $p = 0{,}05$ est fausse. On considère un lot de 1000 pièces d'un euro. Soit $X$ la variable aléatoire égale au nombre de pièces fausses parmi les 1000.

\begin{enumerate}
    \item \question{ Quelle est la loi de la variable aléatoire $X$ ? Quelle est son espérance, son écart-type ? }
    \item \question{ En utilisant l'inégalité de Bienaymé-Tchebychev, majorer la probabilité que le nombre de pièces fausses soit supérieur à 100. }
    \item \question{ Justifier que $X$ peut être approchée par une variable suivant une loi normale de moyenne $\mu = 50$ et d’écart-type $\sigma = 7,5$. }
    \item \question{ À l’aide de cette approximation, estimer la probabilité que le nombre de pièces fausses soit compris entre 48 et 52. }
\end{enumerate}
}
