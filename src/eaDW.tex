\uuid{eaDW}
\chapitre{Probabilité discrète}
\niveau{L2}
\module{Probabilité et statistique}
\sousChapitre{Lois de distributions}
\titre{Calcul par approximation - Réseau de transmissions}
\theme{variables aléatoires discrètes, loi binomiale, loi normale}
\auteur{Maxime Nguyen}
\datecreate{2025-10-07}
\organisation{AMSCC}

\difficulte{3}

\contenu{
	
	\texte{ 
		Un poste de commandement dessert $\nombre{5000}$ terminaux de communication (véhicules, unités mobiles, postes radio). 
		À un instant donné, chaque terminal a une probabilité égale à $0.20$ d’être en communication active. 
		Les comportements des terminaux sont supposés indépendants les uns des autres. 
	}
	
	\begin{enumerate}
		\item \question{ 
			On note $X$ la variable aléatoire égale au nombre de terminaux simultanément en communication à un instant $t$. 
			Quelle est la loi de $X$ ? Quelle est son espérance ? Son écart-type ? 
		}
		\reponse{ 
			$X\sim \mathcal{B}(\nombre{5000},0.2)$, donc $\E(X)=\nombre{1000}$ et $\sigma^2(X)=800$.
		}
		
		\item \question{ 
			On pose $Y=\frac{X-\nombre{1000}}{\sqrt{800}}$. 
			Justifier précisément que l’on peut approcher la loi de $Y$ par la loi normale centrée réduite. 
		}
		\reponse{ 
			Puisque $n$ est grand ($n=5000$) et $p=0.2$ n’est ni trop proche de 0 ni de 1, 
			la loi binomiale peut être approchée par une loi normale : 
			\[ X \approx \mathcal{N}(\nombre{1000},\sigma=\sqrt{800}). \]
			En centrant et réduisant, on obtient que $Y\sim \mathcal{N}(0,1)$. 
		}
		
		\item \question{ 
			Le poste de commandement souhaite dimensionner son réseau pour que la probabilité de saturation (trop de communications simultanées) soit inférieure à $2.5$\%.  
			En utilisant l’approximation précédente, proposer une valeur approchée du nombre maximal de connexions que le système doit pouvoir gérer. 
		}
		\reponse{ 
			Soit $n$ le nombre maximal de connexions simultanées supportées.  
			On cherche $n$ tel que $\prob(X\geq n)\leq 0.025$, soit :
			\[ \prob\left(Y\geq \frac{n-\nombre{1000}}{\sqrt{800}}\right) \leq 0.025, \]
			c’est-à-dire encore :
			\[  \prob\left(Y\leq \frac{n-\nombre{1000}}{\sqrt{800}}\right) \geq 0.975. \]
			D’après la table de la loi normale centrée réduite, on a $z_{0.975}\simeq 1.96$, 
			d’où :
			\[ \frac{n-\nombre{1000}}{\sqrt{800}}\simeq 1.96, \quad \text{soit} \quad n\simeq 1055.44. \]
			Le réseau doit donc pouvoir gérer au minimum $1056$ connexions simultanées 
			pour que la probabilité de saturation soit inférieure à $2.5$\%. 
		}
	\end{enumerate}
}
