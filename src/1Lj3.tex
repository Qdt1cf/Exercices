\uuid{1Lj3}
\chapitre{Probabilité continue}
\sousChapitre{Densité de probabilité}
\titre{Probabilité d'un maximum}
\theme{variables aléatoires, densité}
\auteur{}
\datecreate{2023-09-13}
\organisation{AMSCC}
%
\contenu{

On considère $X_1$, $X_2$ et $X_3$ trois \vas indépendantes de même loi uniforme sur l'intervalle $[0,2a]$, dont la densité est donc
$$
f_{X_i}(t)=
\left\{\begin{array}{cl}
	\frac{1}{2a} & \text{si } 0\leq t < 2a, \\[1mm]
	0 & \text{sinon.}
\end{array}\right.
$$
On note $F_X$ la fonction de répartition de $X_i$ pour $i\in\{1,2,3\}$.
\begin{enumerate}
	\item \question{ Calculer $\E(X_i)$ et $\var(X_i)$, pour $i\in\{1,2,3\}$. On considère la variable aléatoire $Y=\frac{1}{3}(X_1+X_2+X_3)$. Exprimer $\E(Y)$ et $\var(Y)$ sans déterminer la loi de $Y$. }
	\reponse{ 
		Espérance de $X_i$:
		\[ \E(X_i)=\int_0^{2a}\frac{t}{2a}  \dx t = \left[ \frac{1}{4a}t^2 \right]_0^{2a} =\frac{4a}{(2a)^2}=a\]
		Variance de $X_i$:
		\begin{align*}
			&\E(X_i^2)=\int_0^{2a}\frac{t^2}{2a}  \dx t = \left[ \frac{t^3}{6a} \right]_0^{2a} =\frac{6a}{(2a)^3}=\frac{4}{3} a^2 \\
			&\var(X_i)=\E(X_i^2)-\E(X_i)^2=\frac{4}{3}a^2-a^2=\frac{1}{3}a^2
		\end{align*}
		Espérance de $Y$:
		\[\E(Y)=\frac{1}{3}(\E(X_1)+\E(X_2)+\E(X_3))=a\]
		Variance de $Y$: 
		\begin{align*}
			\var(Y)&=\frac{1}{9}\var(X_1+X_2+X_3) \\
			&= \frac{1}{9}(\var(X_1)+\var(X_2)+\var(X_3)) \quad \text{ car les va }X_i\text{ sont indépendantes} \\
			&=\frac{a^2}{9}
		\end{align*}
	}
	
	\item \question{ On pose $Z=\max(X_1,X_2,X_3)$. Justifier que la fonction de répartition $F_Z$ de $Z$ vérifie: $\displaystyle F_Z(t)=\prod_{i=1}^3 F_{X_i}(t)$. En déduire qu'une densité de $Z$ est :
	$$
	f_{Z}(t)=
	\left\{\begin{array}{cl}
		\frac{3}{(2a)^3}t^2 &\text{si } 0\leq t \leq 2a, \\[1mm]
		0& \text{sinon.}
	\end{array}\right.
	$$ }
	\reponse{ 
		Fonction de répartition de $Z$:
		\begin{align*}
			\forall t \in\R, \quad  F_Z(t)&=\p(Z\leq t)\\
			& =\p(\max(X_1,X_2,X_3) \leq t) \\
			&=\p(\{X_1\leq t\}\cap \{X_2\leq t\}\cap \{X_3\leq t\}) \\
			&=\p(X_1\leq t)\p(X_2\leq t)\p(X_3\leq t) \quad \text{ car les va }X_i\text{ sont indépendantes}\\
			&=\prod_{i=1}^3 F_{X_i}(t) 
		\end{align*}
		Densité de $Z$:
		\[ \forall t \in \R, \quad f_Z(t)=F_Z'(t)=3F_{X_1}(t)\times f_{X_1}(t) \]
		or
		\[ F_{X_1}(t)=\begin{cases} 0 & \text{ si } t<0 \\
			\frac{t}{2a} & \text{ si } t\in[0;2a] \\
			1 & \text{ si } t> 2a
		\end{cases}
		\]
		donc
		\begin{align*}
			f_Z(t)&= \begin{cases}
				0 & \text{ si } t< 0 \text{ ou } t>2a \\
				3\left(\frac{t}{2a}\right)^2\times \frac{1}{2a} &  \text{ si } t\in[0;2a]
			\end{cases} \\
			&= \begin{cases}
				0 & \text{ si } t< 0 \text{ ou } t>2a \\
				\frac{3}{(2a)^3}t^2 & \text{ si } t\in[0;2a]
			\end{cases}
		\end{align*}
	}
	
	\item \question{ Calculer $\E(Z)$ et $\var(Z)$. Soit la \va $T=\alpha Z$. Déterminer $\alpha$ de sorte que $\E(T)=\E(Y)$. }
	\reponse{ 
		Espérance de $Z$:
		\[ \E(Z)=\int_0^{2a} \frac{3}{(2a)^3}t^3  \dx t
		=\left[ \frac{3}{4}\frac{1}{(2a)^3}t^4 \right]_0^{2a}
		=\frac{3a}{2}
		\]
		Variance de $Z$:
		\begin{align*}
			&\E(Z^2)=\int_0^{2a} \frac{3}{(2a)^3}t^4  \dx t
			=\left[ \frac{3}{5}\frac{1}{(2a)^3}t^5 \right]_0^{2a}
			=\frac{12}{5}a^2 \\
			&\var(Z)=\E(Z^2)-\E(Z)^2
			=\frac{12}{5}a^2-\frac{9}{4}a^2=\frac{3}{20}a^2
		\end{align*}
		Détermination de $\alpha$:
		\begin{align*}
			\E(T)=\E(Y) \quad \Leftrightarrow \quad \alpha\E(Z)=a
			\quad \Leftrightarrow \quad \alpha \frac{3a}{2}=a
			\quad \Leftrightarrow \quad \alpha=\frac{2}{3}
		\end{align*}
	}
	
	\item\question{  Comparer $\var(T)$ et $\var(Y)$. }
	\reponse{ 
		On a
		$$\var(T)=\sigma^2\left(\frac{2}{3}Z\right)
		= \frac{4}{9}\var(Z)
		=\frac{4}{9}\times \frac{3}{20}a^2 
		=\frac{a^2}{15}.$$
		Comme $\displaystyle \var(Y)=\frac{a^2}{9}$, on a $\var(T) <  \var(Y)$.
	}
	
\end{enumerate}
}