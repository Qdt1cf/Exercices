\uuid{8}
\titre{Loi Bêta dilatée}
\theme{}
\auteur{}
\datecreate{2025-03-20}
\organisation{}

\contenu{

\texte{
On considère un échantillon i.i.d. $(X_i)_{1 \leq i \leq n}$ avec $X_1$ de densité $f_\theta(x) = \frac{2}{\theta^2} x \mathbf{1}_{[0,\theta]}(x)$, où $\theta > 0$ est un paramètre inconnu.
}

\begin{enumerate}
    \item \question{On pose $X^{(n)} = \max_{1 \leq i \leq n} X_i$. (a) Donner la densité de $X^{(n)}$. (b) Donner les deux premiers moments de $X^{(n)}$. En déduire sa variance. (c) Montrer que $X^{(n)}$ converge en probabilité vers $\theta$. (d) $X^{(n)}$ est-il un estimateur fortement consistant de $\theta$ ?}
    \indication{}
    \reponse{}
    \item \question{Étudier la convergence de $X_n$. En déduire un estimateur consistant de $\theta$.}
    \indication{}
    \reponse{}
    \item \question{Qui de $X^{(n)}$ ou $\frac{3}{2}X_n$ choisiriez-vous pour estimer $\theta$ ?}
    \indication{}
    \reponse{}
    \item \question{Construire un intervalle de confiance pour $\theta$ de niveau $(1-\alpha)$.}
    \indication{}
    \reponse{}
    \item \question{Construire un test de niveau $\alpha$ de l’hypothèse nulle $\theta = 1$ contre l’alternative $\theta \neq 1$. On observe $x^{(20)} = 0.85$, quelle est la p-valeur ? Rejette-t-on $H_0$ au niveau 5% ?}
    \indication{}
    \reponse{}
\end{enumerate}

}