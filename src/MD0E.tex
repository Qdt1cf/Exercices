\uuid{MD0E}
\chapitre{Probabilité discrète}
\niveau{L2}
\module{Probabilité et statistique}
\sousChapitre{Lois de distributions}
\titre{Loi de Poisson}
\theme{variables aléatoires discrètes, loi de Poisson}
\auteur{  }
\datecreate{2023-08-30}
\organisation{AMSCC}


\difficulte{2}
\contenu{
	\texte{ Une population de 80 personnes comporte en moyenne une personne mesurant plus de 1{,}90m. Soit une population $E_n$ de $n$ personnes et $X_n$ le nombre de personnes mesurant plus de 1{,}90m dans la population $E_n$. On admet que $X_n$ suit une loi de Poisson de paramètre $\lambda = \frac{n}{80}$. 
	}
	
	\question{  Quelle est la probabilité d'avoir au moins une personne  mesurant plus de 1{,}90m dans une population de 100 personnes ? 300 personnes ? }
	
	\reponse{  Dans une population de 100 personnes, il suffit de calculer $1-\prob(X=0) = 1- e^{-\frac{100}{80}} = 1-e^{-\frac{5}{4}} \approx 0.71$. 
		
		Dans une population de 300 personnes, il suffit de calculer $1-\prob(X=0) = 1- e^{-\frac{300}{80}}  \approx 0.98$. 
	}
	
}
