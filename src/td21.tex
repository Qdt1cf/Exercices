\uuid{td21}
\titre{Comparaison d’estimateurs}
\theme{}
\auteur{Sorbonne Université, Master 1 MU4MA015, Statistique, 2024-2025\\ Cours : A. Guyader\\ TD : C. Dion-Blanc, A. Godichon, A. Guyader}
\datecreate{2025-03-20}
\organisation{}

\contenu{

\texte{
Soit un échantillon i.i.d. $(X_i)_{1 \leq i \leq n}$ tel que $E(X_1) = m$ et $Var(X_1) = \sigma^2$. On suppose $\sigma^2$ connue, $m \in \mathbb{R}$ étant ici le paramètre inconnu. On propose deux estimateurs de $m$ : 
$$\overline{X}_n=\frac{1}{n}\sum_{i=1}^{n}X_i,\,\,\text{et}\,\,Z_n = \frac{1}{2}(X_n + X_{n-1}).$$
}

\begin{enumerate}
    \item \question{Montrer que $\overline{X}_n$ et $Z_n$ sont sans biais.}
    \indication{}
    \reponse{On a :
\begin{equation}
    \mathbb{E}[\overline{X}_n] = \frac{1}{n} \sum_{i=1}^{n} \mathbb{E}[X_i] = m
\end{equation}
\begin{equation}
    \mathbb{E}[Z_n] = \frac{1}{2} \left( \mathbb{E}[X_n] + \mathbb{E}[X_{n-1}] \right) = m
\end{equation}
Ainsi, les deux estimateurs sont sans biais pour l'estimation de $m$.}
    \item \question{Qui de $\overline{X}_n$ ou $Z_n$ choisiriez-vous pour approcher $m$ ?}
    \indication{}
    \reponse{L'écart quadratique moyen d'un estimateur $\hat{m}$ de $m$ est donné par :
\begin{equation}
    EQM(\hat{m}) = (\mathbb{E}[\hat{m}] - m)^2 + \text{Var}(\hat{m}).
\end{equation}
On compare les EQM des deux estimateurs. On a :
\begin{equation}
    EQM(\overline{X}_n) = 0 + \text{Var}(\overline{X}_n) = \frac{\sigma^2}{n}
\end{equation}
\begin{equation}
    EQM(Z_n) = 0 + \frac{1}{4} \text{Var}(X_n + X_{n-1}) = \frac{\sigma^2}{2}.
\end{equation}
Pour tout $n \geq 3$, on a $EQM(\overline{X}_n) < EQM(Z_n)$, donc $\overline{X}_n$ est préférable à $Z_n$ au sens du risque quadratique.}
  %  \item \question{Que pensez-vous de l’estimateur $W_n = 0$ ?}
  % \indication{}
%    \reponse{}
\end{enumerate}

}
