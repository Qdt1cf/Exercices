\chapitre{Probabilité continue}
\sousChapitre{Loi normale}
\uuid{1fDE}
\titre{Approximation par une loi normale}
\theme{probabilités conditionnelles, variables aléatoires, loi binomiale, loi normale}
\auteur{Maxime Nguyen}
\datecreate{2023-09-18}
\organisation{AMSCC}

\contenu{

\texte{ On effectue un contrôle sur des pièces de un euro dont une proportion $p=0.05$ est fausse et sur des pièces de deux euros dont une proportion $p'=0.02$ est fausse. On considère un lot de $500$ pièces dont $150$ pièces de un euro et $350$ pièces de deux euros. }
\begin{enumerate}
	\item \question{ On prend une pièce au hasard dans ce lot: quelle est la probabilité qu'elle soit fausse ? }
	\reponse{ 
		On peut utiliser un arbre de probabilité pour modéliser la situation. En notant $F$ l'événement ``obtenir une pièce fausse'' et $A$ (resp. $B$) l'événement ``obtenir une pièce de un euro (resp. deux euros)'', on a
		\[\prob(F)=\prob(F\cap A)+\prob(F\cap B)=\prob(A)\prob(F|A)+\prob(B)\prob(F|B)=\frac{150}{500}\times 0.05+\frac{350}{500}\times 0.02=0.029.\]
		On a environ $2.9$\% d'avoir une pièce fausse.
	}
	
	\item\question{  Sachant que cette pièce est fausse, quelle est la probabilité qu'elle soit de un euro ? }
	\reponse{ 
		On calcule :
		\[ \prob(A|F)=\frac{\prob(A\cap F)}{\prob(F)}=\frac{\prob(A)\prob(F|A)}{\prob(F)}=\frac{\frac{150}{500}\times 0.05}{0.029}=0.5172.\]
	}
	
	\item \question{ On contrôle à présent un lot de $\nombre{1000}$ pièces de un euro. Soit $X$ la variable aléatoire égale au nombre de pièces fausses parmi les $\nombre{1000}$. \\
	Quelle est la loi de $X$ ? Quelle est son espérance ? Son écart-type ? \\
	En approchant cette loi par celle d'une loi normale adaptée, donner une approximation de la probabilité pour que $X$ soit compris entre $48$ et $52$. }
	\reponse{ 
		On a: $X\sim \mathcal{B}(\nombre{1000},0.05)$, $\E(X)=\nombre{1000}\times 0.05=50$ et $\sigma^2(X)=\nombre{1000}\times 0.05\times 0.95=47.5$. \\
		La \va $X$ peut être approchée (avec correction de continuité) par la \va $Y$ de loi $\mathcal{N}(50,\sigma^2=47.5)$. Ainsi, on a
		\begin{align*}
		\prob(48\leq X\leq 52)
		&= \prob(47.5\leq X\leq 52.5) \\
		&\simeq \prob(47.5\leq Y\leq 52.5) \\
		& \simeq \prob(-0.36 \leq Z \leq 0.36) \quad \text{ avec } Z=\frac{Y-50}{\sqrt{47.5}} \sim \mathcal{N}(0,1) \\
		&\simeq 2\prob(Z\leq 0.36)-1 \\
		& \simeq 2\times 0.6406 - 1\\
		1\simeq 0.2812
		\end{align*}
	}
	
\end{enumerate}
}