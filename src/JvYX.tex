\uuid{JvYX}
\chapitre{Probabilité continue}
\niveau{L2}
\module{Probabilité et statistique}
\sousChapitre{Densité de probabilité}
\titre{Changement de variable aléatoire}
\theme{variables aléatoires à densité}
\auteur{Maxime Nguyen}
\datecreate{2025-09-15}
\organisation{AMSCC}
%
\difficulte{2}
\contenu{
	
	\texte{ Soit $X$ une variable aléatoire réelle suivant une loi uniforme sur $[0;1]$. Soit la variable aléatoire $Y=-2\ln(1-X)$. On note $F_X$, respectivement $F_Y$, la fonction de répartition de la variable $X$, respectivement $Y$. }
	
	\begin{enumerate}
		\item \question{ Pour tout $t \in \mathbb{R}$, exprimer $F_Y(t)$ en fonction de $F_X$. }
		\reponse{ Soit $t\in\mathbb{R}$. On a
			\begin{align*}
				F_Y(t) &= \prob(Y\leq t) \\
				&= \prob(-2\ln(1-X)\leq t) \\
				&= \prob(\ln(1-X)\geq -t/2) \\
				&= \prob(1-X\geq e^{-t/2}) \\
				&= \prob(X\leq 1-e^{-t/2}) \\
				&= F_X(1-e^{-t/2}).
		\end{align*} }
		\item \question{ En déduire la loi suivie par $Y=-2\ln(1-X)$ ? }
		\reponse{ 
			Or $X\sim \mathcal{U}([0;1])$ donc $F_X(x)=\begin{cases} 0 & \text{ si } x<0 \\ x & \text{ si } x\in[0;1[ \\ 1 & \text{ si } x\geq 1 \end{cases}$.
			
			De plus, si $t\geq 0$, $1-e^{-t/2} \in [0;1[$ et si $t\leq 0$, $1-e^{-t/2}<0$.
			
			Par conséquent,
			\[ F_Y(t)=\begin{cases}
				0 & \text{ si } t<0 \\
				1-e^{-t/2} & \text{ si } t\geq 0
			\end{cases}
			\]
			ce qui nous permet de reconnaître la fonction de répartition de la loi exponentielle de paramètre $1/2$ donc $Y\sim \mathcal{E}(1/2)$. }
\end{enumerate}}
