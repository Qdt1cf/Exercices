\uuid{2Aw2}
\chapitre{Statistique}
\niveau{L2}
\module{Probabilité et statistique}
\sousChapitre{Autre}
\titre{Approximation d'une loi binomiale par une loi normale.}
\theme{statistiques, loi binomiale, loi normale}

\auteur{}
\datecreate{2022-08-25}
\organisation{AMSCC}
\contenu{

\texte{Un restaurant universitaire qui est visité par 600 étudiants pour le repas de midi offre un choix de deux plats principaux: risotto ou quiche. Dans le passé, on a observé que 40\% des étudiants prennent du risotto.}

\question{Combien de plats de risotto faut-il prévoir pour que la probabilité qu'il en manque soit inférieure à 10\% ?}

\reponse{Soit $X$ le nombre d'étudiants souhaitant un plat de risotto. D'après les données de l'énoncé, $X$ suit une loi binomiale $\mathcal{B}(600,0.4)$. 

On cherche donc un entier $n$ tel que $\PP(X \geq n) \leq 0.1$. Pour ce faire, on approche la loi de $X$ par une loi normale. L'effectif $n$ est supérieur à $30$ donc d'après le théorème central limite, $X$ suit approximativement une loi normale de paramètres $\mu=600 \times 0.4 = 240$ et de variance $\sigma^2 = 600 \times 0.4 \times 0.6 = 144 = 12^2$. 

Ainsi, 
\begin{align*}
\PP(X \geq n) \leq 0.1 &\iff \PP\left(\frac{X-240}{12} \geq \frac{n-240}{12}\right)  \leq 0.1 \\
&\iff \PP\left(\frac{X-240}{12} \leq \frac{n-240}{12}\right)  \geq 0.9
&\iff \frac{n-240}{12} \geq 1.28 \\
&\iff n \geq 12 \times 1.28 + 240 = 255.36 \\
&\iff n \geq 256
\end{align*}

Il faut donc prévoir au moins 256 plats de risotto pour que le risque qu'il en manque soit inférieur à 10\%.
}
}
