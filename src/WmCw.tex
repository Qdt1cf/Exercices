\uuid{WmCw}
\chapitre{Fonction de plusieurs variables}
\sousChapitre{Dérivée partielle}
\titre{Calcul de dérivées partielles}
\theme{calcul différentiel}
\auteur{}
\datecreate{2023-03-09}
\organisation{AMSCC}
\contenu{

Pour chacune des fonctions suivantes : préciser l'ensemble de définition, puis calculer  $\dpa{f}{x}$, $\dpa{f}{y}$, $\dpsp{f}{x}$, $\dpsp{f}{y}$ et $\dpsm{f}{x}{y}$.
\begin{enumerate}
\item \question{ $f(x,y) = x^2 -6xy -6y^2 + 2x +24y$ }
\reponse{La fonction $f$ est polynomiale en $x$ et $y$, elle définie sur $\R^2$. Les dérivées partielles existent également pour tout $(x,y) \in \R^2$ :
	\begin{align*}
\frac{\partial f}{\partial x}(x,y) &= 2x-6y+2 \\
\frac{\partial f}{\partial y}(x,y) &= -6x-12y+24 \\
\frac{\partial^2 f}{\partial x^2}(x,y) &= 2 \\
\frac{\partial^2 f}{\partial y^2}(x,y) &= -12 \\
\frac{\partial^2 f}{\partial x \partial y}(x,y) &= \frac{\partial^2 f}{\partial y \partial x}(x,y)= -6 
\end{align*}}
\item \question{ $f(x, y) = x^2 + 2y^2 - \frac{x^3}{y}$ }
\reponse{La fonction $f$ est  définie sur $\{ (x,y) \in \R^2 \mid y\neq 0 \}$. Les dérivées partielles existent également pour tout $(x,y) \in \R^2$ telles que $y \neq 0$ :
	\begin{align*}
		\frac{\partial f}{\partial x}(x,y) &= 2x-\frac{3x^2}{y} \\
		\frac{\partial f}{\partial y}(x,y) &= 4y+\frac{x^3}{y^2} \\
		\frac{\partial^2 f}{\partial x^2}(x,y) &= 2-\frac{6x}{y} \\
		\frac{\partial^2 f}{\partial y^2}(x,y) &= 4-2\frac{x^3}{y^3} \\
		\frac{\partial^2 f}{\partial x \partial y}(x,y) &= \frac{\partial^2 f}{\partial y \partial x}(x,y)= \frac{3x^2}{y^2}
\end{align*}}
\item \question{ $f(x, y) = \exp (2x^2+xy+7x+y^2)$ }
\reponse{La fonction $f$ est une composée d'une exponentielle avec une fonction polynomiale en $x$ et $y$, elle définie sur $\R^2$. Les dérivées partielles existent également pour tout $(x,y) \in \R^2$ :
	\begin{align*}
		\frac{\partial f}{\partial x}(x,y) &= (4x+y+7)\exp (2x^2+xy+7x+y^2) \\
		\frac{\partial f}{\partial y}(x,y) &= (x+2y)\exp (2x^2+xy+7x+y^2) \\
		\frac{\partial^2 f}{\partial x^2}(x,y) &= (4+(4x+y+7)^2)\exp (2x^2+xy+7x+y^2) \\
		\frac{\partial^2 f}{\partial y^2}(x,y) &= (2+(x+2y)^2)\exp (2x^2+xy+7x+y^2) \\
		\frac{\partial^2 f}{\partial x \partial y}(x,y) &= \frac{\partial^2 f}{\partial y \partial x}(x,y)= (1+(x+2y)(4x+y+7))\exp (2x^2+xy+7x+y^2)
\end{align*}}
\item \question{ $f(x, y) = \sin(xy)$ }
\reponse{La fonction $f$ est une composée d'un cosinus avec une fonction polynomiale en $x$ et $y$, elle définie sur $\R^2$. Les dérivées partielles existent également pour tout $(x,y) \in \R^2$ et on observe que les rôles sont symétriques en $x$ et $y$ :
	\begin{align*}
		\frac{\partial f}{\partial x}(x,y) &= y\cos(xy) \\
		\frac{\partial f}{\partial y}(x,y) &= x\cos(xy) \\
		\frac{\partial^2 f}{\partial x^2}(x,y) &= -y^2\sin(xy) \\
		\frac{\partial^2 f}{\partial y^2}(x,y) &= -x^2\sin(xy) \\
		\frac{\partial^2 f}{\partial x \partial y}(x,y) &= \frac{\partial^2 f}{\partial y \partial x}(x,y)= \cos(xy)-xy\sin(xy) 
\end{align*}}
\item \question{ $f(x, y) = \ln(x + y)$ }
\reponse{La fonction $f$ est une composée d'un $\ln$ avec une fonction polynomiale en $x$ et $y$, elle est  définie sur le demi plan $\{ (x,y) \in \R^2 \mid x+y>0 \}$. Les dérivées partielles existent également pour tout $(x,y) \in \R^2$ telles que $x+y > 0$ et on observe que les rôles sont symétriques en $x$ et $y$ :
	\begin{align*}
		\frac{\partial f}{\partial x}(x,y) &= \frac{1}{x+y} \\
		\frac{\partial f}{\partial y}(x,y) &= \frac{1}{x+y} \\
		\frac{\partial^2 f}{\partial x^2}(x,y) &= -\frac{1}{(x+y)^2} \\
		\frac{\partial^2 f}{\partial y^2}(x,y) &= -\frac{1}{(x+y)^2} \\
		\frac{\partial^2 f}{\partial x \partial y}(x,y) &= \frac{\partial^2 f}{\partial y \partial x}(x,y)= -\frac{1}{(x+y)^2 }
\end{align*}}
%\item $f(x, y) = cos(x^3y^2 + 5x + 7y - 1)$
\end{enumerate}}
