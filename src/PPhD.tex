\uuid{PPhD}
\chapitre{Probabilité continue}
\niveau{L2}
\module{Probabilité et statistique}
\sousChapitre{Densité de probabilité}
\titre{Etude d'une variable aléatoire à densité}
\theme{variables aléatoires à densité}
\auteur{Maxime Nguyen}
\datecreate{2023-09-18}
\organisation{AMSCC}

\difficulte{}
\contenu{

\texte{ Soit $X$ une variable aléatoire absolument continue de densité $f$ définie sur $\R$ par :
$$f_X \colon x \mapsto \begin{cases}
    kx^2 & \text{si } x \in [-1;1] \\
    0 & \text{sinon}
\end{cases}$$ }

\begin{enumerate}
    \item \question{ Déterminer la valeur de $k$. }
    \reponse{ Pour que $f_X$ soit une densité, il faut que $\int_{\R} f_X(x) dx = 1$. On a donc :
    \begin{align*}
        \int_{\R} f_X(x) dx &= \int_{-1}^1 kx^2 dx \\
        &= \left[ \frac{k}{3}x^3 \right]_{-1}^1 \\
        &= \frac{2k}{3} = 1
    \end{align*}
    Donc $k = \frac{3}{2}$. }
    \item \question{ Déterminer la fonction de répartition $F_X$ de $X$. }
    \reponse{
    \begin{align*}
        F_X \colon t \mapsto \begin{cases}
            0 & \text{si } t < -1 \\
            \frac{3}{2} \int_{-1}^t x^2 dx & \text{si } t \in [-1;1] \\
            1 & \text{si } t > 1
        \end{cases}
    \end{align*}
    donc :
    \begin{align*}
        F_X \colon x \mapsto \begin{cases}
            0 & \text{si } t < -1 \\
            \frac{1}{2}t^3 + \frac{1}{2} & \text{si } t \in [-1;1] \\
            1 & \text{si } t > 1
        \end{cases}
    \end{align*}
    }
    \item \question{ Calculer la probabilité conditionnelle $\prob(X \leq \frac{1}{2} \mid X > 0)$. }
    \reponse{
D'après la formule des probabilités conditionnelles, on a :
\begin{align*}
    \prob(X \leq \frac{1}{2} \mid X > 0) &= \frac{\prob(X \leq \frac{1}{2} \cap X > 0)}{\prob(X > 0)} \\
    &= \frac{\prob(0 < X \leq \frac{1}{2})}{\prob(X > 0)}
\end{align*}
Or $\prob(X>0) = 1 - \prob(X \leq 0) = 1 - F_X(0) = 1 - \frac{1}{2} = \frac{1}{2}$ et : 
\begin{align*}
    \prob(0 < X \leq \frac{1}{2}) &= F_X(\frac{1}{2}) - F_X(0) \\
    &= \frac{1}{2}(\frac{1}{2})^3 + \frac{1}{2} - \frac{1}{2} \\
    &= \frac{1}{16} 
\end{align*}
Donc $\prob(X \leq \frac{1}{2} \mid X > 0) = \frac{\frac{1}{16}}{\frac{1}{2}} = \frac{1}{8}$. }
    \item \question{ Calculer $\E(X)$ et $\var(X)$. }
    \reponse{
    \begin{align*}
        \E(X) &= \int_{\R} xf_X(x) dx \\
        &= \int_{-1}^1 x \frac{3}{2}x^2 dx \\
        &= \left[ \frac{3}{8}x^4 \right]_{-1}^1 \\
        &= 0
    \end{align*}
    D'après le théorème de transfert, on a : 
    \begin{align*}
        \E(X^2) &= \int_{\R} x^2 f_X(x) dx \\
        &= \int_{-1}^1 x^2 \frac{3}{2}x^2 dx \\
        &= \left[ \frac{3}{10}x^5 \right]_{-1}^1 \\
        &= \frac{3}{5}
    \end{align*}
    Donc $\var(X) = \E(X^2) - \E(X)^2 = \frac{3}{5}$.
    }
    \item \question{ Soit $Y = X^2$. Déterminer la fonction de répartition $F_Y$ de $Y$ et en déduire sa densité. }
    \reponse{
    Par définition, si $t \in \R$ alors $F_Y(t) = \prob(Y \leq t) = \prob(X^2 \leq t)$. Donc : 
    $$F_Y(t) = \begin{cases}
        0 & \text{si } t < 0 \\
        \prob(-\sqrt{t} \leq X \leq \sqrt{t}) & \text{si } t \geq 0
    \end{cases}$$
    Or si $t \geq 0$, $\prob(-\sqrt{t} \leq X \leq \sqrt{t}) = F_X(\sqrt{t}) - F_X(-\sqrt{t}) = \begin{cases}
        1 & \text{si } t \geq 1 \\
        \left[ \frac{1}{2} x^3 \right]_{-\sqrt{t}}^{\sqrt{t}} = t\sqrt{t} & \text{si } t \in [0;1]
    \end{cases}$.

    En définitive, $$F_Y(t) = \begin{cases}
        0 & \text{si } t < 0 \\
        t\sqrt{t} & \text{si } t \in [0;1] \\
        1 & \text{si } t \geq 1
    \end{cases}$$
Par dérivation de la fonction de répartition, on obtient la densité de $Y$ : 
$$f_Y(t) = \begin{cases}
    \frac{3}{2}\sqrt{t} & \text{si } t \in [0;1] \\
    0 & \text{sinon}
\end{cases}$$
    }
\end{enumerate}
}