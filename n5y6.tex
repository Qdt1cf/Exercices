\uuid{n5y6}
\chapitre{Probabilité continue}
\sousChapitre{Loi normale}

\titre{Distribution normale}
\theme{loi normale}
\auteur{L'Haridon E.}
\datecreate{2025-10-07}
\organisation{AMSCC}
\difficulte{3}
\contenu{

\texte{
Une base militaire doit planifier la distribution de rations alimentaires pour ses soldats. Le
nombre de rations nécessaires chaque jour est une variable aléatoire qui suit une loi normale de
moyenne $m$ et d'écart-type $\sigma$. Les données historiques montrent que :
\begin{itemize}
\item La probabilité que le nombre de rations nécessaires soit supérieur ou égal à $1522$ est de $0,33$.
\item La probabilité que le nombre de rations nécessaires soit inférieur ou égalà $1598$ est de $0,975$.
\end{itemize}
}
\question{Quel est le nombre minimal de rations qui doivent être préparées chaque jour pour avoir $99$ chances sur $100$ de satisfaire la demande?}
\reponse{Commençons par déterminer $m$ et $\sigma$ grâce aux données de l'énoncé:
	\begin{align*}
		&\p(X\geq 1522)=0.33 \ \Leftrightarrow \ \p\left(\frac{X-m}{\sigma}\leq \frac{1522-m}{\sigma}\right)=0.67
		\ \Leftrightarrow \ \frac{1522-m}{\sigma}=0.44 \\
		&\p(X\leq 1598)=0.975 \ \Leftrightarrow \ \p\left(\frac{X-m}{\sigma}\leq \frac{1598-m}{\sigma}\right)=0.975
		\ \Leftrightarrow \ \frac{1598-m}{\sigma}=1.96
	\end{align*}
	Il vient ainsi
	\begin{align*}
		\begin{cases}
			1522-m=0.44\sigma \\
			1598-m=1.96\sigma
		\end{cases}
	\end{align*}
	soit $\sigma=50$ et $m=1500$. Donc $X\sim \mathcal{N}(1500,\sigma=50)$.
	\vspace{1em}
	
	On note $n$ le nombre de rations à prévoir pour satisfaire $99$\% des demandes. Par définition: $\p(X\leq n)=0.99$, ce qui donne
	\[ \p\left(\frac{X-1500}{50}\leq \frac{n-1500}{50}\right)= 0.99\]
	soit $\frac{n-1500}{50}=2.33$ et finalement $n=1617$.
	Il faut donc prévoir $1617$ rations au minimum pour satisfaire la demande dans $99$\% des cas.
	
}

}

