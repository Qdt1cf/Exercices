%print
\newboolean{solution}
\newboolean{isuuid}
\newboolean{link}
\newboolean{isindication}

\newcounter{num}

\definecolor{solutioncolor}{RGB}{0,100,0}
\definecolor{titrecolor}{RGB}{36,76,146}
\definecolor{indicationcolor}{RGB}{255,193,7} % Amélioration : couleur dédiée pour les indications

%liens vers des ressources extérieures :
\newcommand{\pathcorrige}{https://openyourmath.org/exercice?uuid=}
\newcommand{\pathnotebook}{https://github.com/smaxx73/Exercices/blob/main/notebook/}
\newcommand{\pathexercice}{https://raw.githubusercontent.com/smaxx73/Exercices/main/pdf/pdf_sansol/}

% Amélioration : Ajout de commandes configurables pour les chemins
\newcommand{\setpathcorrige}[1]{\renewcommand{\pathcorrige}{#1}}
\newcommand{\setpathnotebook}[1]{\renewcommand{\pathnotebook}{#1}}
\newcommand{\setpathexercice}[1]{\renewcommand{\pathexercice}{#1}}

\newtcolorbox{marker}[1][]{enhanced,hbox,before skip=2mm,after skip=3mm,fontupper=\small\sffamily,%
	boxrule=0.4pt,left=1mm,right=4mm,top=1pt,bottom=1pt,colback=blue!15,colframe=yellow!20!black,%
	sharp corners,rounded corners=southeast,arc is angular,arc=3mm,%
	underlay={%
		\path[fill=tcbcolback!80!black] ([yshift=3mm]interior.south east)--++(-0.4,-0.1)--++(0.1,-0.2);
		\path[draw=tcbcolframe,shorten <=-0.05mm,shorten >=-0.05mm] ([yshift=3mm]interior.south east)--++(-0.4,-0.1)--++(0.1,-0.2);
	},%
	drop fuzzy shadow,#1
}

% Amélioration : Boîte de solution avec titre personnalisable
\newtcolorbox{solutionbox}[1][Solution]{
	enhanced,
	breakable,
	colback=solutioncolor!30!white,
	colframe=solutioncolor,
	boxsep=5pt,
	frame hidden,
	interior style={top color=solutioncolor!15!white, bottom color=solutioncolor!20!white},
	overlay={
		\node[anchor=north west, inner sep=2pt] at (frame.north west) {
			\tikz{\node[draw, fill=solutioncolor, inner sep=1pt, rounded corners=2pt] {\textcolor{white}{\tiny\bfseries #1}};};
		};
	},
	% Amélioration : Ajout d'un petit espace en haut pour le titre
	top=8pt
}

% Amélioration : Boîte d'indication améliorée avec couleur dédiée
\newtcolorbox{indicationbox}[1][Indication]{
	enhanced,
	breakable,
	colback=indicationcolor!30!white,
	colframe=indicationcolor!80!black,
	boxsep=5pt,
	frame hidden,
	interior style={top color=indicationcolor!15!white, bottom color=indicationcolor!20!white},
	overlay={
		\node[anchor=north west, inner sep=2pt] at (frame.north west) {
			\tikz{\node[draw, fill=indicationcolor!80!black, inner sep=1pt, rounded corners=2pt] {\textcolor{white}{\tiny\bfseries #1}};};
		};
	},
	top=8pt
}

% Amélioration : Boîte de titre avec gestion d'erreurs
\newtcolorbox{titrebox}[3][]{
	enhanced,
	breakable,
	boxrule=0pt,
	colframe=white,
	colback=white,
	sharp corners,
	top=20pt,
	borderline north={0pt}{0pt}{white!100!black},
	overlay unbroken and first={
		\node[anchor=north west, inner xsep=-2pt, inner ysep=-7pt, rounded corners] at (frame.north west) {
			\tikz{\node[draw, fill=titrecolor, inner sep=5pt, rounded corners=3pt] {\small\textbf{\sffamily{\textcolor{white}{Ex #1 - #2}}}};}
		};
		\ifthenelse{\boolean{isuuid}}{
			\node[anchor=north east, inner ysep=-7pt] at (frame.north east) {
				\tikz{
					\node[fill=gray!20, rounded corners=2pt, inner sep=2pt] {
						\ifthenelse{\boolean{link}}{
							\href{\pathcorrige#3}{\texttt{\footnotesize #3}}
						}{
							\texttt{\footnotesize #3}
						}
					};
				}
			};
		}{}
	},
}

% Amélioration : Variables globales avec valeurs par défaut
\newcommand{\titre}[1]{%
	\def\TitreExo{#1}
}
\newcommand{\contenu}[1]{%
	\def\Contenu{#1}
}

% Amélioration : Définition de listes d'exercices courantes
\newcommand{\TD}{}  % Liste vide par défaut, à redéfinir selon les besoins
\newcommand{\setTD}[1]{\renewcommand{\TD}{#1}}  % Pour définir la liste TD

% Amélioration : Commandes plus explicites
\newcommand{\code}[1]{\texttt{#1}}
\newcommand{\texte}[1]{#1}

% Amélioration : Question avec numérotation optionnelle
\newcounter{questionnum}
\newcommand{\question}[2][]{%
	\ifx\relax#1\relax
		#2
	\else
		\stepcounter{questionnum}
		\textbf{Question \thequestionnum :} #2
	\fi
}

% Amélioration : Réponse et indication avec titres personnalisables
\newcommand{\reponse}[2][Solution]{
	\ifthenelse{\boolean{solution}} 
	{%
		\begin{solutionbox}[#1]
			\footnotesize #2
		\end{solutionbox}
	}{}
} 

\newcommand{\indication}[2][Indication]{
	\ifthenelse{\boolean{isindication}} 
	{%
		\begin{indicationbox}[#1]
			\footnotesize #2
		\end{indicationbox}
	}{}
}

\newcommand{\nextexo}{
	\addtocounter{num}{1}
	\vspace{1em}
}

% Amélioration : Gestion d'erreur pour les fichiers manquants
\newcommand{\insertexo}[6]{%contenu, solution, uuid, lien solution, numerotation, indication
	\noindent
	% Vérification de l'existence du fichier
	\IfFileExists{\path src/#1}{
		\input{\path src/#1}
	}{
		\PackageWarning{exercices}{Fichier #1 introuvable}
		\def\TitreExo{Exercice non trouvé}
		\def\Contenu{Le fichier d'exercice #1 n'a pas pu être chargé.}
	}
	
	\setboolean{solution}{#2}
	\setboolean{isuuid}{#3}
	\setboolean{link}{#4}
	\setboolean{isindication}{#6}
	
	\begin{titrebox}[#5]{\TitreExo}{#1}
		\Contenu
	\end{titrebox}	
}

% Amélioration : Version simplifiée de insertexo
\newcommand{\insertexosimple}[2][true]{%exercice, avec solution par défaut
	\nextexo
	\insertexo{#2}{#1}{false}{false}{\thenum}{false}
}

% Amélioration : Gestion plus flexible des listes
\newcommand{\listexo}[1]{%liste
	\foreach \ex in #1 {
		\nextexo
		\insertexo{\ex}{\ifboolean{solution}{true}{false}}{\ifboolean{isuuid}{true}{false}}{\ifboolean{link}{true}{false}}{\thenum}{\ifboolean{isindication}{true}{false}}
	}
}

% Amélioration : Lien notebook plus robuste
\newcommand{\insertnotebook}[2][Notebook]{%titre, fichier
	\href{\pathnotebook#2.ipynb}{#1} 
}

% Amélioration : Gestion des colonnes simplifiée
\newcommand{\debutcolonnes}[3][2]{%[nb_colonnes_defaut], nb_sans_sol, nb_avec_sol
	\setboolean{solution}{\solution}
	\ifthenelse{\boolean{solution}} {
		\ifnumcomp{#3}{=}{1}{}{\begin{multicols}{#3}}
	}{
		\ifnumcomp{#2}{=}{1}{}{\begin{multicols}{#2}}
	}
}

\newcommand{\fincolonnes}[3][2]{%[nb_colonnes_defaut], nb_sans_sol, nb_avec_sol
	\setboolean{solution}{\solution}
	\ifthenelse{\boolean{solution}} {
		\ifnumcomp{#3}{=}{1}{}{\end{multicols}}
	}{
		\ifnumcomp{#2}{=}{1}{}{\end{multicols}}
	}
}

\newcounter{qrcodenum}

% Amélioration : QR codes avec gestion d'erreur et mise en page améliorée
\newcommand{\listeqrcode}[3][4]{%[nb_par_ligne], liste, compteur_initial
	\setcounter{qrcodenum}{#3}
	\noindent
	\newcounter{qrcount}
	\setcounter{qrcount}{0}
	
	\foreach \exo in #2{
		\stepcounter{qrcount}
		\begin{minipage}{\dimexpr\textwidth/#1-2\fboxsep\relax}
			\centering
			\small
			\href{\pathcorrige\exo}{\textbf{Ex \theqrcodenum}}\\
			\texttt{\footnotesize \exo}\\[0.5em]
			\IfFileExists{\pathcorrige\exo}{
				\qrcode[height=2cm]{\pathcorrige\exo}
			}{
				\fbox{\parbox{2cm}{\centering QR\\indisponible}}
			}
		\end{minipage}%
		\ifnumcomp{\value{qrcount}}{=}{#1}{
			\setcounter{qrcount}{0}\\[1em]
		}{\hfill}
		\stepcounter{qrcodenum}
	}
}

% Amélioration : Commandes de configuration globale
\newcommand{\setexercicestyle}[4]{%solution, uuid, link, indication
	\setboolean{solution}{#1}
	\setboolean{isuuid}{#2}
	\setboolean{link}{#3}
	\setboolean{isindication}{#4}
}

% Amélioration : Raccourcis utiles
\newcommand{\exosansol}{\setexercicestyle{false}{false}{false}{false}}
\newcommand{\exoavecsol}{\setexercicestyle{true}{false}{false}{false}}
\newcommand{\exocomplet}{\setexercicestyle{true}{true}{true}{true}}
