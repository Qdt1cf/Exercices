\uuid{hCfF}
\titre{Lignes de niveau d'un polynôme}
\theme{calcul différentiel}
\auteur{}
\organisation{AMSCC}
\contenu{

\texte{ 	Soit $f\ : \R^{2} \to \R$ définie par $f(x,y) = x^{2}-4x +y^{2}+6y$. }

\begin{enumerate}
	
	\item \question{ Trouver une écriture de la forme
	
	$$ f(x,y) = (x+a)^{2} + (y+b)^{2} + c $$ 
	
	où $a,b,c$ sont trois réels que l'on explicitera. }
	\reponse{ Obtenir cette écriture est un des classiques de manipulation du trinôme du second degré~: on écrit $x^2-4x = x^2 - 2 \cdot 2x + 2^2 - 2^2 = (x-2)^2 - 4$. De même $y^{2}+6y = (y+3)^2-9$. Ainsi $f(x,y) = (x-2)^2 + (y+3)^2 - 13$ ce qui donne la forme voulue avec $a = 2, b=-3, c=-13$
	}
	
	\item \question{ En déduire une équation et la nature de la ligne de niveau $k$ de $f$, pour $k\in \R$ (on peut distinguer selon la valeur de $k$). }
	\reponse{La ligne de niveau $k$ de $g$ a pour équation $f(x,y) = k$, donc d'après la question 1 
		$$ (x-2)^2 + (y+3)^2 = k+13 $$. 
		
		Il s'agit de l'équation d'un cercle de centre $(2, -3)$ à condition que $k+13 \geq 0$. Sinon c'est l'ensemble vide. Si $k+13 =0$, le cercle est de rayon $0$ et est donc réduit à un point (son centre), et si $k+13 >0$ c'est un cercle de rayon $\sqrt{k+13}$ de centre $(2,-3)$.  
	}
	
\end{enumerate}}
