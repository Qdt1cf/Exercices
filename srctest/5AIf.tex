\uuid{5AIf}
\titre{ Optimisation d'un stock }
\theme {probabilités}
\auteur{ }
\organisation{ AMSCC }

\contenu{
    \texte{ La demande d'un produit au cours d'une saison suit une loi normale de moyenne \nombre{5000} et d'écart-type \nombre{1000}. }
    
    \question{ Quel niveau de stock doit-on prévoir pour satisfaire la demande avec une probabilité de \nombre{0,90} ? }

    \reponse{ 
        On modélise la demande par une variable aléatoire $X$ qui suit une loi normale $\mathcal{N}(\mu=5000,\sigma=1000)$. On cherche $t$ tel que $\prob(X \leq t) = 0,90$. On a $\prob(X \leq t) = \Phi\left(\frac{t-\mu}{\sigma}\right) = 0,90$ où $\Phi$ est la fonction de répartition de la loi normale centrée réduite. On a donc $\frac{t-\mu}{\sigma} = \Phi^{-1}(0,90) \approx 1,28$ et donc $t = \mu + \sigma \Phi^{-1}(0,90) \approx 6280$. 

        Ainsi, pour satisfaire la demande avec une probabilité de \nombre{0,90}, il faut prévoir un stock de \nombre{6280} produits.
    }
}