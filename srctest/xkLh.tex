\uuid{xkLh}
\titre{ Classification linéaire par un perceptron }
\theme{réseau de neurones}
\auteur{ Maxime NGUYEN }
\organisation{ AMSCC }

\contenu{
\question{ Expliquer à l'aide d'un graphique ce que réalise le réseau de neurones suivant :
\begin{center}
	\begin{tikzpicture}[scale=1.5]
		\def\layersep{2cm}
		\tikzstyle{every pin edge}=[thick]
		\tikzstyle{neuron}=[circle,fill=black!25,minimum size=12pt,inner sep=0pt]
		\tikzstyle{entree}=[];
		\tikzstyle{input neuron}=[neuron, fill=green!50];
		\tikzstyle{output neuron}=[neuron, fill=red!50];
		\tikzstyle{hidden neuron}=[neuron, fill=blue!50];
		\tikzstyle{annot} = [text width=4em, text centered]
		
		% Entree
		\node[entree,blue] (E-1) at (-\layersep,-0.5) {$x$};
		\node[entree,blue] (E-2) at (-\layersep,-2.5) {$y$};
		
		% Premiere couche
		\node[input neuron] (I-1) at (0,0) {};
		%\node[input neuron] (I-2) at (0,-1.5) {};
		\node[input neuron] (I-3) at (0,-3) {};
		
		\node[above right=0.8ex,scale=0.7] at (I-1) {$H$};
		%\node[above right=0.8ex,scale=0.7] at (I-2) {$H$};
		\node[below right=0.8ex,scale=0.7] at (I-3) {$H$};
		
		\node[below right=0.8ex,scale=0.7] at (I-1) {};
		\node[below right=0.8ex,scale=0.7] at (I-3) {};
		%\node[below right=0.8ex,scale=0.7] at (I-2) {};
		
		% \node[above right=0.8ex,blue] at (I-1) {$s_1$};
		% \node[above right=0.8ex,blue] at (I-2) {$s_2$};
		% \node[above right=0.8ex,blue] at (I-3) {$s_3$};
		
		%Seconde couche et sortie
		\node[output neuron] (O) at (\layersep,-1.5 cm) {};
		\node[below right=0.8ex,scale=0.7] at (O) {};
		\node[above right=0.8ex,scale=0.7] at (O) {$H$};
		
		% Arrete et poids
		\path[thick] (E-1) edge node[pos=0.8,above,scale=0.7]{$-1$} (I-1) ;
		\path[thick] (E-2) edge node[pos=0.8,above left,scale=0.7]{$1$} (I-1);
		\draw[-o,thick] (I-1) to node[midway,below right,scale=0.7]{$-1$} ++ (-110:0.8);
		
		%\path[thick] (E-1) edge node[pos=0.8,above,scale=0.7]{$1$} (I-2);
		%\path[thick] (E-2) edge node[pos=0.8,above,scale=0.7]{$1$} (I-2);
		%\draw[-o,thick] (I-2) to node[midway,below right,scale=0.7]{$-2$} ++ (-130:0.8);
		
		\path[thick] (E-1) edge node[pos=0.9,above right,scale=0.7]{$1$} (I-3);
		\path[thick] (E-2) edge node[pos=0.8,above,scale=0.7]{$0$} (I-3);
		\draw[-o,thick] (I-3) to node[midway,below right,scale=0.7]{$1$} ++ (-130:0.8);
		
		\path[thick] (I-1) edge node[pos=0.8,above,scale=0.7]{$1$} (O);
		%\path[thick] (I-2) edge node[pos=0.8,below,scale=0.7]{$1$}(O);
		\path[thick] (I-3) edge node[pos=0.8,below,scale=0.7]{$1$}(O);
		\draw[-o,thick] (O) to node[midway,below right,scale=0.7]{$-2$} ++ (-110:0.8) ;
		
		% Sortie
		\draw[->,thick] (O)-- ++(1,0) node[right,blue]{$F(x,y)$};
		
	\end{tikzpicture}  
\end{center} }
}