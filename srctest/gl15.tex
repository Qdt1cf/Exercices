\uuid{gl15}
\titre{Estimation par inégalité}
\theme{probabilités}
\auteur{}
\organisation{AMSCC}
\contenu{

\texte{ Une population de personnes doit voter à l'élection présidentielle. La proportion souhaitant voter pour la candidate Mme A. est inconnue, on la note $p$. Pour approcher cette valeur, on effectue un sondage sur $n$ personnes : l'échantillon est modélisé par une suite de variables aléatoires indépendantes $(X_1,...,X_n)$ suivant chacune une loi de Bernoulli de paramètre $p$ ($X_i=1$ si le $i$-ème individu souhaite voter pour Mme A., $X_i=0$ sinon). 

On note $S_n=\sum_{i=1}^n X_i$ de sorte que $\frac{S_n}{n}$ représente la proportion de personnes votant pour Mme A. dans l'échantillon. }

\begin{enumerate}
	\item \question{ Quelle est la loi suivie par $S_n$ ? }
	\reponse{ $S_n$ suit une loi binomiale $\mathcal{B}(n,p)$. }
	\item \question{ Déterminer l'espérance et la variance de $\frac{S_n}{n}$. }
	\reponse{ On en déduit que $\mathbb{E}(S_n)=np$ et $V(S_n)=np(1-p)$. D'après les propriétés de l'espérance et de la variance, on en déduit que $\mathbb{E}\left(\frac{S_n}{n}\right)=p$ et $V\left(\frac{S_n}{n}\right)=\frac{p(1-p)}{n}$. }
	\item \question{ Soit $\varepsilon >0$. Démontrer que 
	$$\PP\left(\left| \frac{S_n}{n}-p \right| > \varepsilon \right) \leq \frac{1}{4n\varepsilon^2}$$ }
\reponse{ D'après l'inégalité de Bienaymé-Tchebychev, 
	$$\PP\left(\left| \frac{S_n}{n}-p \right| > \varepsilon \right) \leq \frac{p(1-p)}{n\varepsilon^2}$$
	et on conclut en remarquant que $p(1-p) \leq \frac{1}{4}$. }
	\item \question{ Comment choisir la taille de l'échantillon de sorte que le résultat du sondage soit proche de $p$ à $\varepsilon=0.05$ près avec une probabilité supérieure à $95\%$ ? }
	\reponse{ Il faut choisir $n$ tel que $\frac{1}{4} n \varepsilon^2 \leq 0.05$, on peut prendre $n=2000$. }
\end{enumerate}
}