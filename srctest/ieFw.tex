\uuid{ieFw}
\titre{Domaine de convergence de séries sentières}
\theme{séries}
\auteur{}
\organisation{AMSCC}	

\contenu{


	\texte{ 	On considère la série entière de la variable réelle $x$ : 
$$\sum_{n \geq 1} \frac{(-1)^n}{\sqrt{n}+1}x^n$$ }

\begin{enumerate}
	\item\question{  Vérifier que le rayon de convergence est égal à $1$. }
	\reponse{On pose $u_n(x) = \frac{(-1)^n}{\sqrt{n}+1}x^n$ et on utilise le théorème de d'Alembert : 
		\begin{align*}
			\frac{|u_{n+1}(x)|}{|u_n(x)|} &= \frac{ \sqrt{n}+1 }{\sqrt{n+1}+1}\frac{|x^{n+1}|}{|x^{n}|} \\
			& \sim  \frac{\sqrt{n}}{\sqrt{n}} |x| \\
			&\xrightarrow[n\to+\infty]{}  |x|
		\end{align*}
		Donc la série converge si et seulemlent si $|x|<1$. Donc le rayon de convergence est bien $R=1$.
	}
	\item \question{ Déterminer le domaine de convergence de cette série entière. }
	\reponse{On sait que la série converge sur l'intervalle $]-R;R[ = ]-1;1[$. Il reste à étudier le cas où $x=-1$ et $x=1$. 
		
		Or $u_n(-1) = \frac{1}{\sqrt{n}+1} \sum \frac{1}{n^{\frac{1}{2}}}$ : la série $\sum u_n(-1)$ est donc une série à termes positifs et le terme général est équvalent au terme général d'une série de Riemann divergente ($\alpha = 1/2<1$) donc la série $\sum u_n(-1)$ diverge. 
		
		De plus, 	$u_n(1) = \frac{(-1)^n}{\sqrt{n}+1} = (-1)^n a_n$ avec $a_n = \frac{1}{\sqrt{n}+1} >0$ pour tout $n \geq 1$. Donc $u_n(1)$ est le terme général d'une série alternée. Or il est clair que $(a_n)$ est une suite décroissante et $\lim\limits_{n\to+\infty} a_n = 0$ donc d'après le théorème des séries alternées, la série $\sum u_n(1)$ converge.
		
		En définitive, le domaine de convergence est 
		$$D = ]-1;1]$$
	}
\end{enumerate}
}